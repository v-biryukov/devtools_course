\documentclass{article}
\usepackage[english,russian]{babel}
\usepackage{textcomp}
\usepackage{geometry}
  \geometry{left=2cm}
  \geometry{right=1.5cm}
  \geometry{top=1.5cm}
  \geometry{bottom=2cm}
\usepackage{tikz}
\usepackage{multicol}
\usepackage{hyperref}
\usepackage{listings}
\usepackage{pmboxdraw}
\usepackage{fancyvrb}
\usepackage[shortlabels]{enumitem}
\usepackage{upquote}
\usepackage{chngcntr}
\pagenumbering{gobble}
\counterwithout{subsection}{section}

\lstdefinestyle{csMiptCStyle}{
  language=C,
  basicstyle=\linespread{1.1}\ttfamily,
  columns=fixed,
  fontadjust=true,
  basewidth=0.5em,
  keywordstyle=\color{blue}\bfseries,
  commentstyle=\color{gray},
  texcl=true,
  stringstyle=\ttfamily\color{orange!50!black},
  showstringspaces=false,
  numbersep=5pt,
  numberstyle=\tiny\color{black},
  numberfirstline=true,
  stepnumber=1,      
  numbersep=10pt,
  backgroundcolor=\color{white},
  showstringspaces=false,
  captionpos=b,
  breaklines=true
  breakatwhitespace=true,
  xleftmargin=.2in,
  extendedchars=\true,
  keepspaces = true,
  tabsize=4,
  upquote=true,
}


\lstdefinestyle{csMiptCLinesStyle}{
  style=csMiptCStyle,
  frame=lines,
}

\lstdefinestyle{csMiptCBorderStyle}{
  style=csMiptCStyle,
  framexleftmargin=5mm, 
  frame=shadowbox, 
  rulesepcolor=\color{gray}
}


\lstdefinestyle{csMiptBash}{
  	style=csMiptCStyle,
	breaklines=true,
	frame=tb,
	language=bash,
	breakatwhitespace=true,
	alsoletter={*()"'0123456789.},
	alsoother={\{\=\}},
	basicstyle={\ttfamily},
	keywordstyle={\bfseries},
	literate={{=}{{{=}}}1},
	prebreak={\textbackslash},
	sensitive=true,
	stepnumber=1,
	tabsize=4,
	morekeywords={echo, function},
	otherkeywords={-, \{, \}},
	literate={\$\{}{{{{\bfseries{}\$\{}}}}2,
	upquote=true,
	frame=none
}

\lstset{style=csMiptBash}
\lstset{
        literate={~}{{\raisebox{0.5ex}{\texttildelow}}}{1}
}

\renewcommand{\thesubsection}{\arabic{subsection}}
\makeatletter
\def\@seccntformat#1{\@ifundefined{#1@cntformat}%
   {\csname the#1\endcsname\quad}
   {\csname #1@cntformat\endcsname}}
\newcommand\section@cntformat{}     
\newcommand\subsection@cntformat{Задача \thesubsection.\space} 
\newcommand\subsubsection@cntformat{\thesubsubsection.\space}
\makeatother



\begin{document}
\title{Семинар \#9: systemd. Практика. \vspace{-5ex}}\date{}\maketitle
\subsection*{Как сдавать задачи}
Для сдачи ДЗ вам нужно создать репозиторий на GitLab (если он ещё не создан) под названием \texttt{devtools-homework}. Структура репозитория должна иметь вид:
\begin{center}
\begin{BVerbatim}
├── seminar9_systemd/
│   ├── 01.sh
│   ├── 02.sh
│   └── ...
└── ...
\end{BVerbatim}
\end{center}
Для каждой задачи, если не сказано иное, нужно создать один файл решения с расширением \texttt{.sh}. Если в задаче есть подзадачи, то их нужно оформлять внутри каждого из файлов в следующем формате:
\begin{lstlisting}
# Subtask a
systemctl status nginx
# Subtask b
...
\end{lstlisting}
Для каждой подзадачи нужно прописать все команды, которые исполняются в ходе выполнения этой подзадачи. Если в задаче встречается вопрос, то на этот вопрос нужно ответить в комментариях (начинаются с \texttt{\#}) скрипта.



\subsection{Включить сервис}
В данной задаче будет показана простейшая работа с сервисами systemd: просмотр информации о сервисе, включение/отключение сервиса. В качестве примера сервиса был выбран веб сервис nginx. Если у вас не установлен nginx в системе, то нужно будет его установить:
\begin{itemize}
\item Для Debian-based дистрибутивов:
\begin{lstlisting}
$ sudo apt install nginx
\end{lstlisting} 
\item Для RHEL-based дистрибутивов:
\begin{lstlisting}
$ sudo dnf install nginx
\end{lstlisting} 
\end{itemize}
Выполните подзадачи:
\begin{enumerate}[(a)]
\item Проверьте статус сервиса \texttt{nginx}:
\begin{lstlisting}
$ systemctl status nginx
\end{lstlisting}
Ответьте на следующие вопросы о сервисе:
\begin{itemize}
\item Загружен ли этот юнит-файл сервиса в systemd и что на это указывает?
\item Включён ли сервис в автозагрузку и что на это указывает?
\item Активен ли сервис сейчас и что на это указывает?
\end{itemize}

\item Если сервис активен в данный момент, то остановите его командой:
\begin{lstlisting}
$ sudo systemctl stop nginx
\end{lstlisting}
Проверьте теперь статус сервиса nginx. Что изменилось в статусе сервиса? Откройте любой браузер и зайдите по адресу \texttt{http://localhost}. Если сервис nginx не активен, то браузер должен выдать ошибку.

\item Активируйте сервис командой:
\begin{lstlisting}
$ sudo systemctl start nginx
\end{lstlisting}
Проверьте статус сервиса. Откройте браузер и зайдите по адресу \texttt{http://localhost}. Если сервис nginx  активен, то браузер должен выдать страницу веб-сервера nginx по умолчанию (Welcome to nginx!).

\item Используйте
\begin{lstlisting}
$ sudo systemctl enable nginx
\end{lstlisting}
чтобы включить автозапуск сервиса при загрузке системы и
\begin{lstlisting}
$ sudo systemctl disable nginx
\end{lstlisting}
чтобы отключить автозапуск. Проверьте статус сервиса после включения/отключения автозапуска сервиса.

\end{enumerate}



\subsection{Создаём свой простейший сервис}
В данной задаче будет создан простейший сервис. Этот сервис мы назовём \texttt{alpaca}. При старте этого сервиса текущие дата и время будут записываться в файл \texttt{/var/log/alpaca.txt} .

\begin{enumerate}[(a)]
\item В директории \texttt{/var/log/} создайте файл \texttt{alpaca.txt} с владельцем \texttt{root} и группой владельцем \texttt{root}.

\item В директории \texttt{/usr/local/bin} создайте скрипт \texttt{alpaca.sh}, который должен дозаписывать текущую дату и время в конец файла \texttt{/var/log/alpaca.txt}. Дайте этому скрипту права на исполнение.

\item Перейдите в \texttt{/etc/systemd/system} и создайте там юнит-файл сервиса по имени \texttt{alpaca.service}. Содержимое юнит-файла должно быть:
\begin{lstlisting}
[Unit]
Description=Simplest service, writes to the file /var/log/alpaca.txt once

[Service]
ExecStart=/usr/local/bin/alpaca.sh

[Install]
WantedBy=multi-user.target
\end{lstlisting}
В этом файле:
\begin{itemize}
\item \texttt{Description} -- описание сервиса
\item \texttt{ExecStart} -- команда, которая будет исполнятся при старте сервиса
\item \texttt{WantedBy} -- эта строка говорит, к какому таргету будет подключаться данный сервис при вызове \texttt{systemctl enable}. \texttt{multi-user.target} -- это таргет, который исполняется при загрузке системы. Соответственно, эта настройка означает, что сервис \texttt{alpaca} будет запускаться при запуске системы.
\end{itemize}

\item Вызовите
\begin{lstlisting}
$ sudo systemctl daemon-reload
\end{lstlisting}
для того, чтобы systemd перечитал конфигурационные файлы юнитов и добавил наш юнит в систему.

\item Проверьте, что наш сервис добавился, используя:
\begin{lstlisting}
$ systemctl status alpaca
\end{lstlisting}
\item Запустите сервис \texttt{alpaca}
\begin{lstlisting}
$ sudo systemctl start alpaca
\end{lstlisting}
В результате этого должен запуститься сервис \texttt{alpaca}, а это значит, что запустится скрипт \texttt{alpaca.sh}.
Проверьте что в файл \texttt{/var/log/alpaca.txt} будет дозаписываться строка при каждом запуске сервиса \texttt{alpaca}.

\item Включите сервис в автозапуск с помощью \texttt{systemctl enable}.
Перезагрузите виртуальную машину и убедитесь, что сервис запустился при запуске программы, посмотрев содержимое файла \texttt{/var/log/alpaca.txt}.

\item Если посмотреть статус нашего сервиса, то будет отображаться, что он неактивен. Сервис неактивен, так как \texttt{alpaca.sh} быстро отрабатывает и завершается. По умолчанию (если тип сервиса \texttt{simple}) в этом случае он считается неактивным. Измените скрипт \texttt{/usr/local/bin/alpaca.sh} так, чтобы он:
\begin{itemize}
\item Сначала дозаписывал в конец файла \texttt{/var/log/alpaca.txt} строку
\begin{lstlisting}
Started < дата и время >
\end{lstlisting}
\item Затем ожидал 30 секунд
\item Наконец, дозаписывал в конец файла \texttt{/var/log/alpaca.txt} строку
\begin{lstlisting}
Finished < дата и время >
\end{lstlisting}
\end{itemize}
После этого запустите сервис и посмотрите его статус. В течении 30 секунд этот статус должен быть \texttt{active}.

\item Посмотрите на все запущенные сервисы:
\begin{lstlisting}
$ systemctl list-units --type=service
\end{lstlisting}
Если ваш сервис активен, то он отобразится в этом списке.

\end{enumerate}




\subsection{Сервис логов}
Создайте сервис который бы записывал в файл \texttt{/var/log/mylog.log} текущую дату и время, загрузку процессора и количество занятой оперативной памяти каждые 30 секунд (используйте \texttt{sleep 30}). Сервис должен быть активен пока его не остановят. 

Для решения этой задачи создайте файл \texttt{0\thesubsection.sh} в котором бы содержались все команды нужные для решения данной задачи. Также создайте файл \texttt{u0\thesubsection.service}, который бы представлял собой юнит-файл данного сервиса.


\subsection{Сервис логов по таймеру}
Создайте сервис который бы записывал в файл \texttt{/var/log/mylog2.log} текущую дату и время, загрузку процессора и и количество занятой оперативной памяти каждые две минуты, когда значение минут чётно, а значение секунд равно нулю. В этой задаче не используйте \texttt{sleep}. Вместо этого используйте systemd timers.

Для решения этой задачи создайте файл \texttt{0\thesubsection.sh} в котором бы содержались все команды нужные для решения данной задачи. Также создайте файл \texttt{u0\thesubsection.service}, который бы представлял собой юнит-файл данного сервиса.


\subsection{Бессмертный файл}
Создайте файл \texttt{immortal.txt} с содержимым \texttt{"I am immortal"}. Создайте сервис, который будет следить за этим файлом. В случае удаления этого файла, сервис должен сразу же восстанавливать его с тем же содержимым. В случае изменения содержимого файла, его содержимое должно сразу меняться на изначальное \texttt{"I am immortal"}. Решите эту задачу двумя способами:
\begin{enumerate}[(a)]
\item Неэффективный подход с использованием одного сервиса, который будет запускать демона. Этот демон должен будет проверять состояние файла каждые пять секунд и восстанавливать его.

Для решения этой подзадачи создайте файл \texttt{0\thesubsection a.sh} в котором бы содержались все команды нужные для решения. Также создайте файл \texttt{u0\thesubsection a.service}, который бы представлял собой юнит-файл данного сервиса.

\item Правильный подход с использованием дополнительного path-юнита.

Для решения этой подзадачи создайте файл \texttt{0\thesubsection b.sh} в котором бы содержались все команды нужные для решения задачи. Также создайте файл \texttt{u0\thesubsection b.service}, который бы представлял собой юнит-файл данного сервиса и файл \texttt{u0\thesubsection b.path}, который бы представлял собой файл path-юнита.
\end{enumerate}





\end{document}
