\documentclass{article}
\usepackage[english,russian]{babel}
\usepackage{textcomp}
\usepackage{geometry}
  \geometry{left=2cm}
  \geometry{right=1.5cm}
  \geometry{top=1.5cm}
  \geometry{bottom=2cm}
\usepackage{tikz}
\usepackage{multicol}
\usepackage{hyperref}
\usepackage{listings}
\usepackage{pmboxdraw}
\usepackage{fancyvrb}
\usepackage[shortlabels]{enumitem}
\usepackage{upquote}
\usepackage{chngcntr}
\pagenumbering{gobble}
\counterwithout{subsection}{section}

\lstdefinestyle{csMiptCStyle}{
  language=C,
  basicstyle=\linespread{1.1}\ttfamily,
  columns=fixed,
  fontadjust=true,
  basewidth=0.5em,
  keywordstyle=\color{blue}\bfseries,
  commentstyle=\color{gray},
  texcl=true,
  stringstyle=\ttfamily\color{orange!50!black},
  showstringspaces=false,
  numbersep=5pt,
  numberstyle=\tiny\color{black},
  numberfirstline=true,
  stepnumber=1,      
  numbersep=10pt,
  backgroundcolor=\color{white},
  showstringspaces=false,
  captionpos=b,
  breaklines=true
  breakatwhitespace=true,
  xleftmargin=.2in,
  extendedchars=\true,
  keepspaces = true,
  tabsize=4,
  upquote=true,
}


\lstdefinestyle{csMiptCLinesStyle}{
  style=csMiptCStyle,
  frame=lines,
}

\lstdefinestyle{csMiptCBorderStyle}{
  style=csMiptCStyle,
  framexleftmargin=5mm, 
  frame=shadowbox, 
  rulesepcolor=\color{gray}
}


\lstdefinestyle{csMiptBash}{
  	style=csMiptCStyle,
	breaklines=true,
	frame=tb,
	language=bash,
	breakatwhitespace=true,
	alsoletter={*()"'0123456789.},
	alsoother={\{\=\}},
	basicstyle={\ttfamily},
	keywordstyle={\bfseries},
	literate={{=}{{{=}}}1},
	prebreak={\textbackslash},
	sensitive=true,
	stepnumber=1,
	tabsize=4,
	morekeywords={echo, function},
	otherkeywords={-, \{, \}},
	literate={\$\{}{{{{\bfseries{}\$\{}}}}2,
	upquote=true,
	frame=none
}

\lstset{style=csMiptBash}
\lstset{
        literate={~}{{\raisebox{0.5ex}{\texttildelow}}}{1}
}

\renewcommand{\thesubsection}{\arabic{subsection}}
\makeatletter
\def\@seccntformat#1{\@ifundefined{#1@cntformat}%
   {\csname the#1\endcsname\quad}
   {\csname #1@cntformat\endcsname}}
\newcommand\section@cntformat{}     
\newcommand\subsection@cntformat{Задача \thesubsection.\space} 
\newcommand\subsubsection@cntformat{\thesubsubsection.\space}
\makeatother



\begin{document}
\title{Семинар \#9: systemd. Практика. \vspace{-5ex}}\date{}\maketitle
\subsection*{Как сдавать задачи}
Для сдачи ДЗ вам нужно создать репозиторий на GitLab (если он ещё не создан) под названием \texttt{devtools-homework}. Структура репозитория должна иметь вид:
\begin{center}
\begin{BVerbatim}
├── seminar9_systemd/
│   ├── 01.sh
│   ├── 02.sh
│   └── ...
└── ...
\end{BVerbatim}
\end{center}
Для каждой задачи, если в самой задаче не сказано иное, нужно создать 1 скрипт с расширением \texttt{.sh} и шебангом в начале скрипта. Если задача делится на подзадачи нужно, если в самой задаче не сказано иное, создать скрипт для каждой подзадачи. Названия файлов решений для всех задач/подзадач должны начинаться с номера задачи, например \texttt{01.sh} или \texttt{04b.sh}, даже если в условии задачи используется другое имя для скрипта.\\



\subsection{Включить сервис}
\begin{enumerate}[(a)]
\item Проверьте, установлен ли на вашей системе http-сервер nginx, выполнив команду
\begin{lstlisting}
$ apt list --installed nginx 
\end{lstlisting}
Если пакет не установлен, то программа ничего не напечатает.

\item Проверьте, есть ли в системе сервис \texttt{nginx}:
\begin{lstlisting}
$ systemctl status nginx
\end{lstlisting}

\item В Linux откройте браузер и зайдите по адресу \texttt{http://localhost}. Если страница не открывается, то либо nginx не установлен, либо сервис не запущен.

\item Если \texttt{nginx} не был установлен, то установите его:
\begin{lstlisting}
$ sudo apt install nginx
\end{lstlisting}
Используйте вместо \texttt{apt} пакетный менеджер вашей системы.

\item Проверьте теперь статус сервиса nginx. Затем запустите сервис с помощью команды:
\begin{lstlisting}
$ sudo systemctl start nginx
\end{lstlisting}

\item В браузер и зайдите по адресу \texttt{http://localhost}. Теперь сервис должен работать и в браузере для вас (и для всех, кто зайдёт на ваш сервер по \texttt{http}) должно отображаться сообщение от nginx.

\item Остановите сервис
\begin{lstlisting}
$ sudo systemctl stop nginx
\end{lstlisting}
и снова зайдите в браузере на локальный адрес.

\item Используйте
\begin{lstlisting}
$ sudo systemctl enable nginx
\end{lstlisting}
чтобы сервис включался при запуске системы и
\begin{lstlisting}
$ sudo systemctl disable nginx
\end{lstlisting}
чтобы не включался. Проверьте статус сервиса после включения/отключения автозапуска сервиса.

\end{enumerate}



\subsection{Создаём свой простейший сервис}
\begin{enumerate}[(a)]
\item В вашей домашней директории создайте файл \texttt{a.txt} и скрипт \texttt{a.sh}, который должен дозаписывать текущую дату и время в конец файла \texttt{a.txt}. Добавьте скрипту права на исполнение для всех пользователей. В скрипте нужно использовать полный путь до файла \texttt{a.txt}, так как предполагается, что это файл может запускать \texttt{root}, у которого домашняя директория будет отличаться от вашей.

\item Перейдите в \texttt{/etc/systemd/system} и создайте там юнит-файл сервиса по имени \texttt{alpaca.service}. Нужно будет использовать \texttt{sudo}. Содержимое должно быть:
\begin{lstlisting}
[Unit]
Description=Simplest server, writes to the file a.txt once

[Service]
ExecStart=/home/<user>/a.sh
User=<user>

[Install]
WantedBy=multi-user.target
\end{lstlisting}
Где вместо \texttt{<user>} подставьте имя пользователя, которого вы используете. В этом файле:
\begin{itemize}
\item \texttt{Description} -- описание сервиса
\item \texttt{ExecStart} -- bash-скрипт, который будет исполнятся при старте сервиса
\item \texttt{User} -- от имени этого пользователя будет запускаться скрипт (по умолчанию будет \texttt{root}).
\item \texttt{WantedBy} -- эта строка говорит, к какому таргету будет подключаться данный сервис при вызове \texttt{systemctl enable}. \texttt{multi-user.target} -- это таргет, который исполняется при загрузке системы. Соответственно, эта настройка означает, что скрипт \texttt{a.sh} будет исполнятся при запуске системы.
\end{itemize}

\item Вызовите
\begin{lstlisting}
$ sudo systemctl daemon-reload
\end{lstlisting}
для того, чтобы systemd перечитал конфигурационные файлы юнитов и добавил наш юнит в систему.

\item Проверьте, что наш сервис добавился, используя:
\begin{lstlisting}
$ systemctl status alpaca
\end{lstlisting}

\item Запустите сервис \texttt{alpaca}
\begin{lstlisting}
$ systemctl start alpaca
\end{lstlisting}
В результате этого должен запуститься сервис \texttt{alpaca}, а это значит, что запустится скрипт \texttt{a.sh}.
Проверьте что в файл \texttt{a.txt} будет дозаписываться строка при каждом запуске сервиса \texttt{alpaca}.

\item Включите сервис в автозапуск с помощью \texttt{systemctl enable}.
Перезагрузите виртуальную машину и убедитесь, что сервис запустился при запуске программы, посмотрев содержимое файла \texttt{a.txt}.

\item Если посмотреть статус нашего сервиса, то будет отображаться, что он неактивен. Сервис неактивен, так как скрипт \texttt{a.sh} быстро отрабатывает и завершается. Измените скрипт \texttt{a.sh} так, чтобы он 
\begin{itemize}
\item Сначала дозаписывал в конец файла \texttt{a.txt} строку
\begin{lstlisting}
Started <дата и время>
\end{lstlisting}
\item Затем ожидал 30 секунд
\item Наконец, дозаписывал в конец файла \texttt{a.txt} строку
\begin{lstlisting}
Finished <дата и время>
\end{lstlisting}
\end{itemize}
После этого запустите сервис и посмотрите его статус. В течении 30 секунд этот статус должен быть \texttt{active}.

\item Посмотрите на все запущенные сервисы:
\begin{lstlisting}
$ systemctl list-units --type=service
\end{lstlisting}
Если ваш сервис активен, то он отобразиться в этом списке.

\end{enumerate}




\subsection{Сервис логов}
Создайте сервис который бы записывал в файл \texttt{/var/log/mylogs} текущую дату и время, загрузку процессора и и количество занятой оперативной памяти каждые 30 секунд (используйте \texttt{sleep 30}). Сервис должен быть активен пока его не остановят. Используйте \texttt{Restart=always} в юнит файле.


\subsection{Сервис логов по таймеру}
Создайте сервис который бы записывал в файл \texttt{/var/log/mylogs} текущую дату и время, загрузку процессора и и количество занятой оперативной памяти каждый день ровно в 12:00 дня. В этой задаче не используйте \texttt{sleep}. Вместо этого используйте systemd timers.



\subsection{Бессмертный файл}
Создайте файл \texttt{immortal.txt} с содержимым \texttt{"I am immortal"}. Создайте сервис, который будет следить за этим файлом. В случае удаления этого файла, сервис должен сразу же восстанавливать его с тем же содержимым. В случае изменения содержимого файла, его содержимое должно сразу меняться на изначальное \texttt{"I am immortal"}. Решите эту задачу двумя способами:
\begin{enumerate}[(a)]
\item С использованием одного сервиса, который будет запускать демона. Этот демон должен будет проверять состояние файла каждые пять секунд и восстанавливать его.
\item С использованием дополнительного path юнита.

\end{enumerate}




\end{document}
