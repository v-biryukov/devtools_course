\documentclass{article}
\usepackage[english,russian]{babel}
\usepackage{textcomp}
\usepackage{geometry}
  \geometry{left=2cm}
  \geometry{right=1.5cm}
  \geometry{top=1.5cm}
  \geometry{bottom=2cm}
\usepackage{tikz}
\usepackage{multicol}
\usepackage{hyperref}
\usepackage{listings}
\usepackage{pmboxdraw}
\usepackage{fancyvrb}
\usepackage[shortlabels]{enumitem}
\usepackage{upquote}
\usepackage{chngcntr}
\pagenumbering{gobble}
\counterwithout{subsection}{section}

\lstdefinestyle{csMiptCStyle}{
  language=C,
  basicstyle=\linespread{1.1}\ttfamily,
  columns=fixed,
  fontadjust=true,
  basewidth=0.5em,
  keywordstyle=\color{blue}\bfseries,
  commentstyle=\color{gray},
  texcl=true,
  stringstyle=\ttfamily\color{orange!50!black},
  showstringspaces=false,
  numbersep=5pt,
  numberstyle=\tiny\color{black},
  numberfirstline=true,
  stepnumber=1,      
  numbersep=10pt,
  backgroundcolor=\color{white},
  showstringspaces=false,
  captionpos=b,
  breaklines=true
  breakatwhitespace=true,
  xleftmargin=.2in,
  extendedchars=\true,
  keepspaces = true,
  tabsize=4,
  upquote=true,
}


\lstdefinestyle{csMiptCLinesStyle}{
  style=csMiptCStyle,
  frame=lines,
}

\lstdefinestyle{csMiptCBorderStyle}{
  style=csMiptCStyle,
  framexleftmargin=5mm, 
  frame=shadowbox, 
  rulesepcolor=\color{gray}
}


\lstdefinestyle{csMiptBash}{
  	style=csMiptCStyle,
	breaklines=true,
	frame=tb,
	language=bash,
	breakatwhitespace=true,
	alsoletter={*()"'0123456789.},
	alsoother={\{\=\}},
	basicstyle={\ttfamily},
	keywordstyle={\bfseries},
	literate={{=}{{{=}}}1},
	prebreak={\textbackslash},
	sensitive=true,
	stepnumber=1,
	tabsize=4,
	morekeywords={echo, function},
	otherkeywords={-, \{, \}},
	literate={\$\{}{{{{\bfseries{}\$\{}}}}2,
	upquote=true,
	frame=none
}

\lstset{style=csMiptBash}
\lstset{
        literate={~}{{\raisebox{0.5ex}{\texttildelow}}}{1}
}

\renewcommand{\thesubsection}{\arabic{subsection}}
\makeatletter
\def\@seccntformat#1{\@ifundefined{#1@cntformat}%
   {\csname the#1\endcsname\quad}
   {\csname #1@cntformat\endcsname}}
\newcommand\section@cntformat{}     
\newcommand\subsection@cntformat{Задача \thesubsection.\space} 
\newcommand\subsubsection@cntformat{\thesubsubsection.\space}
\makeatother



\begin{document}
\title{Семинар \#8: Процессы. Практика. \vspace{-5ex}}\date{}\maketitle
\subsection*{Как сдавать задачи}
Для сдачи ДЗ вам нужно создать репозиторий на GitLab (если он ещё не создан) под названием \texttt{devtools-homework}. Структура репозитория должна иметь вид:
\begin{center}
\begin{BVerbatim}
├── seminar8_processes/
│   ├── 01.sh
│   ├── 02.sh
│   └── ...
└── ...
\end{BVerbatim}
\end{center}
Для каждой задачи, если в самой задаче не сказано иное, нужно создать 1 скрипт с расширением \texttt{.sh} и шебангом в начале скрипта. Если задача делится на подзадачи нужно, если в самой задаче не сказано иное, создать скрипт для каждой подзадачи. Названия файлов решений для всех задач/подзадач должны начинаться с номера задачи, например \texttt{01.sh} или \texttt{04b.sh}, даже если в условии задачи используется другое имя для скрипта.  \\


\subsection{Запуск в фоновом режиме}
\begin{enumerate}[(a)]
\item Создадим скрипт \texttt{counter.sh} который будет печатать на экран возрастающие целые числа с шагом в пол секунды.
\begin{lstlisting}
#!/bin/bash
i=0
while true; do
    echo "$i"
    sleep 0.5
    ((i += 1))
done
\end{lstlisting}
Напишите или скачайте этот скрипт с репозитория и дайте ему права на исполнение.

\item Запустите скрипт:
\begin{lstlisting}
$ ./counter.sh
\end{lstlisting}
В результате на экран будут печататься числа. Bash будет занят выводом чисел на экран, поэтому в данной оболочке мы ничего сделать не сможем, пока не завершим программу. Завершите исполнение программы, используя \texttt{Ctrl-C}.


\item Запустите скрипт в фоновом режиме:
\begin{lstlisting}
$ ./counter.sh &
\end{lstlisting}
Bash запустит программу в фоновом режиме. В этом случае bash будет свободен и вы сможете выполнять другие команды. Правда, так как \texttt{counter.sh} всё ещё выводит числа на экран, то работа с оболочкой может быть затруднена. Завершить программу, используя \texttt{Ctrl-C} в данном случае невозможно. Чтобы завершить процесс, его нужно сначала перевести из фонового (background) режима в передний (foreground) режим, используя команду \texttt{fg} и после этого завершить с помощью \texttt{Ctrl-C}.

 
\item Запустите скрипт с выводом в файл: 
\begin{lstlisting}
$ ./counter.sh > a.txt
\end{lstlisting}
В результате в файл \texttt{a.txt} будут записываться числа. Bash будет занят записью чисел в файл, поэтому в данной оболочке мы ничего сделать не сможем. Но можно проверить, что числа печатаются в файл, используя другой терминал. Откройте другой терминал, зайдите в ту же директорию и проверьте, что в файл \texttt{a.txt} в данный момент производится запись, несколько раз вызвав \texttt{cat}. Также можно использовать команду:
\begin{lstlisting}
$ watch -n 2 cat a.txt
\end{lstlisting}
которая будет автоматически повторять команду \texttt{cat a.txt} каждые 2 секунды. Но, так как запись происходит в конец файла, то скорей всего лучше использовать команду \texttt{tail} вместо \texttt{cat}.
\begin{lstlisting}
$ watch -n 2 tail a.txt
\end{lstlisting}
Задача по выводу последних строк изменяемого файла используется настолько часто, что команде \texttt{tail} дали специальную опцию для этого случая:
\begin{lstlisting}
$ tail -f a.txt
\end{lstlisting}
Которая выводит последние строки файла, после изменения этого файла.

Закройте второй терминал, вернитесь на первый и завершите процесс \texttt{counter.sh}.


\item Запустите скрипт с выводом в файл в фоновом режиме:
\begin{lstlisting}
$ ./counter.sh > a.txt &
\end{lstlisting}
В результате в файл \texttt{a.txt} будут записываться числа. В этом случае bash будет свободен и вы сможете выполнять другие команды. Проверьте, что программа \texttt{counter} работает и в файл записываются числа, используя команду \texttt{tail -f}. 
Чтобы завершить процесс, его нужно перевести из фонового (background) режима в передний (foreground) режим, используя команду \texttt{fg} и после этого завершить с помощью \texttt{Ctrl-C}.

\item Запустите скрипт с выводом в файл в фоновом режиме:
\begin{lstlisting}
$ ./counter.sh > a.txt &
\end{lstlisting}
Выполните команду \texttt{ps}, которая покажет все программы, запущенные в данном терминале. Скорей всего на экран выведется информация о процессах \texttt{bash} (оболочка), \texttt{counter.sh}, \texttt{sleep} (запушенный внутри \texttt{counter.sh}) и только что запущенный \texttt{ps}. Для каждого процесса, будет выведен его идентификатор PID (\textit{process identifier}). Используйте полученный PID процесса \texttt{counter.sh} и команду \texttt{kill}, чтобы завершить процесс. Проверьте, что процесс завершился и запись в \texttt{a.txt} больше не производится. Ещё раз выполните команду \texttt{ps} и убедитесь, что процессы \texttt{counter.sh} и \texttt{sleep} отсутствуют в выводе.
\end{enumerate}


\subsection{Приостановка процесса}

\begin{enumerate}[(a)]
\item \textbf{Простая приостановка}
\begin{itemize}[leftmargin=1mm,label={--}]
\item Запустите скрипт \texttt{counter.sh}:
\begin{lstlisting}
$ ./counter.sh
\end{lstlisting}
\item Нажмите комбинацию клавиш \texttt{Ctrl-Z}, чтобы приостановить процесс. Теперь процесс приостановился и ничего не выводит на экран.
\item Для возобновления работы процесса на переднем плане используйте команду \texttt{fg} (\textit{foreground process}).
\item Завершите процесс, используя \texttt{Ctrl-C}.
\end{itemize}


\item \textbf{Передний и фоновый режимы}
\begin{itemize}[leftmargin=1mm,label={--}]
\item Запустите скрипт \texttt{counter.sh} с записью в файл:
\begin{lstlisting}
$ ./counter.sh > a.txt
\end{lstlisting}
\item Откройте новый терминал и запустите на нём просмотр конца файла \texttt{a.txt}, используя \texttt{tail -F a.txt}. Убедитесь, что в файл происходит запись. Не закрывайте этот терминал, на нём мы будем проверять, что наш процесс работает.
\item Перейдите на основной терминал. Нажмите комбинацию клавиш \texttt{Ctrl-Z}, чтобы приостановить процесс. Теперь процесс приостановился и ничего не будет записывать в файл. Убедитесь в этом на втором терминале. 
\item Используйте команду \texttt{fg}, чтобы возобновить работу процесса на переднем плане. Убедитесь, что процесс начал работу, проверив вывод команды \texttt{tail} на втором терминале.
\item Снова приостановите процесс, используя \texttt{Ctrl-Z}. Используйте команду \texttt{bg}, чтобы возобновить процесс, но теперь в фоновом режиме. Убедитесь, что процесс начал работать, посмотрев вывод на другом терминале.
\item Переведите процесс на передний план и завершите его.
\end{itemize}


\newpage
\item \textbf{Список задач}
\begin{itemize}[leftmargin=1mm,label={--}]
\item Завершите все предыдущие процессы и запустите скрипт три раза в фоновом режиме:
\begin{lstlisting}
$ ./counter.sh > a.txt &
$ ./counter.sh > b.txt &
$ ./counter.sh > с.txt &
\end{lstlisting}
\item Убедитесь, что все задачи работают в фоновом режиме, проверив запись в соответствующие файлы.

\item Выполните команду \texttt{jobs}, чтобы посмотреть все фоновые или приостановленные задачи в данной оболочке. Символом \texttt{+} будет отображаться последняя добавленная в \texttt{jobs} задача, а символом \texttt{-} предпоследняя.

\item Переведите задачу записи в \texttt{a.txt} на передний план, используя команду:
\begin{lstlisting}
$ fg %N   # вместо N нужно подставить номер задачи в jobs
\end{lstlisting}
Остановите эту задачу с помощью \texttt{Ctrl-Z} и снова посмотрите на список задач.


\item Теперь задача записи в \texttt{a.txt} приостановлена. Обратите внимание, что рядом с задачей записи в \texttt{a.txt} стоит символ \texttt{+}, так как эта последняя добавленная задача. Возобновите работу этой программы в фоновом режиме, используя:
\begin{lstlisting}
$ bg %N   # вместо N нужно подставить номер задачи в jobs
\end{lstlisting}
или, так как эта задача является последней добавленной, то можно использовать \texttt{bg} без указания номера. В этом случае операция будет применена к последней добавленной задаче. Снова посмотрите на все задачи, используя \texttt{jobs}.

\item Используйте команду
\begin{lstlisting}
$ kill %N   # вместо N нужно подставить номер задачи в jobs
\end{lstlisting}
чтобы завершить задачу записи в файл \texttt{c.txt}. Снова посмотрите на все задачи.

\item Используйте команду
\begin{lstlisting}
$ kill -STOP %N   # вместо N нужно подставить номер задачи в jobs
\end{lstlisting}
чтобы остановить задачу записи в файл \texttt{b.txt}. Посмотрите на все задачи. Используйте \texttt{bg} или 
\begin{lstlisting}
$ kill -CONT %N   # вместо N нужно подставить номер задачи в jobs
\end{lstlisting}
чтобы возобновить приостановленную задачу.
\end{itemize}

\subsection{Список процессов}
\begin{itemize}
\item Напечатайте все процессы данного терминала, используя \texttt{ps}.
\item Напечатайте все процессы, используя \texttt{ps -ef}.
\item Напечатайте все процессы, показав информацию только о uid, pid, ppid, pcpu, pmem, time.
\item Напечатайте все процессы, показав информацию только о uid, pid, ppid, pcpu, pmem, time и отсортировав по полю pmem.
\end{itemize}


\subsection{Родословная}
Напишите скрипт, который будет принимать через аргумент PID процесса и выводить рекурсивно PID его родителей и команды, с помощью которых были все процессы запущены. Последний родитель будет иметь имя \texttt{systemd} (или \texttt{/sbin/init}) и иметь PID равный 1.


\subsection{Ловец сигналов}
Напишите скрипт, который будет ловить сигналы и при поимке сигнала писать сообщение
\begin{lstlisting}
I caught названиесигнала
\end{lstlisting}
Протестируйте скрипт, отправив ему сигналы \texttt{SIGINT},  \texttt{SIGTERM},  \texttt{SIGHUP},  \texttt{SIGTSTP},  \texttt{SIGCONT},  \texttt{SIGQUIT}.



\end{enumerate}
\end{document}
