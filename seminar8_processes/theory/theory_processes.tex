\documentclass{article}
\usepackage[english,russian]{babel}
\usepackage{textcomp}
\usepackage{geometry}
  \geometry{left=2cm}
  \geometry{right=1.5cm}
  \geometry{top=1.5cm}
  \geometry{bottom=2cm}
\usepackage{tikz}
\usepackage{multicol}
\usepackage{hyperref}
\usepackage{listings}
\pagenumbering{gobble}

\lstdefinestyle{csMiptCppStyle}{
  language=C++,
  basicstyle=\linespread{1.1}\ttfamily,
  columns=fixed,
  fontadjust=true,
  basewidth=0.5em,
  keywordstyle=\color{blue}\bfseries,
  commentstyle=\color{gray},
  texcl=true,
  stringstyle=\ttfamily\color{orange!50!black},
  showstringspaces=false,
  numbersep=5pt,
  numberstyle=\tiny\color{black},
  numberfirstline=true,
  stepnumber=1,      
  numbersep=10pt,
  backgroundcolor=\color{white},
  showstringspaces=false,
  captionpos=b,
  breaklines=true
  breakatwhitespace=true,
  xleftmargin=.2in,
  extendedchars=\true,
  keepspaces = true,
  tabsize=4,
  upquote=true,
}


\lstdefinestyle{csMiptCppLinesStyle}{
  style=csMiptCppStyle,
  frame=lines,
}

\lstdefinestyle{csMiptCppBorderStyle}{
  style=csMiptCppStyle,
  framexleftmargin=5mm, 
  frame=shadowbox, 
  rulesepcolor=\color{gray}
}


\lstdefinestyle{csMiptBash}{
breaklines=true,
frame=tb,
language=bash,
breakatwhitespace=true,
alsoletter={*()"'0123456789.},
alsoother={\{\=\}},
basicstyle={\ttfamily},
keywordstyle={\bfseries},
literate={{=}{{{=}}}1},
prebreak={\textbackslash},
sensitive=true,
stepnumber=1,
tabsize=4,
morekeywords={echo, function},
otherkeywords={-, \{, \}},
literate={\$\{}{{{{\bfseries{}\$\{}}}}2,
upquote=true,
frame=none
}


\lstset{style=csMiptCppLinesStyle}
\lstset{literate={~}{{\raisebox{0.5ex}{\texttildelow}}}{1}}


\renewcommand{\thesection}{\arabic{section}}
\makeatletter
\def\@seccntformat#1{\@ifundefined{#1@cntformat}%
   {\csname the#1\endcsname\quad}%    default
   {\csname #1@cntformat\endcsname}}% enable individual control
\newcommand\section@cntformat{Часть \thesection:\space}
\makeatother


\renewcommand{\arraystretch}{1.3}

\begin{document}
\title{Семинар \#8: Процессы \vspace{-5ex}}\date{}\maketitle


\section*{Программа \texttt{ps}}




\begin{flushleft}
\begin{tabular}{ l | l }
Команда 	    	& Действие\\ \hline
\texttt{ps}	    	& Показывает процессы текущего терминала.\\
\texttt{ps -e}	    & Показать все процессы в системе.\\
\texttt{ps -f}	    & Показывает расширенную информацию: uid, pid, ppid, c, stime, tty, time, cmd.\\
\texttt{ps -ef}	    & Показать все процессы в расширенном формате.\\
\texttt{ps -u alice,bob} & Показать процессы пользователей alice и bob. \\
\texttt{ps -p 100,200} & Показать процессы, со значениями PID равными 100 или 200. \\ \hline
\texttt{ps -o pid,user,cmd} & Вывод в пользовательском формате. Для каждого процесса покажет идентификатор \\
                            & процесса (\texttt{pid}), имя пользователя (\texttt{user}) и команду которой был запущен процесс (\texttt{cmd}).\\
\texttt{ps -o pid=,user=,cmd=} & То же, но не выведет первую строку с описанием столбцов.\\
\texttt{ps -C top}             & Все процессы с именем программы \texttt{top}.\\
\texttt{ps -{}-sort time}        & Отсортировать по ключевому слову \texttt{time}.\\
\texttt{ps -{}-sort -time}       & Сортировка в обратном порядке по ключевому слову \texttt{time}.\\ \hline
\texttt{ps aux}                  & Все процессы с кл. словами: user, pid, pcpu, pmem, vsz, rss, tty, stat, start, time, cmd
\end{tabular}
\end{flushleft}


\subsection*{Распространённые ключевые слова для \texttt{ps -o} и \texttt{ps -{}-sort}}
\begin{flushleft}
\begin{tabular}{ l | l }
Ключевое слово 	    		& Расшифровка \\ \hline
\texttt{pid}	    		& Идентификатор процесса (Process ID).\\
\texttt{ppid}	    		& Идентификатор родительского процесса (Parent Process ID).\\
\texttt{user}	    		& Имя пользователя, запустившего процесс.\\
\texttt{uid}	    		& Идентификатор пользователя.\\
\texttt{comm} 	 			& Имя команды (без аргументов). \\
\texttt{cmd}				& Полная командная строка (включая аргументы).\\
\texttt{tty}				& Управляющий терминал.\\
\texttt{time}				& Совокупное время ЦП (CPU time).\\
\texttt{etime}				& Прошедшее время с момента запуска процесса (Elapsed time).\\
\texttt{c}					& Использование ЦП (в минутах).\\
\texttt{pcpu}				& Процент использования ЦП.\\
\texttt{pmem}				& Процент использования физической памяти.\\
\texttt{vsz}				& Размер виртуальной памяти в КиБ (Virtual Size).\\
\texttt{rss}				& Размер резидентного набора в КиБ (Resident Set Size).\\
\texttt{stat}				& Текущее состояние процесса (например, R - выполняется, S - спит).\\
\texttt{ni}					& Значение nice (приоритет).\\
\texttt{priority}			& Приоритет ядра.\\
\texttt{start}				& Время или дата запуска процесса.\\
\end{tabular}
\end{flushleft}

\noindent Больше опций и ключевых слов можно найти в \texttt{man ps}.




\section*{Сигналы}


\subsection*{Распространённые сигналы}
\begin{flushleft}
\begin{tabular}{ c | c | c | l }
Номер 	& Название 	& Действие по умолчанию  & Примечание\\ \hline
2		& SIGINT	& Завершение 			 & Посылается, когда пользователь нажимает \texttt{Ctrl+C}.\\
										 & & & Используется для корректного завершения процесса.\\ \hline
15		& SIGTERM	& Завершение 			 & Стандартный запрос на корректное завершение процесса.\\ \hline
9		& SIGKILL	& Немедленное завершение & Принудительное завершение, не дающее программе  \\
										 & & & возможности сохранить данные или корректно освободить  \\
										 & & & ресурсы. Не может быть перехвачен процессом.\\ \hline
1		& SIGHUP	& Завершение 			 & Посылается при закрытии терминала.\\ \hline
20		& SIGTSTP	& Приостановка 			 & Посылается, когда пользователь нажимает \texttt{Ctrl+Z}.\\
										 & & & Приостанавливает процесс, который затем может быть \\
										 & & & возобновлен командами \texttt{fg} или \texttt{bg}.\\ \hline
19		& SIGSTOP	& Приостановка 			 & Принудительно приостанавливает выполнение процесса.\\
										 & & & Не может быть проигнорирован или перехвачен процессом \\ \hline
18		& SIGCONT	& Возобновление 	     & Возобновляет выполнение процесса, который был остановлен.\\
										 & & & сигналами SIGSTOP или SIGTSTP. Используются командами \\
										 & & & \texttt{fg} и \texttt{bg} для возобновления работы процесса. \\ \hline
3		& SIGQUIT	& Создание дампа ядра 	 & Посылается, когда пользователь нажимает \texttt{Ctrl+\textbackslash}.\\
				    & &	и завершение	     & Корректно завершает процесс. Дополнительно создает файл \\ 
				                         & & & дампа памяти, полезный для отладки\\ \hline								
\end{tabular}
\end{flushleft}


\end{document}
