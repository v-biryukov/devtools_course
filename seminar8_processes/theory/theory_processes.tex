\documentclass{article}
\usepackage[english,russian]{babel}
\usepackage{textcomp}
\usepackage{geometry}
  \geometry{left=2cm}
  \geometry{right=1.5cm}
  \geometry{top=1.5cm}
  \geometry{bottom=2cm}
\usepackage{tikz}
\usepackage{multicol}
\usepackage{hyperref}
\usepackage{listings}
\pagenumbering{gobble}

\lstdefinestyle{csMiptCppStyle}{
  language=C++,
  basicstyle=\linespread{1.1}\ttfamily,
  columns=fixed,
  fontadjust=true,
  basewidth=0.5em,
  keywordstyle=\color{blue}\bfseries,
  commentstyle=\color{gray},
  texcl=true,
  stringstyle=\ttfamily\color{orange!50!black},
  showstringspaces=false,
  numbersep=5pt,
  numberstyle=\tiny\color{black},
  numberfirstline=true,
  stepnumber=1,      
  numbersep=10pt,
  backgroundcolor=\color{white},
  showstringspaces=false,
  captionpos=b,
  breaklines=true
  breakatwhitespace=true,
  xleftmargin=.2in,
  extendedchars=\true,
  keepspaces = true,
  tabsize=4,
  upquote=true,
}


\lstdefinestyle{csMiptCppLinesStyle}{
  style=csMiptCppStyle,
  frame=lines,
}

\lstdefinestyle{csMiptCppBorderStyle}{
  style=csMiptCppStyle,
  framexleftmargin=5mm, 
  frame=shadowbox, 
  rulesepcolor=\color{gray}
}


\lstdefinestyle{csMiptBash}{
breaklines=true,
frame=tb,
language=bash,
breakatwhitespace=true,
alsoletter={*()"'0123456789.},
alsoother={\{\=\}},
basicstyle={\ttfamily},
keywordstyle={\bfseries},
literate={{=}{{{=}}}1},
prebreak={\textbackslash},
sensitive=true,
stepnumber=1,
tabsize=4,
morekeywords={echo, function},
otherkeywords={-, \{, \}},
literate={\$\{}{{{{\bfseries{}\$\{}}}}2,
upquote=true,
frame=none
}


\lstset{style=csMiptCppLinesStyle}
\lstset{literate={~}{{\raisebox{0.5ex}{\texttildelow}}}{1}}


\renewcommand{\thesection}{\arabic{section}}
\makeatletter
\def\@seccntformat#1{\@ifundefined{#1@cntformat}%
   {\csname the#1\endcsname\quad}%    default
   {\csname #1@cntformat\endcsname}}% enable individual control
\newcommand\section@cntformat{Часть \thesection:\space}
\makeatother


\renewcommand{\arraystretch}{1.3}

\begin{document}
\title{Семинар \#8: Процессы \vspace{-5ex}}\date{}\maketitle


\section*{Программа \texttt{ps}}




\begin{flushleft}
\begin{tabular}{ l | l }
Команда 	    	& Действие\\ \hline
\texttt{ps}	    	& Показывает процессы текущего терминала.\\
\texttt{ps -e}	    & Показать все процессы в системе.\\
\texttt{ps -f}	    & Показывает расширенную информацию: uid, pid, ppid, c, stime, tty, time, cmd.\\
\texttt{ps -ef}	    & Показать все процессы в расширенном формате.\\
\texttt{ps -u alice,bob} & Показать процессы пользователей alice и bob. \\
\texttt{ps -p 100,200} & Показать процессы, со значениями PID равными 100 или 200. \\ \hline
\texttt{ps -o pid,user,cmd} & Вывод в пользовательском формате. Для каждого процесса покажет идентификатор \\
                            & процесса (\texttt{pid}), имя пользователя (\texttt{user}) и команду которой был запущен процесс (\texttt{cmd}).\\
\texttt{ps -o pid=,user=,cmd=} & То же, но не выведет первую строку с описанием столбцов.\\
\texttt{ps -C top}             & Все процессы с именем программы \texttt{top}.\\
\texttt{ps -{}-sort time}        & Отсортировать по ключевому слову \texttt{time}.\\
\texttt{ps -{}-sort -time}       & Сортировка в обратном порядке по ключевому слову \texttt{time}.\\ \hline
\texttt{ps aux}                  & Все процессы с кл. словами: user, pid, pcpu, pmem, vsz, rss, tty, stat, start, time, cmd
\end{tabular}
\end{flushleft}


\subsection*{Распространённые ключевые слова для \texttt{ps -o} и \texttt{ps -{}-sort}}
\begin{flushleft}
\begin{tabular}{ l | l }
Ключевое слово 	    		& Расшифровка \\ \hline
\texttt{pid}	    		& Идентификатор процесса (Process ID).\\
\texttt{ppid}	    		& Идентификатор родительского процесса (Parent Process ID).\\
\texttt{user}	    		& Имя пользователя, запустившего процесс.\\
\texttt{uid}	    		& Идентификатор пользователя.\\
\texttt{comm} 	 			& Имя команды (без аргументов). \\
\texttt{cmd}				& Полная командная строка (включая аргументы).\\
\texttt{tty}				& Управляющий терминал.\\
\texttt{time}				& Совокупное время ЦП (CPU time).\\
\texttt{etime}				& Прошедшее время с момента запуска процесса (Elapsed time).\\
\texttt{c}					& Использование ЦП (в минутах).\\
\texttt{pcpu}				& Процент использования ЦП.\\
\texttt{pmem}				& Процент использования физической памяти.\\
\texttt{vsz}				& Размер виртуальной памяти в КиБ (Virtual Size).\\
\texttt{rss}				& Размер резидентного набора в КиБ (Resident Set Size).\\
\texttt{stat}				& Текущее состояние процесса (например, R - выполняется, S - спит).\\
\texttt{ni}					& Значение nice (приоритет).\\
\texttt{priority}			& Приоритет ядра.\\
\texttt{start}				& Время или дата запуска процесса.\\
\end{tabular}
\end{flushleft}

\noindent Больше опций и ключевых слов можно найти в \texttt{man ps}.




\section*{Сигналы}


\subsection*{Распространённые сигналы}
\begin{flushleft}
\begin{tabular}{ c | c | c | l }
Номер 	& Название 	& Действие по умолчанию  & Примечание\\ \hline
2		& SIGINT	& Завершение 			 & Посылается, когда пользователь нажимает \texttt{Ctrl+C}.\\
										 & & & Используется для корректного завершения процесса.\\ \hline
15		& SIGTERM	& Завершение 			 & Стандартный запрос на корректное завершение процесса.\\ \hline
9		& SIGKILL	& Немедленное завершение & Принудительное завершение, не дающее программе  \\
										 & & & возможности сохранить данные или корректно освободить  \\
										 & & & ресурсы. Не может быть перехвачен процессом.\\ \hline
1		& SIGHUP	& Завершение 			 & Посылается при закрытии терминала.\\ \hline
20		& SIGTSTP	& Приостановка 			 & Посылается, когда пользователь нажимает \texttt{Ctrl+Z}.\\
										 & & & Приостанавливает процесс, который затем может быть \\
										 & & & возобновлен командами \texttt{fg} или \texttt{bg}.\\ \hline
19		& SIGSTOP	& Приостановка 			 & Принудительно приостанавливает выполнение процесса.\\
										 & & & Не может быть проигнорирован или перехвачен процессом \\ \hline
18		& SIGCONT	& Возобновление 	     & Возобновляет выполнение процесса, который был остановлен.\\
										 & & & сигналами SIGSTOP или SIGTSTP. Используются командами \\
										 & & & \texttt{fg} и \texttt{bg} для возобновления работы процесса. \\ \hline
3		& SIGQUIT	& Создание дампа ядра 	 & Посылается, когда пользователь нажимает \texttt{Ctrl+\textbackslash}.\\
				    & &	и завершение	     & Корректно завершает процесс. Дополнительно создает файл \\ 
				                         & & & дампа памяти, полезный для отладки\\ \hline								
\end{tabular}
\end{flushleft}


\subsection*{Основные файлы в \texttt{/proc/[pid]}}
\begin{flushleft}
\begin{tabular}{ l | l }
Файл 	    		& Содержимое \\ \hline
\texttt{cmdline}	& Команда и аргументы, использованные для запуска процесса (разделены нулевыми байтами).\\
\texttt{status}		& Обзор состояния процесса: состояние, PID, PPID, UID/GID, потребление памяти и другое.\\
\texttt{stat}		& Однострочный набор числовых данных о процессе, предназначенный для парсинга.\\
\texttt{environ}	& Переменные окружения процесса (разделены нулевыми байтами).\\
\texttt{limits}		& Текущие ограничения ресурсов, установленные для процесса.\\
\texttt{maps}		& Карты памяти процесса, показывающие, что отображено в адресное пространство процесса.\\
\texttt{smaps}		& Усовершенствованная версия \texttt{maps}, предоставляющая более детальную информацию.\\
\texttt{exe}		& Символическая ссылка на оригинальный исполняемый файл программы.\\
\texttt{cwd}		& Символическая ссылка на текущий рабочий каталог процесса.\\
\texttt{fd}			& Поддиректория с символическими ссылками на открытые файловые дескрипторы процесса.\\ \hline
\texttt{io}			& Статистика ввода/вывода процесса (количество байт, считанных и записанных).\\
\texttt{statm}		& Краткая статистика использования памяти в страницах.\\
\texttt{cgroup}		& Информация о группе контроля (cgroups), к которой принадлежит процесс.\\
\texttt{net}		& Поддиректория с информацией о сетевых соединениях процесса\\
\texttt{oom\_score}	& Оценка для OOM killer. Чем выше, тем вероятнее процесс будет убит при нехватке памяти.\\
\texttt{pagemap}	& Предоставляет информацию о том, как виртуальные страницы отображаются на физические.\\
\texttt{fdinfo}		& Подробная информация о каждом открытом файловом дескрипторе.\\
\texttt{syscall}	& Номер и аргументы последнего системного вызова, выполненного процессом.\\
\end{tabular}
\end{flushleft}



\newpage
~
\newpage
\section*{Директория \texttt{/proc}. Информация о (почти) каждом файле.}

\subsection*{Файлы в \texttt{/proc}}
\begin{flushleft}
\begin{tabular}{ l | l }
Файл 	    		& Содержимое \\ \hline
\texttt{cmdline}	& Аргументы командной строки, переданные ядру при загрузке.\\
\texttt{version}	& Версия ядра Linux, компилятор и дата сборки.\\
\texttt{kallsyms}	& Таблица символов ядра.\\
\texttt{kmsg}		& Буфер сообщений ядра (то, что выводит команда dmesg).\\
\texttt{modules}	& Список загруженных в ядро модулей (драйверов).\\
\texttt{loadavg}	& Средняя загрузка системы (load average) за 1, 5 и 15 минут, а также\\
                    & информация о запущенных/общем количестве процессов.\\
\texttt{stat}		& Общая статистика системы\\
\texttt{uptime}		& Время работы системы (uptime) и время простоя (idle) в секундах.\\
\texttt{filesystems}	& Список файловых систем, которые в данный момент поддерживаются ядром.\\
\texttt{sys}		& Директория, содержащая параметры ядра, которые можно просматривать и изменять "на лету".\\
\texttt{sysvipc}	& Информация о механизмах межпроцессного взаимодействия.\\
\texttt{keys}		& Информация о ключах аутентификации, используемых ядром.\\
\texttt{cpuinfo}	& Содержит подробную информацию о каждом процессоре.\\
\texttt{schedstat}	& Статистика планировщика задач (scheduler) для каждого CPU.\\
\texttt{meminfo}	& Показывает общую, свободную, доступную и используемую память,\\
					& а также информацию о буферах, кэше, страницах и подкачке.\\
\texttt{vmstat}		& Статистика виртуальной памяти\\
\texttt{swaps}		& Информация об активных разделах или файлах подкачки.\\
\texttt{buddyinfo}	& Информация о фрагментации памяти.\\
\texttt{slabinfo}	& Информация о кэшах SLAB/SLUB.\\
\texttt{zoneinfo}	& Подробная информация о зонах памяти\\
\texttt{kpage*}		& Низкоуровневая информация о страницах физической памяти.\\
\texttt{pressure}	& Показывает, насколько система замедляется из-за нехватки ресурсов.\\
\texttt{vmallocinfo}& Информация об областях памяти, выделенных с помощью \texttt{vmalloc}\\
\texttt{mtrr}		& Конфигурация того, как CPU кэширует различные области памяти.\\ \hline
\texttt{devices}	& Список зарегистрированных символьных и блочных устройств\\
\texttt{interrupts}	& Статистика по используемым линиям прерываний.\\
\texttt{irq}		& Директория для управления прерываниями\\
\texttt{ioports}	& Зарегистрированные порты ввода-вывода\\
\texttt{iomem}		& Карта физической памяти (I/O memory map), показывающая,\\
					& какие диапазоны адресов используются какими устройствами\\
\texttt{dma}		& Используемые каналы прямого доступа к памяти.\\
\texttt{acpi}		& Информация об управлении питанием, батареи, температуре.\\
\texttt{asound}		& Директория для звуковой подсистемы.\\
\texttt{bus}		& Директория с информацией о шинах.\\
\texttt{driver}		& Директория с информацией о драйверах.\\
\texttt{fb}			& Список доступных фреймбуферов (устройства вывода видео).\\
\texttt{misc}		& Различные драйверы и устройства.\\
\texttt{tty}		& Директория с информацией о терминалах.\\
\end{tabular}
\end{flushleft}



\begin{flushleft}
\begin{tabular}{ l | l }
\texttt{consoles}	& Список зарегистрированных консолей.\\
\texttt{timer\_list}& Список всех ожидающих таймеров в ядре.\\
\texttt{softirqs}	& Статистика по "мягким"{} прерываниям\\
\texttt{diskstats}	& Статистика ввода-вывода для всех блочных устройств.\\
\texttt{partitions}	& Таблица разделов дисков, известная ядру.\\
\texttt{mounts}		& Список всех смонтированных файловых систем.\\
\texttt{locks}		& Файлы, заблокированные ядром\\
\texttt{fs}			& Директория с информацией о состоянии файловых систем (квоты, inode).\\ \hline
\texttt{net/tcp}	& Список активных TCP-соединений.\\
\texttt{net/udp}	& Список активных UDP-сокетов.\\
\texttt{net/dev}	& Статистика по сетевым интерфейсам (принятые/отправленные пакеты, ошибки).\\
\texttt{net/arp}	& ARP-таблица ядра.\\
\texttt{net/route}	& Таблица маршрутизации ядра.\\ \hline
\texttt{self}		& Символическая ссылка на директорию процесса, который в данный момент читает этот файл.\\
\texttt{thread-self}& Аналогично self, но указывает на директорию конкретного потока\\
\texttt{kcore}	& Виртуальный файл, представляющий всю физическую память системы в формате ELF core.\\
\texttt{cgroups}	& Информация о контрольных группах, используемых для ограничения ресурсов\\
\end{tabular}
\end{flushleft}
\subsection*{Файлы в \texttt{/proc/[pid]}}
\begin{flushleft}
\begin{tabular}{ l | l }
Файл 	    		& Содержимое \\ \hline
\texttt{cmdline}	& Команда и аргументы, использованные для запуска процесса (разделены нулевыми байтами).\\
\texttt{environ}	& Переменные окружения процесса (разделены нулевыми байтами).\\
\texttt{status}		& Обзор состояния процесса: состояние, PID, PPID, UID/GID, потребление памяти и другое.\\
\texttt{stat}		& Однострочный набор числовых данных о процессе, предназначенный для парсинга.\\
\texttt{statm}		& Краткая информация об использовании памяти (в страницах).\\
\texttt{auxv}		& Информация, переданная ядром в пространство пользователя при запуске.\\
\texttt{loginuid}	& ID пользователя, который вошел в систему, инициировавший сессию этого процесса.\\
\texttt{sessionid}	& ID сессии процесса\\
\texttt{arch\_status}	& Информация о состоянии, специфичная для архитектуры процессора.\\
\texttt{personality}	& Домен выполнения.\\ \hline
\texttt{cwd}		& Ссылка на текущую рабочую директорию процесса.\\
\texttt{exe}		& Ссылка на исполняемый файл, который был запущен.\\
\texttt{root}		& Ссылка на корневую директорию (/) с точки зрения процесса. Для процессов в \texttt{chroot} \\
					& или контейнерах она будет отличаться от реального корня.\\
\texttt{fd}			& Директория, содержащая файловые дескрипторы, открытые процессом.\\
\texttt{fdinfo}		& Директория с дополнительной информацией о каждом файловом дескрипторе.\\ \hline
\texttt{maps}		& Карта памяти процесса. Показывает, какие области памяти куда отображены.\\
\texttt{smaps}		& Подробная карта памяти. \\
\texttt{smaps\_rollup}	& Суммарная информация из \texttt{smaps}.\\
\texttt{pagemap}	& Низкоуровневая информация о том, как виртуальные страницы памяти процесса\\
					& отображаются на физическую память.\\
\texttt{mem}		& Файл, представляющий всю виртуальную память процесса.\\
\end{tabular}
\end{flushleft}

\begin{flushleft}
\begin{tabular}{ l | l }
\texttt{clear\_refs}& Файл для сброса счетчиков "обращения"{} к страницам памяти.\\
\texttt{numa\_maps}	& Информация о политике использования памяти на NUMA-системах.\\
\texttt{coredump\_filter}	& Битовая маска, определяющая, какие части памяти процесса будут включены в coredump.\\
\texttt{ksm\_*} & Статистика по KSM (Kernel Same-page Merging).\\ \hline
\texttt{task}		& Содержит поддиректории для каждого потока внутри этого процесса.\\
\texttt{sched}		& Информация о политике планировщика для этого процесса\\
\texttt{schedstat}	& Детальная статистика планировщика.\\
\texttt{stack}		& Текущий стек вызовов ядра для этого процесса/потока.\\
\texttt{wchan}		& Если процесс спит (Sleeping), этот файл показывает, в какой функции ядра он "ждет".\\
\texttt{syscall}	& Информация о системном вызове, который в данный момент выполняется процессом.\\
\texttt{timerslack\_ns}	& "Допуск"{} для срабатывания таймеров в наносекундах.\\
\texttt{timens\_offsets}& Смещения времени для временных пространств имен.\\
\texttt{patch\_state}	& Статус ядра по применению "живых"{} патчей (live patching) к этому процессу.\\ \hline
\texttt{limits}		& Ограничения на ресурсы, установленные для процесса.\\
\texttt{oom\_score}	& Оценка для OOM killer. Чем выше, тем вероятнее процесс будет убит при нехватке памяти.\\
\texttt{oom\_score\_adj} & Файлы для корректировки \texttt{oom\_score}.\\
\texttt{cgroup}		& Контрольные группы, к которым принадлежит процесс.\\
\texttt{cpuset}		& Какие CPU и узлы памяти разрешено использовать процессу.\\
\texttt{autogroup}	& Информация о группе автоматической балансировки планировщика.\\
\texttt{cpu\_resctrl\_groups}	& Группы контроля ресурсов CPU.\\
\texttt{ns}			& Директория со ссылками на Namespaces, в которых "живет"{} процесс.\\
\texttt{uid\_map}	& Карты для трансляции ID пользователей в Namespaces.\\
\texttt{gid\_map}	& Карты для трансляции ID пользователей в Namespaces.\\
\texttt{projid\_map}& Карты для трансляции ID проектов в Namespaces.\\
\texttt{setgroups}	& Используется в User Namespaces, чтобы разрешить или запретить процессу изменять \\
					& свой список групп.\\
\texttt{attr}		& Атрибуты безопасности.\\ \hline
\texttt{io}		& Статистика по операциям ввода-вывода, выполненным процессом (прочитано/записано байт).\\
\texttt{mounts}		& Список файловых систем, смонтированных с точки зрения этого процесса.\\
\texttt{mountinfo}	& Более подробная информация о монтировании.\\
\texttt{mountstats}	& Статистика по точкам монтирования.\\
\texttt{net}		& Директория с сетевой статистикой, специфичной для сетевого пространства имен процесса.\\
\end{tabular}
\end{flushleft}


\end{document}
