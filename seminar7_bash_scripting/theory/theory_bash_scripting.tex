\documentclass{article}
\usepackage[english,russian]{babel}
\usepackage{textcomp}
\usepackage{geometry}
  \geometry{left=2cm}
  \geometry{right=1.5cm}
  \geometry{top=1.5cm}
  \geometry{bottom=2cm}
\usepackage{tikz}
\usepackage{multicol}
\usepackage{hyperref}
\usepackage{listings}
\pagenumbering{gobble}

\lstdefinestyle{csMiptCppStyle}{
  language=C++,
  basicstyle=\linespread{1.1}\ttfamily,
  columns=fixed,
  fontadjust=true,
  basewidth=0.5em,
  keywordstyle=\color{blue}\bfseries,
  commentstyle=\color{gray},
  texcl=true,
  stringstyle=\ttfamily\color{orange!50!black},
  showstringspaces=false,
  numbersep=5pt,
  numberstyle=\tiny\color{black},
  numberfirstline=true,
  stepnumber=1,      
  numbersep=10pt,
  backgroundcolor=\color{white},
  showstringspaces=false,
  captionpos=b,
  breaklines=true
  breakatwhitespace=true,
  xleftmargin=.2in,
  extendedchars=\true,
  keepspaces = true,
  tabsize=4,
  upquote=true,
}


\lstdefinestyle{csMiptCppLinesStyle}{
  style=csMiptCppStyle,
  frame=lines,
}

\lstdefinestyle{csMiptCppBorderStyle}{
  style=csMiptCppStyle,
  framexleftmargin=5mm, 
  frame=shadowbox, 
  rulesepcolor=\color{gray}
}


\lstdefinestyle{csMiptBash}{
breaklines=true,
frame=tb,
language=bash,
breakatwhitespace=true,
alsoletter={*()"'0123456789.},
alsoother={\{\=\}},
basicstyle={\ttfamily},
keywordstyle={\bfseries},
literate={{=}{{{=}}}1},
prebreak={\textbackslash},
sensitive=true,
stepnumber=1,
tabsize=4,
morekeywords={echo, function},
otherkeywords={-, \{, \}},
literate={\$\{}{{{{\bfseries{}\$\{}}}}2,
upquote=true,
frame=none
}


\lstset{style=csMiptCppLinesStyle}
\lstset{literate={~}{{\raisebox{0.5ex}{\texttildelow}}}{1}}


\renewcommand{\thesection}{\arabic{section}}
\makeatletter
\def\@seccntformat#1{\@ifundefined{#1@cntformat}%
   {\csname the#1\endcsname\quad}%    default
   {\csname #1@cntformat\endcsname}}% enable individual control
\newcommand\section@cntformat{Часть \thesection:\space}
\makeatother



\begin{document}
\title{Семинар \#7: Скрипты Bash \vspace{-5ex}}\date{}\maketitle

\subsection*{Запуск скриптов Bash}
\texttt{
\begin{flushleft}
\begin{tabular}{ p{4cm} | l }
 chmod +x script.sh & дать права скрипту на выполнение \\
 ./script.sh & запуск скрипта \\
\end{tabular}
\end{flushleft}
}



\subsection*{Манипуляции со строками}
\begin{verbatim}
$var    - значение переменной (строки) var
${var}  - значение переменной (строки) var
${#var} - длина строки var
----------------------------------------------------------------------------------------
${var#*x} -  строка с удалённым префиксом, соответствующим выражению "*x" 
             "*x" это любая подстрока, заканчивающаяся на x
             берётся наименьшее вхождение
            
${var##*x} - строка с удалённым префиксом, соответствующим выражению "*x" 
             берётся наибольшее вхождение
----------------------------------------------------------------------------------------
${var%x*} -  строка с удалённым префиксом, соответствующим выражению "x*" 
             "x*" это любая подстрока, начинающаяся на x
             берётся наименьшее вхождение
            
${var%%x*} - строка с удалённым префиксом, соответствующим выражению "x*" 
             берётся наибольшее вхождение
----------------------------------------------------------------------------------------
    
\end{verbatim}



\end{document}
