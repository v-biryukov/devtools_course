\documentclass{article}
\usepackage[english,russian]{babel}
\usepackage{textcomp}
\usepackage{geometry}
  \geometry{left=2cm}
  \geometry{right=1.5cm}
  \geometry{top=1.5cm}
  \geometry{bottom=2cm}
\usepackage{tikz}
\usepackage{multicol}
\usepackage{hyperref}
\usepackage{listings}
\usepackage{pmboxdraw}
\usepackage{fancyvrb}
\usepackage[shortlabels]{enumitem}
\usepackage{upquote}
\usepackage{chngcntr}
\pagenumbering{gobble}
\counterwithout{subsection}{section}

\lstdefinestyle{csMiptCStyle}{
  language=C,
  basicstyle=\linespread{1.1}\ttfamily,
  columns=fixed,
  fontadjust=true,
  basewidth=0.5em,
  keywordstyle=\color{blue}\bfseries,
  commentstyle=\color{gray},
  texcl=true,
  stringstyle=\ttfamily\color{orange!50!black},
  showstringspaces=false,
  numbersep=5pt,
  numberstyle=\tiny\color{black},
  numberfirstline=true,
  stepnumber=1,      
  numbersep=10pt,
  backgroundcolor=\color{white},
  showstringspaces=false,
  captionpos=b,
  breaklines=true
  breakatwhitespace=true,
  xleftmargin=.2in,
  extendedchars=\true,
  keepspaces = true,
  tabsize=4,
  upquote=true,
}


\lstdefinestyle{csMiptCLinesStyle}{
  style=csMiptCStyle,
  frame=lines,
}

\lstdefinestyle{csMiptCBorderStyle}{
  style=csMiptCStyle,
  framexleftmargin=5mm, 
  frame=shadowbox, 
  rulesepcolor=\color{gray}
}


\lstdefinestyle{csMiptBash}{
  	style=csMiptCStyle,
	breaklines=true,
	frame=tb,
	language=bash,
	breakatwhitespace=true,
	alsoletter={*()"'0123456789.},
	alsoother={\{\=\}},
	basicstyle={\ttfamily},
	keywordstyle={\bfseries},
	literate={{=}{{{=}}}1},
	prebreak={\textbackslash},
	sensitive=true,
	stepnumber=1,
	tabsize=4,
	morekeywords={echo, function},
	otherkeywords={-, \{, \}},
	literate={\$\{}{{{{\bfseries{}\$\{}}}}2,
	upquote=true,
	frame=none
}

\lstset{style=csMiptBash}
\lstset{
        literate={~}{{\raisebox{0.5ex}{\texttildelow}}}{1}
}

\renewcommand{\thesubsection}{\arabic{subsection}}
\makeatletter
\def\@seccntformat#1{\@ifundefined{#1@cntformat}%
   {\csname the#1\endcsname\quad}
   {\csname #1@cntformat\endcsname}}
\newcommand\section@cntformat{}     
\newcommand\subsection@cntformat{Задача \thesubsection.\space} 
\newcommand\subsubsection@cntformat{\thesubsubsection.\space}
\makeatother



\begin{document}
\title{Семинар \#11: Прикладные протоколы Практика. \vspace{-5ex}}\date{}\maketitle
\subsection*{Как сдавать задачи}
Для сдачи ДЗ вам нужно создать репозиторий на GitLab (если он ещё не создан) под названием \texttt{devtools-homework}. Структура репозитория должна иметь вид:
\begin{center}
\begin{BVerbatim}
├── seminar11_application_protocols/
│   ├── 01.sh
│   ├── 02.sh
│   └── ...
└── ...
\end{BVerbatim}
\end{center}
Для каждой задачи, если в самой задаче не сказано иное, нужно создать 1 скрипт с расширением \texttt{.sh} и шебангом в начале скрипта. Если задача делится на подзадачи нужно, если в самой задаче не сказано иное, создать скрипт для каждой подзадачи. Названия файлов решений для всех задач/подзадач должны начинаться с номера задачи, например \texttt{01.sh} или \texttt{04b.sh}, даже если в условии задачи используется другое имя для скрипта.\\
Если в задаче встречается вопрос, то на этот вопрос нужно ответить в комментариях (начинаются с \texttt{\#}) скрипта.

\subsection{Программа \texttt{curl}}
\begin{enumerate}[(a)]
\item \textbf{Получение HTML-страницы}\\
Используйте \texttt{curl}, \texttt{GET}-запрос, чтобы получить главную страницу \texttt{mipt.ru} и сохранить её в файле \texttt{mipt.html}.

\item \textbf{Заголовок}\\
Получите только заголовок ответа на GET-запрос с \texttt{mipt.ru}.

\item \textbf{Загрузка файла}\\
Используйте \texttt{curl}, чтобы скачать файл:
\begin{verbatim}
https://upload.wikimedia.org/wikipedia/commons/7/70/The_Blue_Marble%2C_AS17-148-22727.jpg
\end{verbatim}

\item \textbf{GET-запрос}\\
Отправьте GET-запрос на \texttt{https://httpbin.org/get}.

\item \textbf{Ошибка 404}\\
Отправьте GET-запрос на \texttt{https://httpbin.org/status/404}

\item \textbf{POST-запрос}\\
Отправьте POST-запрос на \texttt{https://httpbin.org/post}.

\item \textbf{PUT-запрос}\\
Отправьте PUT-запрос на \texttt{https://httpbin.org/put}.
\end{enumerate}



\subsection{nginx-сервер}
\begin{enumerate}[(a)]
\item Откройте две виртуальные машины в одной сети NAT.
\item Установите и запустите nginx сервер на одной из машин.
\item На другой машине используйте веб-браузер, чтобы зайти на сервер другой ВМ.
\item Отправьте GET-запрос с одной ВМ на другую с помощью \texttt{curl}.
\item Настройте страницу по умолчанию \texttt{/var/www/html/index.html}.
\item Создай новый виртуальный хост для \texttt{example.local}.
\item Сделайте редирект с HTTP на HTTPS 
\end{enumerate}
\end{document}
