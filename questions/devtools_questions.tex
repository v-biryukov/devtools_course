\documentclass{article}
\usepackage[english,russian]{babel}
\usepackage{textcomp}
\usepackage{geometry}
  \geometry{left=2cm}
  \geometry{right=1.5cm}
  \geometry{top=1.5cm}
  \geometry{bottom=2cm}
\usepackage{tikz}
\usepackage{multicol}
\usepackage{hyperref}
\usepackage{listings}
\usepackage{pmboxdraw}
\usepackage{fancyvrb}
\usepackage[shortlabels]{enumitem}
\usepackage{upquote}
\usepackage{chngcntr}
\pagenumbering{gobble}
\counterwithout{subsection}{section}

\lstdefinestyle{csMiptCStyle}{
  language=C,
  basicstyle=\linespread{1.1}\ttfamily,
  columns=fixed,
  fontadjust=true,
  basewidth=0.5em,
  keywordstyle=\color{blue}\bfseries,
  commentstyle=\color{gray},
  texcl=true,
  stringstyle=\ttfamily\color{orange!50!black},
  showstringspaces=false,
  numbersep=5pt,
  numberstyle=\tiny\color{black},
  numberfirstline=true,
  stepnumber=1,      
  numbersep=10pt,
  backgroundcolor=\color{white},
  showstringspaces=false,
  captionpos=b,
  breaklines=true
  breakatwhitespace=true,
  xleftmargin=.2in,
  extendedchars=\true,
  keepspaces = true,
  tabsize=4,
  upquote=true,
}


\lstdefinestyle{csMiptCLinesStyle}{
  style=csMiptCStyle,
  frame=lines,
}

\lstdefinestyle{csMiptCBorderStyle}{
  style=csMiptCStyle,
  framexleftmargin=5mm, 
  frame=shadowbox, 
  rulesepcolor=\color{gray}
}


\lstdefinestyle{csMiptBash}{
  	style=csMiptCStyle,
	breaklines=true,
	frame=tb,
	language=bash,
	breakatwhitespace=true,
	alsoletter={*()"'0123456789.},
	alsoother={\{\=\}},
	basicstyle={\ttfamily},
	keywordstyle={\bfseries},
	literate={{=}{{{=}}}1},
	prebreak={\textbackslash},
	sensitive=true,
	stepnumber=1,
	tabsize=4,
	morekeywords={echo, function},
	otherkeywords={-, \{, \}},
	literate={\$\{}{{{{\bfseries{}\$\{}}}}2,
	upquote=true,
	frame=none
}

\lstset{style=csMiptBash}
\lstset{
        literate={~}{{\raisebox{0.5ex}{\texttildelow}}}{1}
}

\newcommand{\prevsubsection}{%
  \number\numexpr\value{subsection}-1\relax%
}

\renewcommand{\thesubsection}{\arabic{subsection}}
\makeatletter
\def\@seccntformat#1{\@ifundefined{#1@cntformat}%
   {\csname the#1\endcsname\quad}
   {\csname #1@cntformat\endcsname}}
\newcommand\section@cntformat{}     
\newcommand\subsection@cntformat{Задача \thesubsection.\space} 
\newcommand\subsubsection@cntformat{\thesubsubsection.\space}
\makeatother



\renewcommand{\arraystretch}{1.3}


\begin{document}

\section*{Инструменты разработчика. Вопросы.}


\begin{enumerate}
\setcounter{enumi}{-1}
\item \textbf{Основные команды оболочки и работа в терминале Linux}
\begin{enumerate}[(a)]
\item \textbf{Основные команды}
\begin{itemize}
\item \texttt{pwd}
\item \texttt{cd}
\item \texttt{ls} и её опции \texttt{-l}, \texttt{-a} и \texttt{-i}
\item \texttt{echo} и её опции \texttt{-n} и \texttt{-e}
\item \texttt{cat} и её опция \texttt{-n}.
\item \texttt{less} и её опции \texttt{-N} и \texttt{-R}. Горячие клавиши \texttt{Пробел}, \texttt{b}, \texttt{q}, \texttt{g}, \texttt{G}, \texttt{n}, \texttt{N}, \texttt{h}. Поиск текста в файле
\item Текстовый редактор \texttt{nano}
\item \texttt{file}
\item \texttt{alias}, создание и удаление алиасов. Просмотр всех алиасов. Как сохранить алиас навсегда?
\item \texttt{man}, поиск в конкретном разделе
\item \texttt{date}, печать даты/времени в форматированном виде (опция \texttt{+<формат>})
\end{itemize}

\item \textbf{Команды для манипуляции с файлами}
\begin{itemize}
\item \texttt{touch}
\item \texttt{mkdir} и её опция \texttt{-p}
\item \texttt{cp} и её опции \texttt{-r}, \texttt{-a}, \texttt{-t}, \texttt{-i}, \texttt{-u}, \texttt{-n}
\item \texttt{mv}, переименование файлов с помощью \texttt{mv}
\item \texttt{rm} и её опции \texttt{-r}, \texttt{-f}
\end{itemize}

\item \textbf{Сокращение имён директорий}
\begin{itemize}
\item \texttt{.}
\item \texttt{..}
\item \texttt{$\sim$}
\item \texttt{$\sim$alice}
\end{itemize}

\item \textbf{Горячие клавиши, используемые в терминале}
\begin{itemize}
\item \texttt{Ctrl-C}
\item \texttt{Ctrl-D}
\item \texttt{Ctrl-Z}
\end{itemize}

\item \textbf{Команды для архивации}
\begin{itemize}
\item \texttt{gzip} и её опции \texttt{-d} и \texttt{-k}
\item \texttt{bzip2} и её опции \texttt{-d} и \texttt{-k}
\item \texttt{tar} и её опции \texttt{-с}, \texttt{-x}, \texttt{-t}, \texttt{-v}, \texttt{-f}, \texttt{-z} и \texttt{-j}
\end{itemize}

\item \textbf{Пакетные менеджеры}\\
Пакетные менеджеры \texttt{apt} и \texttt{dnf}. Подкоманды этих пакетных менеджеров:
\begin{itemize}
\item \texttt{install}/\texttt{remove}
\item \texttt{update}/\texttt{upgrade}
\item \texttt{search}
\end{itemize}
\end{enumerate}


\item \textbf{Основы \texttt{git}}
\begin{enumerate}[(a)]


\item \textbf{Системы контроля версий}\\
Что такое система контроля версий? Локальные, централизованные и распределённые системы контроля версий.
Отличие \texttt{git} от других систем контроля версий.


\item \textbf{Настройка \texttt{git}}\\
Команда \texttt{git config}. Печать всех настроек \texttt{git}. Добавление и удаление настроек. Уровни настроек:
\begin{itemize}
\item Системный
\item Глобальный
\item Локальный
\end{itemize}
Файлы конфигурации, в которых хранятся эти настройки. Основные настройки:
\begin{itemize}
\item \texttt{user.name}
\item \texttt{user.email}
\item \texttt{core.editor}
\item \texttt{core.autocrlf}
\item \texttt{alias.<name>}
\end{itemize}



\item \textbf{Создание репозитория}\\
Команда \texttt{git init} и её опция \texttt{-{}-bare}. Команда \texttt{git clone}.


\item \textbf{Работа с файлами и коммитами}\\
Рабочая директория. Индекс. Команда \texttt{git add}. Добавление всех изменений из рабочей директории в индекс. Удаление файлов из индекса. Команда \texttt{git rm} и её опция \texttt{-{}-cached}. Локальный репозиторий. Коммит. Хэш коммита. Команда \texttt{git commit} и её опция \texttt{-m}.


\item \textbf{Команды для просмотр информации индекса и локального репозитория}
\begin{itemize}
\item \texttt{git status}
\item \texttt{git show}
\item \texttt{git diff} и её опция \texttt{-{}-staged}
\item \texttt{git log} и её опции \texttt{-{}-oneline}, \texttt{-{}-graph}, \texttt{-{}-all}, \texttt{-{}-pretty=format}
\item \texttt{git log} для поиска в истории, её опции \texttt{-{}-since}, \texttt{-{}-author}, \texttt{-{}-grep}, \texttt{-S}
\item \texttt{git blame}
\end{itemize}

\item \textbf{Работа с ветками}\\
Что такое ветка? Основная ветка \texttt{main}. Команда \texttt{git branch} и её опции \texttt{-d}, \texttt{-D}, \texttt{-m}, \texttt{-M}, \texttt{-v}, \texttt{-r}, \texttt{-vv}.

\item \textbf{Адресация коммитов}\\
Адресация коммитов с использованием хэша. Полный и сокращённый хэш коммита. Адресация коммитов с помощью веток и указателя \texttt{HEAD}. Символы \texttt{$\sim$} и $\wedge$.

\item \textbf{Перемещение по графу коммитов}\\
Указатель \texttt{HEAD}. Команда \texttt{git switch} и её опции \texttt{-c} и \texttt{-{}-detach}. Чем \texttt{git switch} отличается от старой команды \texttt{git checkout}?  Переход на отдельный коммит. Состояние \texttt{detached HEAD} и чем опасно это состояние? 

\item \textbf{Слияние}\\
Слияние веток. Команда \texttt{git merge}. Слияние перемоткой (fast-forward merge) и когда используется такой вид слияния. Опция \texttt{-{}-no-ff}. Конфликты при слиянии. Когда возникают конфликты? Как разрешить конфликт? Команды 
\begin{itemize}
\item \texttt{git merge -{}-abort}
\item \texttt{git merge -{}-continue}
\end{itemize}
Как отменить произведённое слияние?

\item \textbf{Перебазирование}\\
Перебазирование веток. Команда \texttt{git rebase}. Отличие перебазирования от слияния. Преимущества и недостатки перебазирования по сравнению со слиянием. Конфликты при перебазировании. Как разрешить конфликт, возникший при перебазировании? Команды \texttt{git rebase -{}-abort}, \texttt{git rebase -{}-continue} и \texttt{git rebase -{}-skip}. 
\begin{itemize}
\item \texttt{git rebase -{}-abort}
\item \texttt{git rebase -{}-continue}
\item \texttt{git rebase -{}-skip}
\end{itemize}

Как отменить произведённое перебазирование?

\end{enumerate}


\item \textbf{Продолжение \texttt{git}}
\begin{enumerate}[a.]
\item \textbf{Откат состояния}\\
Команда \texttt{git reset} и её режимы \texttt{-{}-soft}, \texttt{-{}-mixed} и \texttt{-{}-hard}. Как меняется рабочая директория, индекс и локальный репозиторий при использовании \texttt{git reset} в каждом из этих режимов? Как отменить произведённый откат состояния? Команда \texttt{git reflog}, что она показывает? В каких случаях использование \texttt{git reset} может привести к безвозвратной потери данных? Команда \texttt{git restore} и её опция \texttt{-{}-source}.
Как восстановить случайно удалённый файл в рабочей директории?

\item \textbf{Копирование отдельных коммитов}\\
Команда \texttt{git cherry-pick}. В чём недостатки использования \texttt{git cherry-pick}? Конфликты при копировании коммитов. Команды:
\begin{itemize}
\item \texttt{git cherry-pick -{}-abort}
\item \texttt{git cherry-pick -{}-continue}
\item \texttt{git cherry-pick -{}-skip}
\end{itemize}

\item \textbf{Интерактивное перебазирование}\\
Команда \texttt{git rebase -i}. Файл \texttt{git-rebase-todo}. Команды интерактивного перебазирования:
\vspace{-3mm}
\begin{multicols}{2}
\begin{itemize}
\item \texttt{pick}
\item \texttt{reword}
\item \texttt{edit}
\item \texttt{squash}
\item \texttt{fixup}
\item \texttt{drop}
\end{itemize}
\end{multicols}
\vspace{-3mm}
Конфликты при интерактивном перебазировании. Как отменить произведённое интерактивное перебазирование?

\item \textbf{Типы файлов в \texttt{git}}
\begin{itemize}
\item Отслеживаемые (tracked)
\begin{itemize}
\item Индексированные (staged)
\item Неизменённые (unmodified)
\item Изменённые (modified)
\end{itemize}
\item Неотслеживаемые (untracked)
\item Игнорируемые (ignored)
\end{itemize}

\item \textbf{Игнорируемые файлы}\\
Файл \texttt{.gitignore} и как с его помощью:
\begin{itemize}
\item Игнорировать все файлы в репозитории с данным именем
\item Игнорировать все директории в репозитории с данным именем
\item Игнорировать один конкретный файл
\item Игнорировать файлы по шаблону \texttt{*}, \texttt{?}, \texttt{[...]}
\item Сделать исключение из игнорирования
\end{itemize}

Игнорирование пустых директорий. Как заставить \texttt{git} не игнорировать пустые директории?


\item \textbf{Очистка репозитория}\\
На какие типы файлов не действует команда \texttt{git reset -{}-hard}? Какие типы файлов меняются при использовании \texttt{git switch}?
Команда \texttt{git clean} и её опции \texttt{-f}, \texttt{-d}, \texttt{-x}, \texttt{-X}, \texttt{-n}.


\item \textbf{Удалённый репозиторий}\\
Удалённый репозиторий. Ремоут (remote). В чём разница между удалённым репозиторием и ремоутом? Команда \texttt{git remote}, её опция \texttt{-v} и подкоманды \texttt{add}, \texttt{remove}, \texttt{rename}, \texttt{set-url} и \texttt{show}.

\item \textbf{Удалённые ветки}\\
Что такое удалённая ветка (remote branch)? Что такое ветка отслеживания (remote-tracking branch)? Какие имена имеют ветки отслеживания в git? На что указывают такие ветки? Когда обновляется состояние таких веток? Команда:
\begin{itemize}
\item \texttt{git branch -a}
\end{itemize}


\item \textbf{Работа с удалённым репозиторием}\\
Команды для взаимодействия локального репозитория с удалённым:
\begin{itemize}
\item \texttt{git push} и её опции \texttt{-{}-all}, \texttt{-{}-tags}, \texttt{-f}, \texttt{-{}-force-with-lease}, \texttt{-u}, \texttt{-{}-set-upstream}.
\item \texttt{git fetch} и её опции \texttt{-{}-prune} и \texttt{--tags}.
\item \texttt{git pull} и её опции \texttt{-{}-rebase} и \texttt{-{}-ff-only}. Конфликты при использовании \texttt{git pull}.
\end{itemize}
Чем \texttt{git pull} отличается от \texttt{git fetch}? Перезапись истории в удалённом репозитории. Отмена изменений в удалённом репозитории. Команда \texttt{git push -f} и в чём её опасность? Команда \texttt{git revert}.

\item \textbf{Отслеживающие и upstream-ветки}\\
Что такое отслеживающая ветка (tracking branch)? Что такое upstream-ветка (upstream-branch)? Как привязать отслеживающую ветку к \texttt{upstream}-ветке? Какие преимущества даёт такая привязка? Что будет, если не сделать такую привязку? Команды:
\begin{itemize}
\item \texttt{git branch -vv}
\item \texttt{git branch -{}-set-upstream-to=origin/main}
\item \texttt{git branch -{}-unset-upstream}
\item \texttt{git push -u origin main}
\end{itemize}
\item \textbf{Тэги}\\
Чем тэги отличаются от веток? Зачем нужны тэги? Команда \texttt{git tag} и её опции \texttt{-d}, \texttt{-l}, \texttt{-a} и \texttt{-m}.

\item \textbf{Pull request}\\
Веб-сервисы для хранения и работы с удалёнными репозиториями. GitHub. GitLab. Форк. Pull request. Merge request.
\end{enumerate}


\item \textbf{Продвинутый \texttt{git} (не будет на коллоквиуме)}

\item \textbf{Потоки и конвейеры}
\begin{enumerate}[a.]
\item \textbf{Команды, которые часто используются в конвейерах}
\begin{itemize}
\item \texttt{tac}
\item \texttt{head}
\item \texttt{tail}
\item \texttt{wc}
\end{itemize}


\item \textbf{Потоки и перенаправление}\\
Потоки \texttt{stdin}, \texttt{stdout} и \texttt{stderr}. Перенаправление потоков.

\item \textbf{Синтаксис подстановки команд}\\
\item \textbf{Конвейеры}\\
\item \textbf{Wildcards}\\
\item \textbf{Команда \texttt{find}}\\
\item \textbf{Команда \texttt{xargs}}\\
\item \textbf{Команда \texttt{grep}}\\
\item \textbf{Команда \texttt{tee}}\\
\end{enumerate}


\item \textbf{Пользователи и права доступа Linux}
\begin{enumerate}[a.]
\item \textbf{Системные файлы, хранящие информацию о пользователях и группах}
\begin{itemize}
\item \texttt{/etc/passwd}
\item \texttt{/etc/shadow}
\item \texttt{/etc/group}
\item \texttt{/etc/gshadow}
\end{itemize}
В каком формате хранится информация в каждом из файлов? Как хранится информация о паролях пользователей?

\item \textbf{Просмотр информации о пользователях и группах}\\
Команды \texttt{id}, \texttt{groups} и \texttt{getent}.

\item \textbf{Работа с пользователями}
\begin{itemize}
\item \texttt{useradd}
\item \texttt{usermod}
\item \texttt{userdel}
\end{itemize}
Блокировка пользователей.


\item \textbf{Работа с паролями}\\
Команда \texttt{passwd}.


\item \textbf{\texttt{sudo} и \texttt{su}}\\

\item \texttt{Права доступа}\\
Просмотр прав доступа с помощью \texttt{ls}. Символьное и числовое представление прав доступа. Конвертация из одного представления в другое. Изменение прав доступа с помощью команды \texttt{chmod}. На что влияют права доступа обычных файлов? На что влияют права доступа директорий?

\item \textbf{SUID, SGID и Sticky Bit}\\
За что отвечает SUID? Примеры системных файлов, которые имеют SUID бит. Работает ли SUID на скриптах? За что отвечает SGID? SGID для обычных файлов и для директорий. За что отвечает Sticky Bit. Пример системной директории, которая имеет Sticky Bit. Символьное и числовое представление прав доступа вместе с данными битами. 

\end{enumerate}


\item \textbf{Диски и файловые системы}
\begin{enumerate}[a.]
\item \textbf{Единицы измерения информации}
\item \textbf{Просмотр информации о дисках, разделах и точках монтирования}
\begin{itemize}
\item \texttt{lsblk}
\item \texttt{blkid}
\item \texttt{findmnt}
\end{itemize}

\item \textbf{Просмотр информации об использованной памяти}
\begin{itemize}
\item \texttt{df}
\item \texttt{du}
\end{itemize}


\item \textbf{Таблица разделов}\\
MBR и GPT

\item \textbf{Команда \texttt{dd}}\\


\item \textbf{Ссылки}\\
Мягкие и жёсткие.

\item \textbf{Файловые системы}\\
Блоки. Суперблок. Таблица inode. Что хранится в inode-ах?

\item \textbf{Распространённые файловые системы}\\
ext4, FAT, NTFS.

\end{enumerate}


\item \textbf{Язык Bash}
\begin{enumerate}[a.]
\item \textbf{Переменные Bash}\\
Создание и использование переменных в bash. Переменные среды. Команды \texttt{unset} и \texttt{export}. Файл \texttt{.bashrc}.

\item \textbf{Переменная среды \texttt{PATH}}

\item \textbf{Аргументы}\\
Передача аргументов в Bash.

\item \textbf{Коды возврата}\\

\item \textbf{Управляющая конструкция \texttt{if fi}}\\

\item \textbf{Управляющая конструкция \texttt{case esac}}\\

\item \textbf{Циклы}\\

\item \textbf{Манипуляции со строками в bash}\\

\item \textbf{Работа с целыми числами}\\

\item \textbf{Функции}\\

\item \textbf{Массивы и словари}\\

\item \textbf{Чтение из стандартного входа или из файла}\\


\end{enumerate}


\item \textbf{Процессы}
\begin{enumerate}[a.]

\item \textbf{Основные определения}\\
Что такое программа? Что такое процесс? 

\item \textbf{Управление заданиями}\\
Фоновый и передний режим. Как запустить процесс в фоновом режиме? Команды \texttt{jobs}, \texttt{fg}, \texttt{bg}. Использование \texttt{kill} для того, чтобы посылать сигналы процессам из \texttt{jobs}. Как остановить и возобновить процесс?

\item \textbf{Просмотр информации о процессах}\\
Команда \texttt{ps} и её опции \texttt{-e}, \texttt{-f}, \texttt{aux}, \texttt{-u}, \texttt{-o}. Команда \texttt{pstree}.

\item \textbf{Идентификаторы}\\
Идентификатор процесса PID. Как узнать идентификатор процесса? Как узнать идентификатор дочернего процесса? Группа процессов. Сессия.


\item \textbf{Сигнал}\\
Что такое сигнал? Основные сигналы:
\begin{itemize}
\item SIGTERM
\item SIGHUP
\item SIGINT
\item SIGKILL
\item SIGSTOP
\item SIGTSTP
\item SIGCONT
\item SIGCHLD
\end{itemize}
Как послать сигнал процессу?

\item \textbf{Перехват сигналов}\\
Команда \texttt{trap}. Перехват сигналов. Игнорирование сигналов.

\item \textbf{Отсоединённый процесс}\\
Отсоединение процесса. Команда \texttt{nohup}. 

\item \textbf{Состояния процессов}
\begin{itemize}
\item Running
\item Sleeping
\item Uninterruptible sleep
\item Stopped
\item Zombie
\end{itemize}
В каких случаях процесс попадает в то или иное состояние?

\item \textbf{Мониторинг процессов}\\
Команда \texttt{top} и \texttt{htop}. Директория \texttt{/proc}.

\end{enumerate}



\item \textbf{systemd}
\begin{enumerate}[a.]
\item \textbf{Основы}\\
Что такое systemd и какую роль он выполняет в Linux? Юнит в systemd. Типы юнитов:
\begin{itemize}
\item service
\item target
\item timer
\item path
\end{itemize}

\item \textbf{Команда \texttt{systemctl}}\\
Подкоманды:
\begin{itemize}
\item \texttt{systemctl start}
\item \texttt{systemctl stop}
\item \texttt{systemctl enable}
\item \texttt{systemctl disable}
\item \texttt{systemctl status}
\item \texttt{systemctl cat}
\item \texttt{systemctl reload}
\item \texttt{systemctl daemon-reload}
\item \texttt{systemctl list-units}
\end{itemize}


\item \textbf{Юнит-файлы}\\
В каких директориях systemd ищет юнит файлы? Как создать свой сервис? Строение юнит файла. Секции. Директивы секции \texttt{[Unit]}:
\begin{itemize}
\item \texttt{Description}
\item \texttt{After}/\texttt{Before}
\item \texttt{Wants}/\texttt{Requires}
\item \texttt{Condition*}
\end{itemize}

Директивы секции \texttt{[Service]}:
\begin{itemize}
\item \texttt{Type}, типы \texttt{simple}, \texttt{forking}, \texttt{oneshot}, \texttt{notify}.
\item \texttt{ExecStart}
\item \texttt{ExecStop}
\item \texttt{Restart}
\item \texttt{User}/\texttt{Group}
\item \texttt{Environment}/\texttt{EnvironmentFile}
\item \texttt{PIDFile}
\item \texttt{TimeoutStartSec}
\item \texttt{KillMode}
\end{itemize} 

Директива секции \texttt{[Install]}:
\begin{itemize}
\item \texttt{WantedBy}
\end{itemize}

Таргеты по умолчанию:
\begin{itemize}
\item \texttt{rescue.target}
\item \texttt{multi-user.target}
\item \texttt{graphical.target}
\item \texttt{default.target}
\end{itemize}

\end{enumerate}


\end{enumerate}

\end{document}