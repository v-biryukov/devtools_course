\documentclass{article}
\usepackage[english,russian]{babel}
\usepackage{textcomp}
\usepackage{geometry}
  \geometry{left=2cm}
  \geometry{right=1.5cm}
  \geometry{top=1.5cm}
  \geometry{bottom=2cm}
\usepackage{tikz}
\usepackage{multicol}
\usepackage{hyperref}
\usepackage{listings}
\usepackage{pmboxdraw}
\usepackage{fancyvrb}
\usepackage[shortlabels]{enumitem}
\usepackage{upquote}
\usepackage{chngcntr}
\pagenumbering{gobble}
\counterwithout{subsection}{section}

\lstdefinestyle{csMiptCStyle}{
  language=C,
  basicstyle=\linespread{1.1}\ttfamily,
  columns=fixed,
  fontadjust=true,
  basewidth=0.5em,
  keywordstyle=\color{blue}\bfseries,
  commentstyle=\color{gray},
  texcl=true,
  stringstyle=\ttfamily\color{orange!50!black},
  showstringspaces=false,
  numbersep=5pt,
  numberstyle=\tiny\color{black},
  numberfirstline=true,
  stepnumber=1,      
  numbersep=10pt,
  backgroundcolor=\color{white},
  showstringspaces=false,
  captionpos=b,
  breaklines=true
  breakatwhitespace=true,
  xleftmargin=.2in,
  extendedchars=\true,
  keepspaces = true,
  tabsize=4,
  upquote=true,
}


\lstdefinestyle{csMiptCLinesStyle}{
  style=csMiptCStyle,
  frame=lines,
}

\lstdefinestyle{csMiptCBorderStyle}{
  style=csMiptCStyle,
  framexleftmargin=5mm, 
  frame=shadowbox, 
  rulesepcolor=\color{gray}
}


\lstdefinestyle{csMiptBash}{
  	style=csMiptCStyle,
	breaklines=true,
	frame=tb,
	language=bash,
	breakatwhitespace=true,
	alsoletter={*()"'0123456789.},
	alsoother={\{\=\}},
	basicstyle={\ttfamily},
	keywordstyle={\bfseries},
	literate={{=}{{{=}}}1},
	prebreak={\textbackslash},
	sensitive=true,
	stepnumber=1,
	tabsize=4,
	morekeywords={echo, function},
	otherkeywords={-, \{, \}},
	literate={\$\{}{{{{\bfseries{}\$\{}}}}2,
	upquote=true,
	frame=none
}

\lstset{style=csMiptBash}
\lstset{
        literate={~}{{\raisebox{0.5ex}{\texttildelow}}}{1}
}

\newcommand{\prevsubsection}{%
  \number\numexpr\value{subsection}-1\relax%
}

\renewcommand{\thesubsection}{\arabic{subsection}}
\makeatletter
\def\@seccntformat#1{\@ifundefined{#1@cntformat}%
   {\csname the#1\endcsname\quad}
   {\csname #1@cntformat\endcsname}}
\newcommand\section@cntformat{}     
\newcommand\subsection@cntformat{Задача \thesubsection.\space} 
\newcommand\subsubsection@cntformat{\thesubsubsection.\space}
\makeatother



\renewcommand{\arraystretch}{1.3}


\begin{document}

\section*{Инструменты разработчика. Вопросы.}


\begin{enumerate}
\setcounter{enumi}{-1}
\item \textbf{Основные команды оболочки и работа в терминале Linux}
\begin{enumerate}[(a)]
\item \textbf{Основные команды}
\begin{itemize}
\item \texttt{pwd}
\item \texttt{cd}
\item \texttt{ls} и её опции \texttt{-l}, \texttt{-a} и \texttt{-i}. Какая информация отображается при вызове \texttt{ls -l}?
\item \texttt{echo} и её опции \texttt{-n} и \texttt{-e}
\item \texttt{printf}, спецификаторы \texttt{\%s},  \texttt{\%c}, \texttt{\%d}, \texttt{\%x}, \texttt{\%f}, \texttt{\%8s}, \texttt{\%08d} и \texttt{\%.2f}. 
\item \texttt{cat} и её опция \texttt{-n}.
\item \texttt{less} и её опции \texttt{-N} и \texttt{-R}. Горячие клавиши \texttt{Пробел}, \texttt{b}, \texttt{q}, \texttt{g}, \texttt{G}, \texttt{n}, \texttt{N}, \texttt{h}. Поиск текста в файле
\item Текстовый редактор \texttt{nano}
\item \texttt{file}
\item \texttt{alias}, создание и удаление алиасов. Просмотр всех алиасов. Как сохранить алиас навсегда?
\item \texttt{man}, поиск в конкретном разделе
\item \texttt{date}, печать даты/времени в форматированном виде (опция \texttt{+<формат>})
\end{itemize}

\item \textbf{Команды для манипуляции с файлами}
\begin{itemize}
\item \texttt{touch}
\item \texttt{mkdir} и её опция \texttt{-p}
\item \texttt{cp} и её опции \texttt{-r}, \texttt{-a}, \texttt{-t}, \texttt{-i}, \texttt{-u}, \texttt{-n}
\item \texttt{mv}, переименование файлов с помощью \texttt{mv}
\item \texttt{rm} и её опции \texttt{-r}, \texttt{-f}
\end{itemize}

\item \textbf{Путь к файлу}\\
Абсолютный и относительный путь. Сокращение имён директорий:
\vspace{-3mm}
\begin{multicols}{2}
\begin{itemize}
\item \texttt{.}
\item \texttt{..}
\item \texttt{$\sim$}
\item \texttt{$\sim$alice}
\end{itemize}
\end{multicols}
\vspace{-3mm}

\item \textbf{Горячие клавиши, используемые в терминале}
\begin{itemize}
\item \texttt{Ctrl-C}
\item \texttt{Ctrl-D}
\item \texttt{Ctrl-Z}
\end{itemize}

\item \textbf{Команды для архивации}
\begin{itemize}
\item \texttt{gzip} и её опции \texttt{-d} и \texttt{-k}
\item \texttt{bzip2} и её опции \texttt{-d} и \texttt{-k}
\item \texttt{tar} и её опции \texttt{-с}, \texttt{-x}, \texttt{-t}, \texttt{-v}, \texttt{-f}, \texttt{-z} и \texttt{-j}
\end{itemize}

\item \textbf{Пакетные менеджеры}\\
Пакетные менеджеры \texttt{apt} и \texttt{dnf}. Подкоманды этих пакетных менеджеров:
\begin{itemize}
\item \texttt{install}/\texttt{remove}
\item \texttt{update}/\texttt{upgrade}
\item \texttt{search}
\end{itemize}
\end{enumerate}


\item \textbf{Основы \texttt{git}}
\begin{enumerate}[(a)]


\item \textbf{Системы контроля версий}\\
Что такое система контроля версий? Локальные, централизованные и распределённые системы контроля версий.
Отличие \texttt{git} от других систем контроля версий.


\item \textbf{Настройка \texttt{git}}\\
Команда \texttt{git config}. Печать всех настроек \texttt{git}. Добавление и удаление настроек. Уровни настроек:
\begin{itemize}
\item Системный
\item Глобальный
\item Локальный
\end{itemize}
Файлы конфигурации, в которых хранятся эти настройки. Основные настройки:
\begin{itemize}
\item \texttt{user.name}
\item \texttt{user.email}
\item \texttt{core.editor}
\item \texttt{core.autocrlf}
\item \texttt{alias.<name>}
\end{itemize}



\item \textbf{Создание репозитория}\\
Команда \texttt{git init} и её опция \texttt{-{}-bare}. Команда \texttt{git clone}.


\item \textbf{Работа с файлами и коммитами}\\
Рабочая директория. Индекс. Команда \texttt{git add}. Добавление всех изменений из рабочей директории в индекс. Удаление файлов из индекса. Команда \texttt{git rm} и её опция \texttt{-{}-cached}. Локальный репозиторий. Коммит. Хэш коммита. Команда \texttt{git commit} и её опция \texttt{-m}.


\item \textbf{Команды для просмотра информации индекса и локального репозитория}
\begin{itemize}
\item \texttt{git status}
\item \texttt{git show}
\item \texttt{git diff} и её опция \texttt{-{}-staged}
\item \texttt{git log} и её опции \texttt{-{}-oneline}, \texttt{-{}-graph}, \texttt{-{}-all}, \texttt{-{}-pretty=format}
\item \texttt{git log} для поиска в истории, её опции \texttt{-{}-since}, \texttt{-{}-author}, \texttt{-{}-grep}, \texttt{-S}
\item \texttt{git blame}
\end{itemize}

\item \textbf{Работа с ветками}\\
Что такое ветка? Основная ветка \texttt{main}. Команда \texttt{git branch} и её опции \texttt{-d}, \texttt{-D}, \texttt{-m}, \texttt{-M}, \texttt{-v}, \texttt{-r}, \texttt{-vv}.

\item \textbf{Адресация коммитов}\\
Адресация коммитов с использованием хэша. Полный и сокращённый хэш коммита. Адресация коммитов с помощью веток и указателя \texttt{HEAD}. Символы \texttt{$\sim$} и $\wedge$.

\item \textbf{Перемещение по графу коммитов}\\
Указатель \texttt{HEAD}. Команда \texttt{git switch} и её опции \texttt{-c} и \texttt{-{}-detach}. Чем \texttt{git switch} отличается от старой команды \texttt{git checkout}?  Переход на отдельный коммит. Состояние \texttt{detached HEAD}. Чем опасно это состояние? 

\item \textbf{Слияние}\\
Слияние веток. Команда \texttt{git merge}. Слияние перемоткой (fast-forward merge). Когда используется такой вид слияния? Опция \texttt{-{}-no-ff}. Конфликты при слиянии. Когда возникают конфликты? Как разрешить конфликт? Команды 
\begin{itemize}
\item \texttt{git merge -{}-abort}
\item \texttt{git merge -{}-continue}
\end{itemize}
Как отменить произведённое слияние?

\item \textbf{Перебазирование}\\
Перебазирование веток. Команда \texttt{git rebase}. Отличие перебазирования от слияния. Преимущества и недостатки перебазирования по сравнению со слиянием. Конфликты при перебазировании. Как разрешить конфликт, возникший при перебазировании? Команды: 
\begin{itemize}
\item \texttt{git rebase -{}-abort}
\item \texttt{git rebase -{}-continue}
\item \texttt{git rebase -{}-skip}
\end{itemize}

Как отменить произведённое перебазирование?

\end{enumerate}


\item \textbf{Продолжение \texttt{git}}
\begin{enumerate}[a.]
\item \textbf{Откат состояния}\\
Команда \texttt{git reset} и её режимы \texttt{-{}-soft}, \texttt{-{}-mixed} и \texttt{-{}-hard}. Как меняется рабочая директория, индекс и локальный репозиторий при использовании \texttt{git reset} в каждом из этих режимов? Как отменить произведённый откат состояния? Команда \texttt{git reflog}, что она показывает? В каких случаях использование \texttt{git reset} может привести к безвозвратной потере данных? Команда \texttt{git restore} и её опция \texttt{-{}-source}.
Как восстановить случайно удалённый файл в рабочей директории?

\item \textbf{Копирование отдельных коммитов}\\
Команда \texttt{git cherry-pick}. В чём недостатки использования \texttt{git cherry-pick}? Конфликты при копировании коммитов. Команды:
\begin{itemize}
\item \texttt{git cherry-pick -{}-abort}
\item \texttt{git cherry-pick -{}-continue}
\item \texttt{git cherry-pick -{}-skip}
\end{itemize}

\item \textbf{Интерактивное перебазирование}\\
Команда \texttt{git rebase -i}. Файл \texttt{git-rebase-todo}. Команды интерактивного перебазирования:
\vspace{-3mm}
\begin{multicols}{2}
\begin{itemize}
\item \texttt{pick}
\item \texttt{reword}
\item \texttt{edit}
\item \texttt{squash}
\item \texttt{fixup}
\item \texttt{drop}
\end{itemize}
\end{multicols}
\vspace{-3mm}
Конфликты при интерактивном перебазировании. Как отменить произведённое интерактивное перебазирование?

\item \textbf{Типы файлов в \texttt{git}}
\begin{itemize}
\item Отслеживаемые (tracked)
\begin{itemize}
\item Индексированные (staged)
\item Неизменённые (unmodified)
\item Изменённые (modified)
\end{itemize}
\item Неотслеживаемые (untracked)
\item Игнорируемые (ignored)
\end{itemize}

\item \textbf{Игнорируемые файлы}\\
Файл \texttt{.gitignore} и то, как с его помощью:
\begin{itemize}
\item Игнорировать все файлы в репозитории с данным именем
\item Игнорировать все директории в репозитории с данным именем
\item Игнорировать один конкретный файл
\item Игнорировать файлы по шаблону \texttt{*}, \texttt{?}, \texttt{[...]}
\item Сделать исключение из игнорирования
\end{itemize}

Игнорирование пустых директорий. Как заставить \texttt{git} не игнорировать пустые директории?


\item \textbf{Очистка репозитория}\\
На какие типы файлов не действует команда \texttt{git reset -{}-hard}? Какие типы файлов не меняются при использовании \texttt{git switch}?
Команда \texttt{git clean} и её опции \texttt{-f}, \texttt{-d}, \texttt{-x}, \texttt{-X}, \texttt{-n}.


\item \textbf{Удалённый репозиторий}\\
Удалённый репозиторий. Ремоут (remote). В чём разница между удалённым репозиторием и ремоутом? Команда \texttt{git remote}, её опция \texttt{-v} и подкоманды \texttt{add}, \texttt{remove}, \texttt{rename}, \texttt{set-url} и \texttt{show}.

\item \textbf{Удалённые ветки}\\
Что такое удалённая ветка (remote branch)? Что такое ветка отслеживания (remote-tracking branch)? Какие имена имеют ветки отслеживания в git? На что указывают такие ветки? Когда обновляется состояние таких веток? Команда:
\begin{itemize}
\item \texttt{git branch -a}
\end{itemize}


\item \textbf{Работа с удалённым репозиторием}\\
Команды для взаимодействия локального репозитория с удалённым:
\begin{itemize}
\item \texttt{git push} и её опции \texttt{-{}-all}, \texttt{-{}-tags}, \texttt{-f}, \texttt{-{}-force-with-lease}, \texttt{-u}, \texttt{-{}-set-upstream}.
\item \texttt{git fetch} и её опции \texttt{-{}-prune} и \texttt{-{}-tags}.
\item \texttt{git pull} и её опции \texttt{-{}-rebase} и \texttt{-{}-ff-only}. Конфликты при использовании \texttt{git pull}.
\end{itemize}
Чем \texttt{git pull} отличается от \texttt{git fetch}? Перезапись истории в удалённом репозитории. Отмена изменений в удалённом репозитории. Команда \texttt{git push -f} и в чём её опасность? Команда \texttt{git revert}.

\item \textbf{Отслеживающие и upstream-ветки}\\
Что такое отслеживающая ветка (tracking branch)? Что такое upstream-ветка (upstream-branch)? Как привязать отслеживающую ветку к \texttt{upstream}-ветке? Какие преимущества даёт такая привязка? Что будет, если не сделать такую привязку? Команды:
\begin{itemize}
\item \texttt{git branch -vv}
\item \texttt{git branch -{}-set-upstream-to=origin/main}
\item \texttt{git branch -{}-unset-upstream}
\item \texttt{git push -u origin main}
\end{itemize}
\item \textbf{Тэги}\\
Чем тэги отличаются от веток? Зачем нужны тэги? Команда \texttt{git tag} и её опции \texttt{-d}, \texttt{-l}, \texttt{-a} и \texttt{-m}.

\item \textbf{Pull request}\\
Веб-сервисы для хранения и работы с удалёнными репозиториями. GitHub. GitLab. Форк. Pull request. Merge request.
\end{enumerate}


\item \textbf{Продвинутый \texttt{git} (не будет на коллоквиуме)}

\item \textbf{Потоки и конвейеры}
\begin{enumerate}[a.]
\item \textbf{Основные команды, которые часто используются в конвейерах}
\begin{itemize}
\item \texttt{tac}
\item \texttt{head} и её опции \texttt{-n} и \texttt{-c}
\item \texttt{tail} и её опции \texttt{-n}, \texttt{-c}, \texttt{-f} и \texttt{-F}
\item \texttt{uniq} и её опция \texttt{-c}
\item \texttt{sort} и её опции \texttt{-n}, \texttt{-r}, \texttt{-u}, \texttt{-h}, \texttt{-k} и \texttt{-t}.
\item \texttt{wc} и её опции \texttt{-l}, \texttt{-w}, \texttt{-c} и \texttt{-m}
\end{itemize}


\item \textbf{Потоки и перенаправление}\\
Стандартные потоки ввода-вывода:
\begin{itemize}
\item \texttt{stdin}
\item \texttt{stdout}
\item \texttt{stderr}
\end{itemize}
Куда эти потоки направлены по умолчанию? Перенаправление потока в файл для перезаписи. Перенаправление потока в файл для дозаписи. Перенаправление файла в \texttt{stdin}. Перенаправление \texttt{stderr} в файл. Как перенаправить \texttt{stdout} и \texttt{stderr} в один файл? Файл \texttt{/dev/null}. Что произойдёт, если перенаправить потоки \texttt{stdout}/\texttt{stderr} в этот файл? Что произойдёт, если перенаправить файл \texttt{/dev/null} в \texttt{stdin}?

\item \textbf{Расширения оболочки}
\begin{itemize}
\item Шаблон поиска (wildcard/glob pattern). Wildcard-символы: 
\vspace{-3mm}
\begin{multicols}{2}
\begin{itemize}
\item[$\circ$] \texttt{*}
\item[$\circ$] \texttt{?}
\item[$\circ$] \texttt{[...]}
\item[$\circ$] \texttt{[!...]}
\end{itemize}
\end{multicols}
\vspace{-3mm}
\item Brace expansion \texttt{\{...\}}. 
\item Синтаксис подстановки команд \texttt{\$(...)}. 
\item Подстановка процесса \texttt{<(...)} и \texttt{>(...)}. Примеры использования подстановки процесса.
\end{itemize}


\item \textbf{Конвейеры}\\
Что такое конвейер оболочки Bash и как его использовать? Примеры простых конвейеров. Код возврата конвейера. Переменная \texttt{PIPESTATUS}. Как перенаправить поток \texttt{stderr} в конвейер?


\item \textbf{Команда \texttt{find}}\\
Опции команды \texttt{find}:
\vspace{-3mm}
\begin{multicols}{2}
\begin{itemize}
\item \texttt{-type}
\item \texttt{-name}/\texttt{-iname}
\item \texttt{-size}
\item \texttt{-mtime}
\item \texttt{-and}, \texttt{-or}, \texttt{-not}
\item \texttt{-maxdepth}
\item \texttt{-print0}
\item \texttt{-exec}
\end{itemize}
\end{multicols}
\vspace{-3mm}
Два вида синтаксиса \texttt{find -exec}:
\begin{itemize}
\item Синтаксис с \texttt{\textbackslash;}
\item Синтаксис с \texttt{+}
\end{itemize}
Чем различаются эти два вида синтаксиса? Примеры использования \texttt{find -exec}.

\item \textbf{Команда \texttt{xargs}}\\
Что делает команда \texttt{xargs}? В каких ситуациях эта команда используется чаще всего? Примеры использования команды \texttt{xargs}. Опции \texttt{-0}, \texttt{-I} и \texttt{-P}. Преимущества \texttt{xargs} перед \texttt{find -exec}.

\item \textbf{Команда \texttt{tee}}\\
Примеры использования команды \texttt{tee}. Опция \texttt{-a}. Использование \texttt{tee} вместе с подстановкой процесса. Использование \texttt{echo}, \texttt{sudo} и \texttt{tee} для записи в файл с правами \texttt{root}.

\item \textbf{Команда \texttt{grep}}\\
Синтаксис команды \texttt{grep}. Опции:

\vspace{-3mm}
\begin{multicols}{2}
\begin{itemize}
\item \texttt{-i}
\item \texttt{-r}
\item \texttt{-n}
\item \texttt{-v}
\item \texttt{-E} / \texttt{egrep}
\item \texttt{-l}
\item \texttt{-c}
\item \texttt{-o}
\item \texttt{-w}
\item \texttt{-A}, \texttt{-B} и \texttt{-C}
\item \texttt{-{}-color=auto}
\item \texttt{-q}
\end{itemize}
\end{multicols}
\vspace{-3mm}

Приведите примеры использования \texttt{grep} в следующих ситуациях:
\begin{itemize}
\item Поиск строк, содержащих подстроку, в одном файле.
\item Поиск строк, содержащих подстроку, рекурсивно во всех файлах директории.
\item Поиск файлов, содержащих подстроку, рекурсивно во всех файлах директории.
\item Фильтрация вывода команды с помощью конвейера и \texttt{grep}.
\item Поиск строк в файле или файлах, соответствующих регулярному выражению.
\end{itemize}


\item \textbf{Регулярные выражения в \texttt{grep}}\\
Два типа регулярных выражений: BRE и ERE. Основные элементы регулярных выражений ERE:
\begin{itemize}
\item Обычные символы
\item Точка \texttt{.}
\item Символьные классы \texttt{[...]}
\item Группировка \texttt{(...)}
\item Альтернация \texttt{|}
\item Квантификаторы:
\vspace{-3mm}
\begin{multicols}{2}
\begin{itemize}
\item[$\circ$] \texttt{*}
\item[$\circ$] \texttt{+}
\item[$\circ$] \texttt{?}
\item[$\circ$] \texttt{\{m\}}
\item[$\circ$] \texttt{\{m,\}}
\item[$\circ$] \texttt{\{m,n\}}
\end{itemize}
\end{multicols}
\vspace{-3mm}
\item Якоря строки: $\wedge$ и \$.
\end{itemize}
Предопределённые классы:
\vspace{-3mm}
\begin{multicols}{2}
\begin{itemize}
\item \texttt{[:alpha:]}
\item \texttt{[:digit:]}
\item \texttt{[:alnum:]}
\item \texttt{[:space:]}
\item \texttt{[:lower:]}
\item \texttt{[:upper:]}
\item \texttt{[:punct:]}
\item \texttt{[:xdigit:]}
\end{itemize}
\end{multicols}
\vspace{-3mm}

\end{enumerate}


\item \textbf{Пользователи и права доступа Linux}
\begin{enumerate}[a.]
\item \textbf{Пользователи}\\
UID. Суперпользователь \texttt{root}. Системные пользователи. Зачем нужны системные пользователи? Какой диапазон UID обычно используется для системных пользователей, а какой для обычных? 

\item \textbf{Группы пользователей}\\
Группы. Основная группа пользователя. Дополнительные группы. GID.
Команда \texttt{id} и её опция \texttt{-u}. Команда \texttt{groups}.

\item \textbf{Домашняя директория}\\
Стандартное расположение домашней директории. Что обычно хранится в домашней директории? Файлы \texttt{$\sim$/.bash\_profile} и \texttt{$\sim$/.bashrc}. Директория \texttt{/etc/skel}.

\clearpage

\item \textbf{Системные файлы, хранящие информацию о пользователях и группах}
\begin{itemize}
\item \texttt{/etc/passwd}
\item \texttt{/etc/shadow}
\item \texttt{/etc/group}
\item \texttt{/etc/gshadow}
\end{itemize}
Какая информация и в каком формате хранится в каждом из этих файлов? Как хранится информация о паролях пользователей? Как хранится принадлежность пользователя к основной группе? Как хранится принадлежность пользователя к дополнительным группам? 
Как можно отличить обычного пользователя от системного по файлу \texttt{passwd}? Что такое \texttt{/sbin/nologin}? Как можно понять, что пользователь заблокирован по файлу \texttt{/etc/shadow}?

\item \textbf{Работа с пользователями и группами}
\begin{itemize}
\item \texttt{useradd} и её опции \texttt{-m}, \texttt{-d}, \texttt{-s} и \texttt{G}.
\item \texttt{usermod} и её опции \texttt{-G}, \texttt{-aG}, \texttt{-d}, \texttt{-s}, \texttt{-L} и \texttt{-U}.
\item \texttt{userdel} и её опция \texttt{-r}.
\item \texttt{groupadd} и её опция \texttt{-r}.
\item \texttt{groupmod}
\item \texttt{groupdel}
\end{itemize}
Создание/удаление основной группы при создании/удалении пользователя. Использование \texttt{usermod} для блокировки/разблокировки пользователя.


\item \textbf{Работа с паролями}\\
Использование команды \texttt{passwd} для смены собственного пароля. Использование команды \texttt{passwd} для смены пароля другого пользователя.


\item \textbf{\texttt{su} и \texttt{sudo}}\\
Переключение на другого пользователя. Команда \texttt{su}. Полное и неполное переключение. Выполнение команды от имени другого пользователя. Команда \texttt{sudo} и её опции \texttt{-u} и \texttt{-i}. Какой пароль требуется для команды \texttt{su}, а какой пароль -- для команды \texttt{sudo}? Чем различаются следующие способы переключения на пользователя \texttt{root}:
\begin{itemize}
\item \texttt{su}
\item \texttt{su -}
\item \texttt{sudo -i}
\end{itemize}
Файл \texttt{/etc/sudoers}. В каком формате хранятся записи в файле \texttt{/etc/sudoers}? Группа \texttt{sudo}/\texttt{wheel}. В чём опасность редактирования файла \texttt{/etc/sudoers}? Как безопасно редактировать этот файл?




\item \textbf{Права доступа}\\
Владелец файла. Группа-владелец. Может ли владелец файла не входить в группу-владелец этого же файла? Права доступа.
Просмотр прав доступа с помощью \texttt{ls}. Символьное и числовое представление прав доступа. Конвертация из одного представления в другое. Команды:
\begin{itemize}
\item \texttt{chmod} и её опция \texttt{-R}. Как можно изменять права пользователя с помощью этой команды, используя символьное и числовое представление прав? Кто и в каких случаях может изменять права доступа?
\item \texttt{chown} и её опция \texttt{-R}. Как изменить группу владельца, используя \texttt{chown}? Кто и в каких случаях может изменять владельца файла?
\item \texttt{chgrp} и её опция \texttt{-R}. Кто и в каких случаях может изменять группу владельца файла?
\item \texttt{newgrp}.
\end{itemize}
За что отвечают права \texttt{r}, \texttt{w}, \texttt{x} для обычных файлов? За что отвечают права \texttt{r}, \texttt{w}, \texttt{x} для директорий? Права символических ссылок.


\item \textbf{Маска}\\
Права только что созданного файла. Права копии файла. Команда \texttt{umask}.


\item \textbf{SUID, SGID и Sticky Bit}\\
За что отвечает SUID? Примеры системных файлов, которые имеют SUID бит. Работает ли SUID на скриптах? За что отвечает SGID? SGID для обычных файлов и для директорий. За что отвечает Sticky Bit. Пример системной директории, которая имеет Sticky Bit. Символьное и числовое представление прав доступа вместе с данными битами. 

\end{enumerate}


\item \textbf{Диски и файловые системы}
\begin{enumerate}[a.]
\item \textbf{Единицы измерения объёма информации}\\
Десятичные и двоичные единицы измерения.


\item \textbf{Метаданные файлов}\\
Что обычно включает в себя метаданные файла? Где хранятся метаданные? Просмотр метаданных с помощью команды \texttt{ls -l}. Команда \texttt{stat} и её опция \texttt{-c}. Плейсхолдеры для форматированного вывода команды \texttt{stat -c}:
\vspace{-3mm}
\begin{multicols}{4}
\begin{itemize}
\item \texttt{\%n}
\item \texttt{\%i}
\item \texttt{\%s}
\item \texttt{\%F}
\item \texttt{\%u}~~\texttt{\%U}
\item \texttt{\%g}~~\texttt{\%G}
\item \texttt{\%a}~~\texttt{\%A}
\item \texttt{\%x}~~ \texttt{\%y}~~ \texttt{\%z}
\end{itemize}
\end{multicols}
\vspace{-3mm}
Временные метки \texttt{atime}, \texttt{mtime} и \texttt{ctime}. Изменение временных меток с помощью команды \texttt{touch}. Опции команды \texttt{touch}: \texttt{-a}, \texttt{-m} и \texttt{-d}. 


\item \textbf{Типы файлов}
\begin{itemize}
\item Обычный файл
\item Директория
\item Символическая ссылка
\item Файл блочного устройства
\item Файл символьного устройства
\end{itemize}
Как узнать тип файла, используя команды \texttt{ls}, \texttt{stat} и \texttt{file}?  


\item \textbf{Файлы устройств}\\
Директория \texttt{/dev/}. Файлы устройств как интерфейсы. Какие данные и метаданные хранят в себе файлы устройств? Major и minor number.

\begin{enumerate}
\item \textbf{Файлы символьных устройств}\\
Примеры символьных устройств. Как символьные устройства передают данные? Символьные псевдоустройства \texttt{/dev/null}, \texttt{/dev/zero}, \texttt{/dev/random}.

\item \textbf{Файлы блочных устройств}\\
Примеры блочных устройств. Как блочные устройства передают данные? Блочные устройства дисков и разделов. Размер блока блочных устройств. Физический и логический размеры блоков.
\end{enumerate}

\item \textbf{Разделы}\\
Зачем нужны разделы? Блочные файлы разделов. Команда \texttt{lsblk}. Таблицы разделов:
\begin{enumerate}
\item \textbf{Таблица разделов MBR}\\
Где находится таблица MBR и сколько места занимает? Что содержится внутри MBR? Ограничение на количество разделов и размер каждого раздела. Типы разделов: первичный (primary), расширенный (extended), логический (logical).

\item \textbf{Таблица разделов GPT}\\
Где находится таблица GPT и сколько место занимает? Что содержится внутри GPT? Ограничение на количество разделов и размер каждого раздела.
\end{enumerate}
Команда \texttt{parted -l}. Как узнать, какая таблица разделов используется на диске?

\item \textbf{Создание разделов с помощью программы parted}
\vspace{-3mm}
\begin{multicols}{2}
\begin{itemize}
\item \texttt{print}
\item \texttt{mklabel}
\item \texttt{mkpart}
\item \texttt{rm}
\item \texttt{resizepart}
\item \texttt{unit}
\end{itemize}
\end{multicols}
\vspace{-3mm}
Запуск \texttt{parted} в скриптовом и тихом режиме. 

\item \textbf{Файловые системы}\\
Что такое файловая система? UUID файловой системы. Создание файловой системы на разделе. Команда \texttt{mkfs}. Команды для отображения информации о файловых системах:
\begin{itemize}
\item \texttt{lsblk -f}
\item \texttt{blkid}
\item \texttt{df} и её опции \texttt{-h}, \texttt{-T} и \texttt{-i}
\item \texttt{du} и её опции \texttt{-h} и \texttt{-s}
\end{itemize}

\newpage
\item \textbf{Монтирование файловой системы}\\
Что такое монтирование файловой системы? Точка монтирования. Команда \texttt{mount} и её опция \texttt{-o}. Команда \texttt{umount}. Опции монтирования:
\vspace{-3mm}
\begin{multicols}{2}
\begin{itemize}
\item \texttt{defaults}
\item \texttt{rw}
\item \texttt{ro}
\item \texttt{noexec}
\item \texttt{nosuid}
\item \texttt{noatime}
\end{itemize}
\end{multicols}
\vspace{-3mm}

Автоматическое монтирование файловых систем. Файл \texttt{/etc/fstab}.  Поля в этом файле:
\vspace{-3mm}
\begin{multicols}{2}
\begin{itemize}
\item \texttt{file system}
\item \texttt{mount point}
\item \texttt{type}
\item \texttt{options}
\item \texttt{dump}
\item \texttt{pass}
\end{itemize}
\end{multicols}
\vspace{-3mm}
Команда \texttt{mount -a}. Просмотр точек монтирования с помощью \texttt{findmnt}.


\item \textbf{Команда \texttt{dd}}\\
Распространённые варианты использования команды \texttt{dd}:
\begin{itemize}
\item Создание файлов определённого размера, заполненных нулевыми или случайными байтами 
\item Копирование файлов 
\item Просмотр байт файлов 
\item Просмотр байт дисков/разделов 
\item Копирование разделов 
\item Создание образа диска или раздела (файла \texttt{.img})
\item Восстановление диска/раздела из образа
\item Запись ISO-образа на флешку
\end{itemize}
Опции команды \texttt{dd}:
\vspace{-3mm}
\begin{multicols}{3}
\begin{itemize}
\item \texttt{if}
\item \texttt{of}
\item \texttt{bs}
\item \texttt{count}
\item \texttt{skip}
\item \texttt{seek}
\item \texttt{conv=noerror}
\item \texttt{conv=sync}
\item \texttt{conv=notrunc}
\end{itemize}
\end{multicols}
\vspace{-3mm}
На что влияет значение размера блока (\texttt{bs}) в команде \texttt{dd}?
Какой \texttt{bs} используется по умолчанию?

\item \textbf{Жёсткие ссылки}\\
Чем является жёсткая ссылка? Команда \texttt{ln}. Что происходит при создании жёсткой ссылки? Равноправность жёстких ссылок. Зачем нужен счётчик жёстких ссылок в inode?    
\begin{itemize}
\item Что будет, если удалить исходный файл, на который указывает жёсткая ссылка?
\item Жёсткие ссылки на директории (\texttt{.} и \texttt{..}). Можно ли создать жёсткую ссылку на директорию?
\item Можно ли создать жёсткую ссылку на файл, находящийся в другой файловой системе?
\end{itemize}


\item \textbf{Символические (мягкие) ссылки}\\
Чем является символическая ссылка? Команда \texttt{ln -s}. Как узнать, куда указывает символическая ссылка? Команда \texttt{readlink}.

\begin{itemize}
\item Что будет, если удалить исходный файл, на который указывает символическая ссылка?
\item Можно ли создать символическую ссылку на директорию?
\item Можно ли создать символическую ссылку на файл, находящийся в другой файловой системе?
\end{itemize}

Работа стандартных команд с символическими ссылками:
\begin{itemize}
\item Команда \texttt{cp} копирует саму ссылку или файл на который она указывает? А команда \texttt{cp -a}?
\item Команда \texttt{chmod} применяется к ссылке или к файлу на который она указывает?
\item Команда \texttt{rm} применяется к ссылке или к файлу на который она указывает?
\item Переходит ли \texttt{find} по символическим ссылкам на директории? Опция \texttt{-L} команды \texttt{find}.
\end{itemize}


\item \textbf{Строение файловой системы ext4 (основы)}\\
Блоки. Суперблок. Группы блоков. Таблица inode. Что хранится в inode-ах файлов разных типов?
Переполнение таблицы inode. Журналирование.

\item \textbf{Распространённые файловые системы}\\
ext4, xfs, FAT, NTFS. Виртуальные файловые системы. Особенности каждой из этих файловых систем.

\end{enumerate}


\item \textbf{Язык Bash}
\begin{enumerate}[a.]

\item \textbf{Интерпретатор Bash}\\
Что такое терминал? Что такое оболочка? Оболочки \texttt{sh} и \texttt{bash}. Скрипты Bash. Шебанг.

\item \textbf{Переменные Bash}\\
Создание и использование переменных в bash. Команда \texttt{unset}. Переменные среды. Команда \texttt{export}. Команда \texttt{source}. Как задать переменную среды для одной команды? Как создать свою переменную среды? Файл \texttt{$\sim$/.bashrc}.

\item \textbf{Команды оболочки}\\
Внутренние и внешние команды оболочки. Команда \texttt{type -a}. Команда \texttt{which}. Переменная среды \texttt{PATH}. Как создать свою команду?


\item \textbf{Кавычки}\\
Разница между одинарными и двойными кавычками в bash.

\item \textbf{Аргументы}\\
Аргументы командной строки. Как запустить скрипт, передав ему некоторые аргументы?
Переменные \texttt{\$1}, \texttt{\$2}, \texttt{\$0}, \texttt{\$*}, \texttt{\$@}. Чем отличается \texttt{"\$*"} от \texttt{"\$@"}? Проход по аргументам в цикле. Команда \texttt{shift}.

\item \textbf{Коды возврата}\\
Какой код возврата возвращается при успешном завершении программы? Как получить код возврата последней выполненной команды? Коды возврата скобочных команд. Команды \texttt{true} и \texttt{false}.

\item \textbf{Условия \texttt{if}}\\
Сравнение строк. Сравнение чисел. Операции проверки:
\vspace{-3mm}
\begin{multicols}{4}
\begin{itemize}
\item \texttt{-n}
\item \texttt{-z}
\item \texttt{-e}
\item \texttt{-f}
\item \texttt{-d}
\item \texttt{-r}
\item \texttt{-w}
\item \texttt{-x}
\end{itemize}
\end{multicols}
\vspace{-3mm}
Разница между  \texttt{[[ ... ]]} и \texttt{[ ... ]}. Проверка на совпадение с регулярным выражением \texttt{=$\sim$}.

Использование операторов \texttt{\&\&} и \texttt{||} вместо условий \texttt{if}. 

\item \textbf{Группировка команд и подоболочки}

\item \textbf{Управляющая конструкция \texttt{case esac}}\\
Примеры конструкций \texttt{case esac}. Использование glob-шаблонов в этих конструкциях.


\item \textbf{Циклы}\\
Циклы \texttt{while} и \texttt{for}. Примеры циклов. Итерация от нуля до некоторого числа. Итерация по всем элементам из заданного набора. Итерация по всем файлам в директории. Итерация по всем аргументам/элементам массива. Переменная \texttt{IFS}. Перенаправление из цикла в файл и наоборот.

\item \textbf{Манипуляции со строками}\\
Что означают следующие выражения:
\vspace{-3mm}
\begin{multicols}{2}
\begin{itemize}
\item \texttt{\$var}
\item \texttt{\$\{var\}}
\item \texttt{\$\{\#var\}}
\item \texttt{\$\{var:N\}}
\item \texttt{\$\{var:N:K\}}
\item \texttt{\$\{var:-default\}}
\item \texttt{\$\{var:=default\}}
\item \texttt{\$\{var\#pattern\}}
\item \texttt{\$\{var\#\#pattern\}}
\item \texttt{\$\{var\%pattern\}}
\item \texttt{\$\{var\%\%pattern\}}
\item \texttt{\$\{var/pattern/str\}}
\item \texttt{\$\{var\textasciicircum\textasciicircum\}}
\item \texttt{\$\{var,{},\}}
\end{itemize}
\end{multicols}
\vspace{-3mm}


\item \textbf{Работа с целыми числами}\\
Арифметические выражения \texttt{(( ... ))}. Чем \texttt{\$(( ... ))} отличается от \texttt{(( ... ))}?


\item \textbf{Функции}\\
Создание функции. Вызов функции. Передача аргументов функции. Возврат значений из функции. Локальные переменные функции: чем они отличаются от обычных переменных?

\item \textbf{Массивы}\\
Базовые операции с массивами: получение по индексу, изменение элемента, добавление элемента в конец массива, проход по массиву в цикле.

\item \textbf{Чтение из стандартного входа или из файла}\\
Команда \texttt{read} и её опции \texttt{-p}, \texttt{-r}, \texttt{-a}, \texttt{-s}. Использование \texttt{read} вместе с \texttt{while} для чтения строк из файла.
Команда \texttt{readarray}.

\end{enumerate}


\item \textbf{Процессы}
\begin{enumerate}[a.]

\item \textbf{Основные понятия, связанные с процессами}\\
Программа. Процесс. Родительский и дочерний процесс. Системные вызовы \texttt{fork} и \texttt{exec}. Идентификатор процесса. PID. PPID. Контекст процесса. Переключение контекста. Потоки. Разница между процессами и потоками. Квант процессорного времени. Планировщик.



\item \textbf{Просмотр информации о процессах}\\
Команда \texttt{ps} и её опции: \texttt{aux}, \texttt{-e}, \texttt{-f}, \texttt{-u}, \texttt{-p}, \texttt{-C}, \texttt{-{}-ppid}, \texttt{-{}-sort}, \texttt{-o}. Поля в выводе команды \texttt{ps}:
\vspace{-3mm}
\begin{multicols}{2}
\begin{itemize}
\item \texttt{pid}
\item \texttt{uid} / \texttt{user}
\item \texttt{gid} / \texttt{group}
\item \texttt{comm}
\item \texttt{cmd}
\item \texttt{args}
\item \texttt{stat}
\item \texttt{\%cpu}
\item \texttt{\%mem}
\item \texttt{rss}
\item \texttt{vsz}
\item \texttt{etime}
\end{itemize}
\end{multicols}
\vspace{-3mm}

Команда \texttt{pstree}.

\item \textbf{Идентификаторы процессов}\\
PID. PPID. Как узнать идентификатор процесса? Как узнать идентификатор родительского процесса? Как узнать идентификатор дочерних процессов? Переменные \texttt{\$\$} и \texttt{\$!}. Переменная \texttt{\$PPID}.


\item \textbf{Сигнал}\\
Что такое сигнал? Номер сигнала. Основные сигналы:
\vspace{-3mm}
\begin{multicols}{2}
\begin{itemize}
\item SIGTERM
\item SIGHUP
\item SIGINT
\item SIGKILL
\item SIGSTOP
\item SIGTSTP
\item SIGCONT
\item SIGCHLD
\end{itemize}
\end{multicols}
\vspace{-3mm}

Команда \texttt{kill}. Как послать определённый сигнал процессу?
Команды \texttt{killall} и \texttt{pkill}.

\item \textbf{Команда \texttt{trap}}\\
Перехват сигналов с помощью \texttt{trap}. Игнорирование сигналов. Какие сигналы нельзя перехватить и игнорировать? Обработка выхода из скрипта (\texttt{EXIT}). Обработка ошибок (\texttt{ERR}).

\item \textbf{Группы процессов и сессии}\\
Группа процессов. PGID. Какие процессы входят в группу процессов? Как послать сигнал всем процессам в группе? Лидер группы. Сессия. SID. Лидер сессии.

\item \textbf{Состояния процессов}
\begin{itemize}
\item Running / Runnable
\item Sleeping
\item Uninterruptible sleep
\item Stopped
\item Zombie
\end{itemize}
В каких случаях процесс попадает в то или иное состояние? Процессы, находящиеся в каких состояниях, нельзя завершить, послав им сигналы, в том числе сигнал SIGKILL?

\item \textbf{Управление заданиями}\\
Фоновый и передний режим. Как запустить процесс в фоновом режиме? Команды \texttt{jobs}, \texttt{fg}, \texttt{bg}. Использование \texttt{kill} для посылки сигналов процессам из \texttt{jobs}. Как остановить и возобновить процесс? Какой сигнал и каким процессам посылается при следующих действиях:
\begin{itemize}
\item Нажатии \texttt{Ctrl-C} в терминале
\item Нажатии \texttt{Ctrl-Z} в терминале
\item Закрытии окна терминала
\end{itemize}

\item \textbf{Отсоединённый процесс}\\
Отсоединение процесса. Команда \texttt{nohup}. Команда \texttt{disown}. Команда \texttt{setsid}. Демоны. Как написать скрипт, при запуске которого создавался бы процесс-демон?


\item \textbf{Приоритет процессов}\\
Приоритет и niceness. Какие значения может принимать niceness? Команда \texttt{nice} и её опция \texttt{-n}. Команда \texttt{renice} и её опции \texttt{-n} и \texttt{-p}.

\item \textbf{Мониторинг процессов, программа \texttt{top}}\\
Команда \texttt{top}. Какую информацию она показывает? Что означают значения \texttt{load average}?\\
Что означают поля, показывающие распределение времени процессора (\texttt{\%CPU(s)}):
\vspace{-3mm}
\begin{multicols}{3}
\begin{itemize}
\item \texttt{us}
\item \texttt{sy}
\item \texttt{ni}
\item \texttt{id}
\item \texttt{wa}
\item \texttt{hi}
\end{itemize}
\end{multicols}
\vspace{-3mm}

Что означают поля:
\vspace{-3mm}
\begin{multicols}{3}
\begin{itemize}
\item \texttt{PID}
\item \texttt{USER}
\item \texttt{PR}
\item \texttt{NI}
\item \texttt{VIRT}
\item \texttt{RES}
\item \texttt{SHR}
\item \texttt{S}
\item \texttt{\%CPU}
\item \texttt{\%MEM}
\item \texttt{TIME+}
\item \texttt{COMMAND}
\end{itemize}
\end{multicols}
\vspace{-3mm}
Программа \texttt{htop}.


\item \textbf{Директория \texttt{/proc}}\\
Директория \texttt{/proc/<PID>} и файлы в этой директории:
\vspace{-3mm}
\begin{multicols}{2}
\begin{itemize}
\item \texttt{cmdline}
\item \texttt{environ}
\item \texttt{maps}
\item \texttt{cwd}
\item \texttt{exe}
\item \texttt{fd/1}
\item \texttt{status}
\item \texttt{stat}
\end{itemize}
\end{multicols}
\vspace{-3mm}



\end{enumerate}



\item \textbf{systemd}
\begin{enumerate}[a.]
\item \textbf{Основы}\\
Что такое systemd и какую роль он выполняет в Linux? Юнит в systemd. Типы юнитов:
\begin{itemize}
\item service
\item target
\item timer
\item path
\end{itemize}

\item \textbf{Команда \texttt{systemctl}}\\
Команда \texttt{systemctl} и её подкоманды:
\vspace{-3mm}
\begin{multicols}{2}
\begin{itemize}
\item \texttt{systemctl start}
\item \texttt{systemctl stop}
\item \texttt{systemctl restart}
\item \texttt{systemctl enable}
\item \texttt{systemctl disable}
\item \texttt{systemctl status}
\item \texttt{systemctl cat}
\item \texttt{systemctl reload}
\item \texttt{systemctl daemon-reload}
\item \texttt{systemctl list-units}
\end{itemize}
\end{multicols}
\vspace{-3mm}



\item \textbf{Юнит-файлы}\\
В каких директориях systemd ищет юнит-файлы? Как создать свой сервис? Строение юнит-файла. Секции. Директивы секции \texttt{[Unit]}:
\begin{itemize}
\item \texttt{Description}
\item \texttt{After}/\texttt{Before}
\item \texttt{Wants}/\texttt{Requires}
\item \texttt{Condition*}
\end{itemize}

Директивы секции \texttt{[Service]}:
\vspace{-3mm}
\begin{multicols}{2}
\begin{itemize}
\item \texttt{Type}, типы \texttt{simple}, \texttt{forking}, \texttt{oneshot}, \texttt{notify}.
\item \texttt{ExecStart}
\item \texttt{ExecStop}
\item \texttt{Restart}
\item \texttt{User}/\texttt{Group}
\item \texttt{Environment}/\texttt{EnvironmentFile}
\item \texttt{PIDFile}
\item \texttt{TimeoutStartSec}/\texttt{TimeoutStopSec}
\item \texttt{KillMode}
\item \texttt{RemainAfterExit}
\end{itemize} 
\end{multicols}
\vspace{-3mm}




Директива секции \texttt{[Install]}:
\begin{itemize}
\item \texttt{WantedBy}
\end{itemize}

\newpage
\noindent Таргеты по умолчанию:
\begin{itemize}
\item \texttt{rescue.target}
\item \texttt{multi-user.target}
\item \texttt{graphical.target}
\item \texttt{default.target}
\end{itemize}



\item \textbf{Таймеры}\\
Зачем нужны timer‑юниты? Как создать timer-юнит, который бы запускал сервис по расписанию? Директивы:
\vspace{-3mm}
\begin{multicols}{2}
\begin{itemize}
\item \texttt{Unit}
\item \texttt{OnBootSec}
\item \texttt{OnActiveSec}
\item \texttt{Persistent}
\item \texttt{RandomizedDelaySec}
\item \texttt{OnCalendar}
\end{itemize}
\end{multicols}
\vspace{-3mm}

\item \textbf{Path-юниты}\\
Зачем нужны path-юниты? Как создать path-юнит, который бы запускал сервис при изменении файла? Директивы:
\begin{itemize}
\item \texttt{Unit}
\item \texttt{PathExists}
\item \texttt{PathChanged}
\end{itemize}

\item \textbf{Журнал}\\
Команда \texttt{journalctl}.
\end{enumerate}

\end{enumerate}


\newpage
~
\newpage
\subsection{Что добавить}
\begin{itemize}
\item Директория \texttt{.git}
\item Как можно полностью удалить коммиты из git?
\item \texttt{git commit -{}-amend}
\item \texttt{git commit -{}-fixup}
\item Удаление ветки на удалённом сервере \texttt{git push -{}-delete}
\item \texttt{-{}-rebase-merges}
\item \texttt{git check-ignore}
\item \texttt{locate}, \texttt{updatedb}
\item \texttt{pgrep}
\item \texttt{free}
\item \texttt{Какой диапазон значений принимает код возврата}
\end{itemize}


\end{document}