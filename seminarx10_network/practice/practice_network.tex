\documentclass{article}
\usepackage[english,russian]{babel}
\usepackage{textcomp}
\usepackage{geometry}
  \geometry{left=2cm}
  \geometry{right=1.5cm}
  \geometry{top=1.5cm}
  \geometry{bottom=2cm}
\usepackage{tikz}
\usepackage{multicol}
\usepackage{hyperref}
\usepackage{listings}
\usepackage{pmboxdraw}
\usepackage{fancyvrb}
\usepackage[shortlabels]{enumitem}
\usepackage{upquote}
\usepackage{chngcntr}
\pagenumbering{gobble}
\counterwithout{subsection}{section}

\lstdefinestyle{csMiptCStyle}{
  language=C,
  basicstyle=\linespread{1.1}\ttfamily,
  columns=fixed,
  fontadjust=true,
  basewidth=0.5em,
  keywordstyle=\color{blue}\bfseries,
  commentstyle=\color{gray},
  texcl=true,
  stringstyle=\ttfamily\color{orange!50!black},
  showstringspaces=false,
  numbersep=5pt,
  numberstyle=\tiny\color{black},
  numberfirstline=true,
  stepnumber=1,      
  numbersep=10pt,
  backgroundcolor=\color{white},
  showstringspaces=false,
  captionpos=b,
  breaklines=true
  breakatwhitespace=true,
  xleftmargin=.2in,
  extendedchars=\true,
  keepspaces = true,
  tabsize=4,
  upquote=true,
}


\lstdefinestyle{csMiptCLinesStyle}{
  style=csMiptCStyle,
  frame=lines,
}

\lstdefinestyle{csMiptCBorderStyle}{
  style=csMiptCStyle,
  framexleftmargin=5mm, 
  frame=shadowbox, 
  rulesepcolor=\color{gray}
}


\lstdefinestyle{csMiptBash}{
  	style=csMiptCStyle,
	breaklines=true,
	frame=tb,
	language=bash,
	breakatwhitespace=true,
	alsoletter={*()"'0123456789.},
	alsoother={\{\=\}},
	basicstyle={\ttfamily},
	keywordstyle={\bfseries},
	literate={{=}{{{=}}}1},
	prebreak={\textbackslash},
	sensitive=true,
	stepnumber=1,
	tabsize=4,
	morekeywords={echo, function},
	otherkeywords={-, \{, \}},
	literate={\$\{}{{{{\bfseries{}\$\{}}}}2,
	upquote=true,
	frame=none
}

\lstset{style=csMiptBash}
\lstset{
        literate={~}{{\raisebox{0.5ex}{\texttildelow}}}{1}
}

\renewcommand{\thesubsection}{\arabic{subsection}}
\makeatletter
\def\@seccntformat#1{\@ifundefined{#1@cntformat}%
   {\csname the#1\endcsname\quad}
   {\csname #1@cntformat\endcsname}}
\newcommand\section@cntformat{}     
\newcommand\subsection@cntformat{Задача \thesubsection.\space} 
\newcommand\subsubsection@cntformat{\thesubsubsection.\space}
\makeatother



\begin{document}
\title{Семинар \#10: Основы работы с сетями в Linux. Практика. \vspace{-5ex}}\date{}\maketitle
\subsection*{Как сдавать задачи}
Для сдачи ДЗ вам нужно создать репозиторий на GitLab (если он ещё не создан) под названием \texttt{devtools-homework}. Структура репозитория должна иметь вид:
\begin{center}
\begin{BVerbatim}
├── seminar10_network/
│   ├── 01.sh
│   ├── 02.sh
│   └── ...
└── ...
\end{BVerbatim}
\end{center}
Для каждой задачи, если в самой задаче не сказано иное, нужно создать 1 скрипт с расширением \texttt{.sh} и шебангом в начале скрипта. Если задача делится на подзадачи нужно, если в самой задаче не сказано иное, создать скрипт для каждой подзадачи. Названия файлов решений для всех задач/подзадач должны начинаться с номера задачи, например \texttt{01.sh} или \texttt{04b.sh}, даже если в условии задачи используется другое имя для скрипта.\\
Если в задаче встречается вопрос, то на этот вопрос нужно ответить в комментариях (начинаются с \texttt{\#}) скрипта.

\subsection{Подсети}
Для каждой из подсетей:
\begin{lstlisting}
a) 192.168.1.0/24
b) 10.0.0.0/8
c) 172.16.0.0/12
\end{lstlisting}
Определите:
\begin{itemize}
\item Диапазон адресов
\item Маску
\item Количество хостов
\item Адрес сети
\item Широковещательный адрес
\end{itemize}




\subsection{Просмотр информации о сетевом интерфейсе}
Используйте команду \texttt{ip address} (команду можно сокращать, то есть использовать \texttt{ip addr} или даже \texttt{ip a}), чтобы просмотреть информацию о сетевых интерфейсах. Для каждого из интерфейсов определите:
\begin{itemize}
\item Название интерфейса
\item IP-адрес (IPv4 и/или IPv6)
\item Маска подсети
\item MAC-адрес
\item Состояние интерфейса (UP/DOWN)
\item Широковещательный адрес
\item MTU (максимальный размер пакета) 
\end{itemize}

\subsection{Просмотр информации о маршрутах}
Используйте команду \texttt{ip route} (команду можно сокращать, то есть использовать \texttt{ip r}), чтобы просмотреть информацию о маршрутах. Для каждого из маршрутов определите:
\begin{itemize}
\item Целевая сеть
\item Шлюз
\item Интерфейс, через который идёт маршрут
\item Протокол, как был получен маршрут — вручную или через DHCP
\item Источник
\end{itemize}



\subsection{Пинг}
Используйте команду \texttt{ping}, чтобы проверить доступность адресов:
\begin{enumerate}[(a)]
\item Адрес \texttt{1.1.1.1}
\item Адрес \texttt{google.com}. Пошлите 20 пакетов по 1000 байт с интервалом \texttt{0.2} секунды на этот адрес.
\end{enumerate}


\subsection{\texttt{traceroute}}
Используйте команду \texttt{traceroute}, чтобы определить адреса маршрутизаторов, через которые проходят пакеты на пути к адресу \texttt{8.8.8.8}.\\
\texttt{!} На виртуальной машине с NAT это может не работать и отображать один хоп или отображать все хопы звёздочками. Если у вас не работает, то закройте виртуальную машину и в настройках виртуальной машины, во вкладке "Сеть"{} установите "Тип подключения"{} -- "Сетевой мост".


\subsection{Основы работы с \texttt{curl}}
Используйте \texttt{curl}, чтобы загрузить веб-страницу \texttt{google.com}. Сохраните эту веб страницу в файл с расширением \texttt{.html}. Откройте этот файл с помощью браузера.

\subsection{\texttt{tcpdump}}
Запустите \texttt{tcpdump}:
\begin{lstlisting}
$ sudo tcpdump -vvv
\end{lstlisting}
или
\begin{lstlisting}
$ sudo tcpdump -i имяИнтерфейса -vvv
\end{lstlisting}
и проверьте, что он отображает при:
\begin{enumerate}[(a)]
\item Использовании команды \texttt{ping}
\item Использовании \texttt{curl} для загрузки веб страницы.
\item При переходе на веб-страницу в браузере.
\end{enumerate}

\subsection{\texttt{wireshark}}
Установите программу \texttt{wireshark} и откройте её:
\begin{lstlisting}
$ sudo wireshark
\end{lstlisting}
Используете её для анализа трафика, проходящего через основной сетевой интерфейс.
Проанализируйте, какие пакеты отправляются и приходят при следующих операциях:
\begin{enumerate}[(a)]
\item Использовании команды \texttt{ping}
\item Использовании \texttt{curl} для загрузки веб страницы.
\item При переходе на веб-страницу в браузере.
\item При загрузке \texttt{YouTube} видео в браузере.
\end{enumerate}




\subsection*{Связь между двумя ВМ}
Для решения следующих задач вам понадобится ещё одна виртуальная машина. Нужно будет установить связь между виртуальными машинами. Однако по умолчанию они запускается изолировано и не видят друг друга. Чтобы машины видели друг друга сделайте следующее:

\begin{enumerate}
\item Закройте все виртуальные машины.
\item В VirtualBox зайдите в менеджер сетей (Файл -> Инструменты -> Сеть).
\item В менеджере сетей перейдите во вкладку "Сети NAT"{} и создайте новую сеть.
\item Убедитесь, что стоит галочка на "Включить DHCP".
\item Продублируйте вашу ВМ или создайте новую ВМ. У вас должно быть две виртуальные машины.
\item Зайдите в опции самих виртуальных машин и в настройках сети установите "Тип подключения"{} на значение "Сеть NAT". Выберете сеть NAT, созданную ранее. Это нужно сделать для обеих ВМ.
\item Сначала запустите одну ВМ и дождитесь полной загрузки ВМ. Проверьте, что сеть на этой ВМ работает.
\item После этого запустите вторую ВМ и дождитесь её полной загрузки. Проверьте, что сеть на этой ВМ работает.
\item Выполните команду \texttt{ip address} на двух виртуальных машинах. IP адреса виртуальных машин должны различаться. Если это не так, то измените ip адрес одной из машин с помощью команды \texttt{ip}.
\end{enumerate}


\subsection{Пинг другой виртуальной машины}
Используйте команду \texttt{ping}, для пинга другой виртуальной машины.


\subsection{Простой чат}
Используйте команду \texttt{nc}, чтобы создать простой чат между двумя виртуальными машинами.

\subsection{Подключение по SSH}
Создайте SSH подключение между виртуальными машинами. Для этого на одной из машин понадобится установить и запустить SSH-демон. Зайдите по SSH из одной машины на другую и создайте любой файл. Проверьте, что файл на другой машине действительно был создан.


\subsection{Получение файлов с помощью \texttt{scp}}
Создайте файл на одной ВМ.
Используйте \texttt{scp}, чтобы скачать этот файл на другую ВМ. Скачайте также любую директорию из одной ВМ на другую.

\subsection{Веб сервер}
Запустите веб-сервер nginx на одной из машин. После этого зайдите на него через другую виртуальную машину.


\subsection{Веб страница}
Измените страницу веб сервера на простую страницу с сообщение "Hello from other VM". Скорей всего для этого будет достаточно поменять страницу в \texttt{/var/www/html}. Зайдите на эту страницу с другой ВМ.


\subsection{Веб страница - \texttt{curl}}
Используйте \texttt{curl}, чтобы получить веб-страницу из другой ВМ.

\end{document}
