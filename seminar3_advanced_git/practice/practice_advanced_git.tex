\documentclass{article}
\usepackage[english,russian]{babel}
\usepackage{textcomp}
\usepackage{geometry}
  \geometry{left=2cm}
  \geometry{right=1.5cm}
  \geometry{top=1.5cm}
  \geometry{bottom=2cm}
\usepackage{tikz}
\usepackage{multicol}
\usepackage{hyperref}
\usepackage{listings}
\usepackage{pmboxdraw}
\usepackage{fancyvrb}
\usepackage[shortlabels]{enumitem}
\pagenumbering{gobble}

\lstdefinestyle{csMiptCppStyle}{
  language=C++,
  basicstyle=\linespread{1.1}\ttfamily,
  columns=fixed,
  fontadjust=true,
  basewidth=0.5em,
  keywordstyle=\color{blue}\bfseries,
  commentstyle=\color{gray},
  texcl=true,
  stringstyle=\ttfamily\color{orange!50!black},
  showstringspaces=false,
  numbersep=5pt,
  numberstyle=\tiny\color{black},
  numberfirstline=true,
  stepnumber=1,      
  numbersep=10pt,
  backgroundcolor=\color{white},
  showstringspaces=false,
  captionpos=b,
  breaklines=true
  breakatwhitespace=true,
  xleftmargin=.2in,
  extendedchars=\true,
  keepspaces = true,
  tabsize=4,
  upquote=true,
}


\lstdefinestyle{csMiptCppLinesStyle}{
  style=csMiptCppStyle,
  frame=lines,
}

\lstdefinestyle{csMiptCppBorderStyle}{
  style=csMiptCppStyle,
  framexleftmargin=5mm, 
  frame=shadowbox, 
  rulesepcolor=\color{gray}
}


\lstdefinestyle{csMiptBash}{
  	style=csMiptCppStyle,
	breaklines=true,
	frame=tb,
	language=bash,
	breakatwhitespace=true,
	alsoletter={*()"'0123456789.},
	alsoother={\{\=\}},
	basicstyle={\ttfamily},
	keywordstyle={\bfseries},
	literate={{=}{{{=}}}1},
	prebreak={\textbackslash},
	sensitive=true,
	stepnumber=1,
	tabsize=4,
	morekeywords={echo, function},
	otherkeywords={-, \{, \}},
	literate={\$\{}{{{{\bfseries{}\$\{}}}}2,
	upquote=true,
	frame=none
}

\lstset{style=csMiptCppStyle}
\lstset{
        literate={~}{{\raisebox{0.5ex}{\texttildelow}}}{1}
}


\renewcommand{\thesubsection}{\arabic{subsection}}
\makeatletter
\def\@seccntformat#1{\@ifundefined{#1@cntformat}%
   {\csname the#1\endcsname\quad}
   {\csname #1@cntformat\endcsname}}
\newcommand\section@cntformat{}     
\newcommand\subsection@cntformat{Задача \thesubsection.\space} 
\newcommand\subsubsection@cntformat{\thesubsubsection.\space}
\makeatother



\begin{document}
\title{Семинар \#3: git, продвинутый. Практика. \vspace{-5ex}}\date{}\maketitle

\subsection*{Задачи с прошлого семинара}
Решите новые задачи с прошлого семинара. Задача 1 и задача 12.



\subsection*{Подготовка. Генерация репозитория, который будет использоваться в задачах.}

Для решения некоторых задач вам потребуется создать репозиторий коммитов. 
Его можно найти по адресу:
\href{https://mipt-hsse.gitlab.yandexcloud.net/v.biryukov/utils}{mipt-hsse.gitlab.yandexcloud.net/v.biryukov/utils}. Клонируйте этот репозиторий себе. Решите следующие задачи:

\subsection*{Задачи}
\begin{enumerate}
\item \textbf{Просмотрите граф коммитов}:\\
Это можно сделать на GitLab во вкладке \texttt{Code -> Repository Graph} или у вас на машине, с помощью команды:
\begin{lstlisting}
git log --oneline --all --graph
\end{lstlisting}

\item \textbf{Новый alias}:\\
Добавьте новый \texttt{alias} для этой команды, назовите его \texttt{lg}. То есть теперь, при вызове: 
\begin{lstlisting}
git lg
\end{lstlisting}
Должно печататься то же, что и при вызове 
\begin{lstlisting}
git log --oneline --all --graph
\end{lstlisting}



\item \textbf{Слияние с конфликтом}:\\
Слейте ветку \texttt{bob} в ветку \texttt{main}. Для этого вам нужно перейти в ветку \texttt{main} и вызвать \texttt{git merge bob}. При этом могут возникнуть конфликты. Разрешите их. (в этой и последующих задачах необязательно, чтобы код после слияния был корректным, так как это задание на git, а не на язык программирования).

\item \textbf{Отмена слияния}:\\
Отмените, сделанное только что слияние.

\item \textbf{Перебазирование с конфликтом}:\\
Перебазируйте ветку \texttt{bob} на ветку \texttt{main}. При этом могут возникнуть конфликты. Разрешите их.

\item \textbf{Отмена перебазирования}:\\
Отмените, сделанное только что перебазирование.

\item \textbf{Копия коммита с конфликтом}:\\
Перейдите на ветку \texttt{main} и скопируйте в неё один коммит (\texttt{git cherry-pick}) с хэшем \texttt{893e28d} из ветки \texttt{bob}. При этом могут возникнуть конфликты. Разрешите их.

\item \textbf{Отмена \texttt{cherry-pick}}:\\
Отмените, сделанный только что \texttt{cherry-pick}.

\item \textbf{Слияние быстрой перемоткой}:\\
Перейдите на ветку \texttt{alice} и слейте в ветку \texttt{alice} ветку \texttt{main}, используя быструю перемотку.

\item \textbf{Отмена быстрой перемотки}:\\
Отмените, сделанную только что быструю перемотку.

\item \textbf{Слияние без быстрой перемоткой}:\\
Перейдите на ветку \texttt{alice} и слейте в ветку \texttt{alice} ветку \texttt{main}, без быстрой перемотки.

\item \textbf{Отмена последнего слияния}:\\
Отмените, сделанное только что слияние.

\item \textbf{Разница}:\\
Перейдите в ветку \texttt{casper} и найдите \texttt{diff} последних двух коммитов.

\item \textbf{Патч-файл}:\\
Создайте патч-файл, содержащий \texttt{diff} последних двух коммитов ветки \texttt{casper}.
Перейдите в ветку \texttt{main} и примените этот патч. Закоммите изменения в ветке \texttt{main}.

\item \textbf{Разница одного файла}:\\
Напечатайте \texttt{diff} файла \texttt{integration.py} в ветках \texttt{main} и \texttt{casper}.

\item \textbf{git bisect}:\\
Перейдите в ветку \texttt{casper} и запустите скрипт \texttt{sorting.py}
\begin{lstlisting}
python ./sorting.py
\end{lstlisting}
Вы увидите, что одна из сортировок работает неправильно, хотя в других ветках эта сортировка работала правильно. Значит в одном из коммитов ветки \texttt{casper} була допущена ошибка. Найдите коммит, в котором была допущена ошибка с помощью \texttt{git bisect}. Исправьте эту ошибку.

\item \textbf{Интерактивный \texttt{rebase}}:\\
Сделайте интерактивный \texttt{rebase} ветки \texttt{casper} с момента отсоединения её от других веток. Склейте вместе коммиты ветки \texttt{casper}, так чтобы там осталось 4 коммита.

\item \textbf{Отмена интерактивного \texttt{rebase}}:\\
Отмените только что сделанный интерактивный \texttt{rebase}.

\item \textbf{\textbf{git add -p}}:\\
Перейдите на ветку \texttt{casper} и добавьте изменения в рабочую папку и в область индексирования. Изменить функции \texttt{add}, \texttt{factorial} и \texttt{is\_prime} в файле \texttt{arithmetic.py}, а также сортировку \texttt{bubble\_sort} в файле \texttt{sorting.py}. Добавьте в индекс только изменения функций \texttt{add} и \texttt{is\_prime}. Остальные изменения добавлять не нужно. Добавлять новый коммит не нужно.

\item \textbf{\textbf{git stash}}:\\
После изменений, сделанных в прошлой задаче, попробуйте переключиться на ветку \texttt{main}. Произойдёт ошибка, так как есть изменения в рабочей директории и в индексе. Используйте \texttt{git stash}, чтобы спрятать изменения. Перейдите на ветку \texttt{main} и сделайте там какой-нибудь коммит на ваш выбор. Вернитесь на ветку \texttt{casper} и вытащите изменения из \texttt{stash}.

\item \textbf{Перенос \textbf{git stash}}:\\
Снова добавьте эти же изменения в \texttt{stash}. Перейдите на ветку \texttt{bob} и вытащите изменения, сделанные на ветке \texttt{casper}, в ветку \texttt{bob}. Возможны конфликты. Сделайте коммит с этими изменениями в ветке \texttt{bob}.
\end{enumerate}


\subsection*{Низкоуровневый git}
В этом задании нельзя использовать команды, которые мы изучали прежде (кроме \texttt{git init}). Можно использовать только следующие низкоуровневые команды:
\begin{itemize}
\item git init
\item git hash-object
\item git cat-file
\item git ls-files
\item git mktree
\item git commit-tree
\item git update-ref 
\item git ls-tree
\end{itemize}
А также команды для проверки результатов:
\begin{itemize}
\item git status
\item git log
\item git show
\item git diff
\end{itemize}
Создайте репозиторий, состоящий из 3-х коммитов и 2-х веток, используя только эти команды.

\subsection*{* Ещё более низкоуровневый git}
Решите предыдущую задачу вообще без использования команд git. Можно только использовать команды для проверки результатов:
\begin{itemize}
\item git status
\item git log
\item git show
\item git diff
\end{itemize}
\end{document}
