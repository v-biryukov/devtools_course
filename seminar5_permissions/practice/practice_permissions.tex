\documentclass{article}
\usepackage[english,russian]{babel}
\usepackage{textcomp}
\usepackage{geometry}
  \geometry{left=2cm}
  \geometry{right=1.5cm}
  \geometry{top=1.5cm}
  \geometry{bottom=2cm}
\usepackage{tikz}
\usepackage{multicol}
\usepackage{hyperref}
\usepackage{listings}
\usepackage{pmboxdraw}
\usepackage{fancyvrb}
\usepackage[shortlabels]{enumitem}
\pagenumbering{gobble}

\lstdefinestyle{csMiptCStyle}{
  language=C,
  basicstyle=\linespread{1.1}\ttfamily,
  columns=fixed,
  fontadjust=true,
  basewidth=0.5em,
  keywordstyle=\color{blue}\bfseries,
  commentstyle=\color{gray},
  texcl=true,
  stringstyle=\ttfamily\color{orange!50!black},
  showstringspaces=false,
  numbersep=5pt,
  numberstyle=\tiny\color{black},
  numberfirstline=true,
  stepnumber=1,      
  numbersep=10pt,
  backgroundcolor=\color{white},
  showstringspaces=false,
  captionpos=b,
  breaklines=true
  breakatwhitespace=true,
  xleftmargin=.2in,
  extendedchars=\true,
  keepspaces = true,
  tabsize=4,
  upquote=true,
}


\lstdefinestyle{csMiptCLinesStyle}{
  style=csMiptCStyle,
  frame=lines,
}

\lstdefinestyle{csMiptCBorderStyle}{
  style=csMiptCStyle,
  framexleftmargin=5mm, 
  frame=shadowbox, 
  rulesepcolor=\color{gray}
}


\lstdefinestyle{csMiptBash}{
  	style=csMiptCStyle,
	breaklines=true,
	frame=tb,
	language=bash,
	breakatwhitespace=true,
	alsoletter={*()"'0123456789.},
	alsoother={\{\=\}},
	basicstyle={\ttfamily},
	keywordstyle={\bfseries},
	literate={{=}{{{=}}}1},
	prebreak={\textbackslash},
	sensitive=true,
	stepnumber=1,
	tabsize=4,
	morekeywords={echo, function},
	otherkeywords={-, \{, \}},
	literate={\$\{}{{{{\bfseries{}\$\{}}}}2,
	upquote=true,
	frame=none
}

\lstset{style=csMiptBash}
\lstset{
        literate={~}{{\raisebox{0.5ex}{\texttildelow}}}{1}
}

\renewcommand{\thesubsection}{\arabic{subsection}}
\makeatletter
\def\@seccntformat#1{\@ifundefined{#1@cntformat}%
   {\csname the#1\endcsname\quad}
   {\csname #1@cntformat\endcsname}}
\newcommand\section@cntformat{}     
\newcommand\subsection@cntformat{Задача \thesubsection.\space} 
\newcommand\subsubsection@cntformat{\thesubsubsection.\space}
\makeatother



\begin{document}
\title{Семинар \#5: Управление доступом. Практика. \vspace{-5ex}}\date{}\maketitle
\subsection*{Как сдавать задачи}
Для сдачи ДЗ вам нужно создать репозиторий на GitLab (если он ещё не создан) под названием \texttt{devtools-homework}. Структура репозитория должна иметь вид:
\begin{center}
\begin{BVerbatim}
├── seminar5_permissions/
│   ├── 01.txt
│   ├── 02.txt
│   └── ...
└── ...
\end{BVerbatim}
\end{center}
Для каждой задачи нужно создать 1 файл решения с расширением \texttt{.txt}. То есть всего в этом задании будет 10 файлов: \texttt{01.txt}, \texttt{02.txt}, ..., \texttt{10.txt}. Подзадачи внутри каждого из файлов нужно оформлять в следующем формате:
\begin{lstlisting}
# Subtask a
cat /etc/passwd
# Subtask b
cat /etc/passwd | grep root
# Subtask c
...
\end{lstlisting}
Для каждой подзадачи нужно прописать все команды, которые исполняются в ходе выполнения этой подзадачи.\\
Иногда в процессе решения вам потребуется вводить пароль. Чтобы указать это напишите соответствующий комментарий в файле решения:
\begin{lstlisting}
...
# Subtask с
su -
# Вводим пароль пользователя root
...
\end{lstlisting}
Иногда в процессе решения вам потребуется перейти на другой виртуальный терминал. Отметьте это в файле решения следующим образом:
\begin{lstlisting}
...
# Subtask с
# Переключаемся на tty3
alice
# Вводим пароль пользователя alice
whoami
exit
# Переключаемся на tty2
...
\end{lstlisting}
Иногда в процессе решения вам потребуется ответить на вопросы. Отметьте это в файле решения следующим образом:
\begin{lstlisting}
...
# Subtask e
a) да
b) нет
c) да

# Subtask f
...
\end{lstlisting}


\subsection*{Предварительные замечания}
\begin{itemize}
\item \textbf{Важно! Резервное клонирование ВМ.} \\
При выполнении данного задания придётся работать от имени суперпользователя (\texttt{root}), используя команды \texttt{su} и \texttt{sudo}. Выполняя такие команды, можно случайно сломать систему. Поэтому перед выполнением данного задания на всякий случай клонируйте вашу виртуальную машину. В VirtualBox для этого нужно нажать правой кнопкой мыши на виртуальной машине и выбрать "Клонировать".


\item \textbf{Активация root} \\
В только что установленной системе суперпользователь \texttt{root} может быть не активирован. Если это так и/или вы не знаете пароля от \texttt{root}, выполните:
\begin{lstlisting}
$ sudo passwd root
\end{lstlisting}
Команда \texttt{sudo} сначала попросит ввести ваш пароль, а потом выполнит команду \texttt{passwd root} от имени суперпользователя. Эта команда устанавливает пароль суперпользователя \texttt{root}.

\item \textbf{Простые пароли} \\
В процессе выполнения задания придётся часто переключаться между пользователями, поэтому советую в рамках этого задания устанавливать пользователям простые, запоминающиеся пароли.

\item \textbf{Начинайте каждую задачу, работая под своим основным аккаунтом} \\
Убедитесь, что вы переключились на основную учетную запись, прежде чем приступить к новой задаче.

\iffalse
\item \textbf{Создание нового пользователя} \\
Для проверки работы команд нам понадобится ещё один пользователь, помимо вашего основного пользователя и суперпользователя \texttt{root}. Создайте нового пользователя \texttt{alice}, используя следующие команды:
\begin{lstlisting}
sudo adduser -m alice
sudo passwd alice
\end{lstlisting}
\fi
\end{itemize}





\subsection{Просмотр системных файлов о пользователях и группах}
\begin{enumerate}[a.]
\item \textbf{Просмотр информации о пользователях}\\
Просмотрите файл \texttt{/etc/passwd}, содержащий информацию о пользователях:
\begin{lstlisting}
$ cat /etc/passwd
\end{lstlisting}
\item \textbf{Просмотр информации об одном пользователе}\\
Отфильтруйте информацию из файла \texttt{/etc/passwd} с помощью \texttt{grep} так, чтобы отобразить информацию только об одном пользователе \texttt{root}.

\item \textbf{Просмотр информации о группах}\\
Просмотрите файл \texttt{/etc/group}, содержащий информацию о группах пользователей.

\item \textbf{Просмотр информации о паролях пользователей}\\
Просмотрите файл \texttt{/etc/shadow}, содержащий информацию о паролях пользователей. В этом файле хранятся хэши паролей пользователей, а также сроки из действия.
У обычных пользователей нет прав даже на просмотр этого файла.
Поэтому его нужно просматривать от имени \texttt{root}, используя \texttt{sudo}:
\begin{lstlisting}
$ sudo cat /etc/shadow
\end{lstlisting}

\item \textbf{Просмотр информации о паролях групп}\\
Просмотрите файл \texttt{/etc/gshadow}, содержащий информацию о паролях групп.

\item \textbf{Просмотр настроек команды \texttt{sudo}}\\
Просмотрите файл \texttt{/etc/sudoers}, содержащий настройки команды \texttt{sudo}.

\item \textbf{Просмотр skeleton-директории}\\
Просмотрите директорию \texttt{/etc/skel}, используя \texttt{ls -la}. Эта директория содержит файлы-шаблоны, который копируются в домашний каталог нового пользователя. Просмотрите содержимое файла \texttt{.bashrc}, используя \texttt{less}. Это скрипт, который выполняется каждый раз при открытии окна терминала.

\item \textbf{Просмотр UID и GID}\\
Используйте команду \texttt{id}, чтобы просмотреть UID текущего пользователя и GID-ы всех его групп.

\item \textbf{Группы пользователя}\\
Используйте команду \texttt{groups}, чтобы просмотреть названия всех групп текущего пользователя.
\end{enumerate}

\subsection{Создание пользователей}
\begin{enumerate}[a.]
\item \textbf{Создание}\\
Создайте нового пользователя \texttt{alice} вместе с домашней директории \texttt{/home/alice} для этого пользователя. Используйте команду:
\begin{lstlisting}
$ sudo useradd alice -m
\end{lstlisting}
Просмотрите изменения, который произошли при выполнении этой команды:
\begin{itemize}
\item Новая запись в файле \texttt{/etc/passwd}.
\item Новая запись в файле \texttt{/etc/group}. Так как при создании пользователя создаётся новая группа из одного участника с таким же названием. Поэтому при создании пользователя \texttt{alice} автоматически создастся новая группа по имени \texttt{alice}, состоящая из одного участника -- \texttt{alice}.
\item Новая запись в файле \texttt{/etc/shadow}. Так как у пользователя ещё не установлен пароль, то учётная запись ещё считается заблокированной. На это будет указывать восклицательный знак после имени пользователя.
\item Новая папка в директории \texttt{/home} -- домашняя папка пользователя \texttt{alice}.
\item Содержимое папки \texttt{/home/alice} должно соответствовать скелетон-директории \texttt{/etc/skel}. Для просмотра директории \texttt{/home/alice} понадобятся права суперпользователя.
\item Используйте команду \texttt{id} чтобы более удобно посмотреть UID \texttt{alice} и \texttt{GID}-ы её групп. Используйте команду \texttt{groups}, чтобы посмотреть все группы, в которых состоит \texttt{alice}. Она пока должна состоять в одной группе -- \texttt{alice}.
\end{itemize}

\item \textbf{Установка пароля}\\
Установите пароль пользователя \texttt{alice}, используя команду:
\begin{lstlisting}
$ sudo passwd alice
\end{lstlisting}
Просмотрите файл \texttt{/etc/shadow} и убедитесь, что учётная запись \texttt{alice} стала активна (у пользователя \texttt{alice} должен пропасть символ \texttt{!} и появиться хэш пароля).

\item \textbf{Вход за нового пользователя}\\
Переключитесь на третий виртуальный терминал (\texttt{tty3}), используя комбинацию \texttt{Ctrl-Alt-F3}. Когда потребуется вернуться обратно, нужно будет переключиться на второй виртуальный терминал -- \texttt{Ctrl-Alt-F2}.

В \texttt{tty3} зайдите за нового пользователя \texttt{alice}. Сделайте следующее:
\begin{itemize}
\item Выполните \texttt{whoami}, чтобы убедится, что вы зашли за \texttt{alice}.
\item Выполните \texttt{pwd}, чтобы убедится, что вы находитесь в домашней папке \texttt{alice}.
\item Выполните команды \texttt{id} и \texttt{groups}, чтобы посмотреть UID пользователя и GID-ы и названия групп.
\item Выполните команду \texttt{w}, чтобы посмотреть всех работающих в данный момент на машине пользователей.
\item Выполните команду \texttt{echo \$SHELL}, чтобы посмотреть, какая оболочка используется.
\item Просмотрите файл \texttt{/etc/passwd}, используя \texttt{cat}.
\item Попробуйте просмотреть файл \texttt{/etc/shadow}, используя \texttt{sudo} и \texttt{cat}. У вас не удастся этого сделать, так как \texttt{alice} не находится в группе \texttt{sudo} (или \texttt{wheel}).
\item Создайте файл с любым именем в домашней директории \texttt{alice}.
\item Создайте файл \texttt{alice.txt} в директории \texttt{/tmp}. Это директория временных файлов в которой все пользователи могут создавать файлы. Файлы в этой директории удаляются после перезагрузки системы.
\item Выйдите из оболочки, используя \texttt{exit} или \texttt{Ctrl-D}, и вернитесь на второй виртуальный терминал.
\end{itemize}
\end{enumerate}


\subsection{Модификация пользователя}
\begin{enumerate}[a.]
\item \textbf{Изменение оболочки}\\
Измените оболочку пользователя \texttt{alice}. Если \texttt{alice} использует оболочку \texttt{/bin/sh}, то измените её на \texttt{/bin/bash}. Если же используется оболочка \texttt{/bin/bash} или другая, то измените её наоборот на \texttt{/bin/sh}. Используйте команду \texttt{usermod}. После изменения сделайте следующее:
\begin{itemize}
\item Посмотрите файл \texttt{/etc/passwd} и убедитесь, что оболочка изменилась.
\item Перейдите на пользователя \texttt{alice} в \texttt{tty3} и убедитесь, что используется другая оболочка. После этого вернитесь обратно на \texttt{tty2}.
\end{itemize}

\item \textbf{Изменение имени пользователя}\\
Измените имя пользователя с \texttt{alice} на \texttt{bob}, используя \texttt{usermod}.
\begin{itemize}
\item Посмотрите файл \texttt{/etc/passwd} и убедитесь, что имя изменилось.
\item Проверьте, изменилась ли при этом основная группа пользователя (используйте \texttt{id} и \texttt{groups}).
\item Проверьте, изменилась ли при этом домашняя директория пользователя.
\item Зайдите в пользователь \texttt{bob} через \texttt{tty3}. Проверьте в какой директории вы оказались. Проверьте кто является владельцем файла в этой директории. После этого вернитесь обратно на \texttt{tty2}.
\end{itemize}


\item \textbf{Изменение домашней директории}\\
Измените домашнюю директорию пользователя \texttt{bob} на \texttt{/home/bob} с переносом всех файлов из старой директории. Используйте \texttt{usermod}.

\item \textbf{Изменение основной группы пользователя}\\
Можно было бы переименовать группу \texttt{alice} на \texttt{bob} с помощью \texttt{groupmod}, но предположим, что мы хотим оставить группу \texttt{alice} существующей и с тем же именем. Вместо этого создадим новую группу и установим её основной для пользователя \texttt{bob}.
\begin{itemize}
\item Создайте новую группу по имени \texttt{bobg}:
\begin{lstlisting}
$ sudo groupadd bobg
\end{lstlisting}

\item Проверьте, что группа создалась, просмотрев файл \texttt{/etc/group}.
\item Измените основную группу пользователя \texttt{bob} на группу \texttt{bobg}.
\item Проверьте, что группа \texttt{bobg} теперь является основной группой \texttt{bob}, просмотрев файлы \texttt{/etc/passwd} и \texttt{/etc/group}. Обратите внимание, что основная группа указывается именно в файле \texttt{/etc/passwd} через её GID. А в файле \texttt{/etc/group} не указано, что \texttt{bob} входит в группу \texttt{bobg}. Там указывается принадлежность только к дополнительным группам.
\item Проверьте, что группа \texttt{bobg} теперь является основной группой \texttt{bob}, используя команды \texttt{id} и \texttt{groups}.
\end{itemize}

\item \textbf{Изменение дополнительных групп пользователя}
\begin{itemize}
\item Создайте новые группы по имени \texttt{students} и \texttt{devs}.
\item Добавьте эти группы как дополнительные группы пользователя \texttt{bob}.
\item Проверьте результат в файле \texttt{/etc/group}.
\item Проверьте результат используя \texttt{id} и \texttt{groups}.
\end{itemize}


\item \textbf{Добавление ещё одной дополнительной группы}\\
Пользователь \texttt{bob} сейчас не может использовать \texttt{sudo}, так как не входит в группу \texttt{sudo} (или \texttt{wheel}, в зависимости от того, какая группа используется на вашей системе) и не указан в файле \texttt{/etc/sudoers}.
\begin{itemize}
\item Добавьте пользователя \texttt{bob} в ещё одну дополнительную группу \texttt{sudo} (или \texttt{wheel}).
\item Используйте команду \texttt{groups}, чтобы проверить, что \texttt{bob} входит во все группы:
\begin{lstlisting}
$ groups
bobg students devs sudo
\end{lstlisting}
\end{itemize}
Переключитесь на пользователя \texttt{bob} и попробуйте посмотреть файл \texttt{/etc/shadow}, используя \texttt{sudo}.

\item \textbf{Изменение UID}\\
Проверьте владельца файла \texttt{/tmp/alice.txt}. Измените \texttt{uid} пользователя \texttt{bob}. Проверьте, изменился ли владелец у файла \texttt{/tmp/alice.txt}.
После изменения UID нужно будет изменить UID владельца у всех файлов в системе у которых владельцем был \texttt{bob}. Используйте команду \texttt{find} с опцией \texttt{-exec}.




\end{enumerate}

\subsection{Удаление и создание}
\begin{enumerate}[a.]
\item \textbf{Удаление группы}\\
Удалите группу \texttt{alice} из прошлой задачи используя \texttt{groupdel}.

\item \textbf{Удаление пользователя}\\
Удалите пользователя \texttt{bob} из прошлой задачи используя \texttt{userdel}.
\begin{lstlisting}
$ sudo userdel -r bob
\end{lstlisting}
Тут используем опцию \texttt{-r}, чтобы удалить не только пользователя, но и домашнюю директорию и некоторые другие файлы пользователя \texttt{bob}. 
\begin{itemize}
\item Проверьте, удалилась ли домашняя директория \texttt{/home/bob} в этом случае.
\item Проверьте, удалилась ли группа \texttt{bobg}. Это была основная группа пользователя \texttt{bob}, при этом больше никто в эту группу не входил. Если она не удалилась, удалите её.
\item Проверьте, удалились ли группа \texttt{students} и \texttt{devs}. Это были дополнительные группы пользователя \texttt{bob}, при этом больше никто в эти группы не входил. Их пока не удаляйте.
\end{itemize}

\item \textbf{Создание пользователя с опциями}\\
Создайте пользователя \texttt{alice} и сразу же при создании нужно сделать следующее:
\begin{itemize}
\item Задать путь до домашней папки как \texttt{/home/alices\_place}.
\item Создать домашнюю папку \texttt{/home/alices\_place}.
\item Задать оболочку пользователя \texttt{/bin/bash}.
\item Записать пользователя \texttt{alice} в дополнительные группы \texttt{students} и \texttt{devs}.
\item Задать UID пользователя значением 1234.
\end{itemize}
Всё это нужно сделать одной командой. После создания пользователя, просмотрите файл \texttt{/etc/passwd} и убедитесь, что все поля были заданы корректно. Убедитесь, что папка \texttt{/home/alices\_place} была создана. Также убедитесь, что автоматически была создана новая группа \texttt{alice}. 

\item \textbf{Ещё одно удаление пользователя}\\
Удалите пользователя \texttt{alice}, используя команду \texttt{sudo userdel -r alice}.
\begin{itemize}
\item Проверьте, удалилась ли группа \texttt{alice}. Это была основная группа пользователя \texttt{alice}, при этом больше никто в эту группу не входил. Если она не удалилась, удалите её.
\item Проверьте, удалились ли группа \texttt{students} и \texttt{devs}. Это были дополнительные группы пользователя \texttt{bob}, при этом больше никто в эти группы не входил. Если они не удалились, удалите их.
\end{itemize}

\end{enumerate}


\subsection{Блокировка}
\begin{itemize}
\item Создайте нового пользователя \texttt{alice} и установите пароль:
\begin{lstlisting}
$ sudo useradd -s /bin/bash -m alice
$ sudo passwd alice
\end{lstlisting}
\item Проверьте, что учётная запись \texttt{alice} активна, зайдя под пользователем \texttt{alice} на \texttt{tty3}. Вернитесь обратно.
\item Заблокируйте пользователя \texttt{alice}, используя \texttt{usermod}.
\item Посмотрите файл \texttt{/etc/shadow}. В записи \texttt{alice} должен появиться восклицательный знак. При этом хэш пароля должен всё ещё хранится в файле.
\item Попробуйте вновь зайти под пользователем \texttt{alice}.
\item Разблокируйте пользователя \texttt{alice}.
\item Снова проверьте файл \texttt{/etc/shadow}. Восклицательный знак в записи \texttt{alice} должен исчезнуть.
\item Попробуйте вновь зайти под пользователем \texttt{alice}.
\end{itemize}


\subsection{su и sudo}
\begin{enumerate}[a.]

\item \textbf{Просмотр переменных среды}
\begin{itemize}
\item Просмотрите все переменные среды, используя команду \texttt{env}.
\item Просмотрите переменную среды \texttt{USER}:
\begin{lstlisting}
$ echo $USER
\end{lstlisting}
\item Просмотрите также переменные среды \texttt{LOGNAME}, \texttt{HOME} и \texttt{PATH}.
\end{itemize}

\item \textbf{Неполное переключение}
\begin{itemize}
\item Переключитесь на пользователя \texttt{root}, используя команду \texttt{su}.
Введите пароль пользователя \texttt{root}.
\item Выполните команду \texttt{whoami}, чтобы напечатать имя текущего пользователя.
\item Просмотрите переменные среды \texttt{LOGNAME} и \texttt{PATH}.
\item Выйдите из под \texttt{root} выполнив \texttt{exit} или нажав \texttt{Ctrl-D}.
\end{itemize}

\item \textbf{Полное переключение}
\begin{itemize}
\item Переключитесь на пользователя \texttt{root}, используя команду \texttt{su -}.
Введите пароль пользователя \texttt{root}.
\item Выполните команду \texttt{whoami} и просмотрите переменные среды \texttt{LOGNAME} и \texttt{PATH}.
\item Выйдите из под \texttt{root}.
\end{itemize}

\item \textbf{Полное переключение через \texttt{sudo}}
\begin{itemize}
\item Переключитесь на пользователя \texttt{root}, используя команду \texttt{sudo -i}.
Введите пароль вашего текущего пользователя.
\item Выполните команду \texttt{whoami}, и просмотрите переменные среды \texttt{LOGNAME} и \texttt{PATH}.
\item Выйдите из под \texttt{root}.
\end{itemize}

\item \textbf{Полное переключение на пользователя \texttt{alice}}
\begin{itemize}
\item Сделайте полное переключение на пользователя \texttt{alice}
\item Выполните команду \texttt{whoami} и просмотрите переменные среды \texttt{LOGNAME} и \texttt{PATH}.
\item Выйдите из под \texttt{alice}.
\end{itemize}

\item \textbf{Выполнение одной команды от имени другого пользователя}\\
В этой подзадаче нужно не переключаться на других пользователей, а использовать \texttt{sudo} для выполнения команд от имени другого пользователя.
\begin{itemize}
\item Выполните команду \texttt{cat /etc/shadow} от имени пользователя \texttt{root}:
\begin{lstlisting}
$ sudo cat /etc/shadow
\end{lstlisting}

\item Перейдите в папку \texttt{/tmp} и создайте там файл \texttt{a.txt}. Используйте \texttt{ls -l}, чтобы посмотреть владельца этого файла и группу-владельца этого файла.

\item Создайте файл \texttt{/tmp/b.txt} от имени \texttt{root}. Посмотрите владельца этого файла и группу-владельца.

\item Создайте файл \texttt{/tmp/c.txt} от имени \texttt{alice}. Посмотрите владельца этого файла и группу-владельца.

\item Измените владельца файла \texttt{a.txt} на \texttt{alice}.
\item Измените группу владельца файла \texttt{a.txt} на \texttt{root}.
\end{itemize}

\item \textbf{Запрещённая команда}\\
Вы пытаетесь создать новый файл в системной директории \texttt{/etc}:
\begin{lstlisting}
$ touch /etc/myfile
\end{lstlisting}
но у вас ничего не выходит, так как у вас нет прав на это. Выполните эту команду, используя \texttt{sudo}.

\item \textbf{Запрещённая команда 2}\\
Вы пытаетесь записать в новый файл в системной директории \texttt{/etc}:
\begin{lstlisting}
$ echo "Hello" > /etc/myfile
\end{lstlisting}
но у вас ничего не выходит, так как у вас нет прав на это. Выполните эту команду, используя \texttt{sudo}.


\end{enumerate}

\subsection{Права доступа}
\begin{enumerate}[a.]
\item \textbf{Конвертация из символьного представления в числовое}\\
Конвертируйте запись следующих прав доступа из символьного представления в числовое:
\begin{lstlisting}
---------
rwx------
r--r--r--
r---w---x
rw-r--r--
rwxr-xr-x
rwxrwxrwx
\end{lstlisting}
Для решение этой подзадачи нужно в файле \texttt{07.txt} записать все ответы в виде:
\begin{lstlisting}
# Subtask a
000
700
...
\end{lstlisting}

\item \textbf{Просмотр прав файла}\\
В своей домашней директории создайте два файла \texttt{a.txt} и \texttt{b.txt}. Используйте команду \texttt{ls}, чтобы просмотреть права файла (и другую информацию) только файла \texttt{a.txt}.

\item \textbf{Просмотр прав директории}\\
В своей домашней директории создайте директорию \texttt{alpha}. Используйте команду \texttt{ls}, чтобы просмотреть права только директории \texttt{alpha}. Другие файлы при этом отображаться не должны.


\item \textbf{Изменение прав доступа}\\
Измените права доступа файлов \texttt{a.txt} и \texttt{b.txt} на нулевые:
\begin{lstlisting}
chmod 000 a.txt b.txt
\end{lstlisting}
После этого дайте файлу \texttt{a.txt} права доступа: \texttt{rwxr-xr-{}-}, используя синтаксис символьного представления прав доступа. Задачу нужно решить в одну строку.
Файлу \texttt{b.txt} также дайте права доступа: \texttt{rwxr-xr-{}-}, но используя числовое представление прав доступа. 

\item \textbf{Права доступа директории}\\
Переместите файлы \texttt{a.txt} и \texttt{b.txt} в директорию \texttt{alpha}. Измените права директории \texttt{alpha} на \texttt{-wx-wx-wx}. После этого ответьте на вопросы:
\begin{enumerate}
\item Можете ли вы просматривать директорию, используя \texttt{ls alpha}?
\item Можете ли вы создавать файлы в директории, используя \texttt{echo hello > alpha/c.txt}?
\item Можете ли вы изменять файлы в директории?
\item Можете ли вы заходить в директорию, используя \texttt{cd alpha}?
\item Можете ли вы зайти в директорию и просмотреть там файлы, используя \texttt{ls}?
\end{enumerate}
Просмотрите содержимое директории \texttt{alpha}, используя \texttt{sudo}.


\item \textbf{Права доступа директории 2}\\
В своей домашней директории создайте директорию \texttt{beta}. Внутри этой директории создайте файлы \texttt{a.txt} и \texttt{b.txt}. Измените права доступа директории \texttt{beta} на \texttt{rw-rw-rw-}. После этого ответьте на вопросы:
\begin{enumerate}
\item Можете ли вы просматривать директорию, используя \texttt{ls beta}?
\item Можете ли вы просматривать подробную информацию, используя \texttt{ls -la beta}?
\item Можете ли вы создавать файлы в директории, используя \texttt{echo hello > beta/c.txt}?
\item Можете ли вы изменять файлы в директории?
\item Можете ли вы заходить в директорию, используя \texttt{cd beta}?
\item Можете ли вы зайти в директорию и просмотреть там файлы, используя \texttt{ls}?
\end{enumerate}


\item \textbf{Файл в директории}\\
Создайте директорию \texttt{gamma}. В этой директории создайте файл \texttt{a.txt}. Файлу \texttt{gamma/a.txt} дайте права \texttt{-{}-{}-{}-{}-{}-{}-{}-{}-} (нулевые права). Директории \texttt{gamma} дайте права \texttt{rwxrwxrwx} (полные права). После этого ответьте на вопросы:
\begin{enumerate}
\item Можете ли вы просматривать содержимое файла \texttt{a.txt}?
\item Можете ли вы изменить файл \texttt{a.txt}?
\item Можете ли вы удалить файл \texttt{a.txt}?
\item Можете ли вы переименовать файл \texttt{a.txt}?
\end{enumerate}

\item \textbf{Вложенных директории}\\
Создайте директорию \texttt{one} и вложенную в неё директорию \texttt{one/two}. В директории \texttt{two} создайте файл \texttt{a.txt}. Дайте директории \texttt{two} права \texttt{rwxrwxrwx} (полные права), а директории \texttt{one} права \texttt{-{}-{}-{}-{}-{}-{}-{}-{}-} (нулевые права). У файла \texttt{a.txt} должны быть стандартные права \texttt{rw-rw-r-{}-}. После этого ответьте на вопросы:
\begin{enumerate}
\item Можете ли вы просматривать директорию \texttt{one}?
\item Можете ли вы просматривать директорию \texttt{two}?
\item Можете ли вы создавать файлы в директории \texttt{one}?
\item Можете ли вы создавать файлы в директории \texttt{two}?
\item Можете ли вы заходить в директорию \texttt{one}?
\item Можете ли вы заходить в директорию \texttt{two}?
\item Можете ли вы просмотреть содержимое файла \texttt{a.txt}?
\item Можете ли вы изменить файл \texttt{a.txt}?
\end{enumerate}

\item \textbf{Вложенных директории 2}\\
Измените права директории \texttt{one} на \texttt{-{}-x-{}-{}-{}-{}-{}-} и заново ответьте на те же вопросы.

\end{enumerate}



\subsection{Общая папка группы}

\begin{enumerate}[a.]
\item \textbf{Новые пользователи}\\
Создайте пользователя \texttt{alice}, если её не существует. Создайте новых пользователей \texttt{bob}, \texttt{casper} и \texttt{dany}. При создании каждого из пользователей необходимо также создать домашнюю директорию. В качестве оболочки для всех пользователей выберете \texttt{/bin/bash}.

\item \textbf{Новые группы}\\
Создайте новые группы \texttt{devs} и \texttt{students}. Добавьте пользователей \texttt{alice} и \texttt{bob} в группу \texttt{devs}, а пользователей \texttt{casper} и \texttt{dany} в группу \texttt{students}.


\item \textbf{Директория группы \texttt{devs}}\\
Создайте директорию \texttt{/home/devs} с владельцем \texttt{root} и группой-владельцем \texttt{devs}. Установите такие права доступа для этой папки, чтобы все пользователи группы \texttt{devs} обладали полными правами (\texttt{rwx}), а остальные пользователи -- нулевыми правами. 

\item \textbf{Создаём файлы}\\
Переключитесь на пользователя \texttt{alice} и создайте в папке \texttt{devs} следующие файлы:
\begin{itemize}
\item \texttt{a.txt} с владельцем \texttt{alice}, группой владельцем \texttt{alice} и правами \texttt{rw-rw-r}.
\item \texttt{b.txt} с владельцем \texttt{alice}, группой владельцем \texttt{devs} и правами \texttt{rw-rw-r}.
\item \texttt{c.txt} с владельцем \texttt{alice}, группой владельцем \texttt{devs} и правами \texttt{rw-r-{}-r}.
\end{itemize}
Для создания файла с другой группой можно использовать команды \texttt{chgrp} или \texttt{newgrp}.

\item \textbf{Права пользователя той же группы}\\
Переключитесь на пользователя \texttt{bob} и перейдите в папку \texttt{devs}. Ответьте на следующие вопросы:
\begin{enumerate}
\item Можете ли вы изменить файл \texttt{a.txt}?
\item Можете ли вы изменить файл \texttt{b.txt}?
\item Можете ли вы изменить файл \texttt{c.txt}?
\item Можете ли вы удалить файл \texttt{a.txt}?
\item Можете ли вы удалить файл \texttt{b.txt}?
\item Можете ли вы удалить файл \texttt{c.txt}?
\end{enumerate}


\item \textbf{Директория группы \texttt{students}. Sticky bit.}\\
Создайте директорию \texttt{/home/students} с владельцем \texttt{root} и группой-владельцем \texttt{students}. Дайте полные права к этой директории только группе \texttt{students}, а для остальных пользователей права сбросьте на нули. Дополнительно установите на директорию Sticky Bit.

\item \textbf{Ещё раз создаём файлы}\\
Переключитесь на пользователя \texttt{casper} и создайте в папке \texttt{students} следующие файлы:
\begin{itemize}
\item \texttt{a.txt} с владельцем \texttt{casper}, группой владельцем \texttt{casper} и правами \texttt{rw-rw-r}.
\item \texttt{b.txt} с владельцем \texttt{casper}, группой владельцем \texttt{students} и правами \texttt{rw-rw-r}.
\item \texttt{c.txt} с владельцем \texttt{casper}, группой владельцем \texttt{students} и правами \texttt{rw-r-{}-r}.
\end{itemize}

\item \textbf{Права пользователя той же группы. Sticky bit.}\\
Переключитесь на пользователя \texttt{dany} и перейдите в папку \texttt{students}. Ответьте на следующие вопросы:
\begin{enumerate}
\item Можете ли вы изменить файл \texttt{a.txt}?
\item Можете ли вы изменить файл \texttt{b.txt}?
\item Можете ли вы изменить файл \texttt{c.txt}?
\item Можете ли вы удалить файл \texttt{a.txt}?
\item Можете ли вы удалить файл \texttt{b.txt}?
\item Можете ли вы удалить файл \texttt{c.txt}?
\end{enumerate}

\item \textbf{SGID}\\
Проделайте следующие шаги:
\begin{itemize}
\item Зайдите за пользователя \texttt{casper} и перейдите в директорию \texttt{students}.
\item В директории \texttt{students} создайте папку \texttt{alpha}, файл \texttt{alpha/a.txt} и папку \texttt{alpha/beta}.
\item Добавьте к правам директории \texttt{students} SGID бит.
\item В директории \texttt{students} создайте папку \texttt{gamma}, файл \texttt{gamma/c.txt} и папку \texttt{gamma/delta}.
\item Проверьте группу-владельца у всех созданных файлов и папок.
\item Проверьте права доступа у всех созданных файлов и папок.
\end{itemize}



\item \textbf{Политика \texttt{sudo}}\\
На данный момент ни один из созданных нами пользователей не может использовать \texttt{sudo}, так как это не прописано в файле \texttt{/etc/sudoers}. Изменять этот файл обычным редактором может быть опасно, так как если вы сделаете ошибку, то программа \texttt{sudo} сломается. Рекомендуется использовать специальную программу \texttt{visudo}:
\begin{lstlisting}
$ EDITOR=nano sudo visudo
\end{lstlisting}

Измените файл \texttt{/etc/sudoers}, чтобы добавить следующие права для пользователей:
\begin{enumerate}[(a)]
\item Для всех пользователей группы \texttt{devs} -- права на использование \texttt{sudo} для запуска любых программ.
\item Для всех пользователей группы \texttt{students} -- права на использование \texttt{sudo} только для запуска программы \texttt{cat} и команд пакетного менеджера \texttt{apt update} и \texttt{apt install} (или \texttt{dnf update} и \texttt{dnf install} если в вашей системе используется \texttt{dnf}).
\end{enumerate}
Протестируйте изменения:
\begin{itemize}
\item Переключитесь на пользователя \texttt{alice} и создайте файл \texttt{alice.txt} в корневой директории.
\item Переключитесь на пользователя \texttt{casper} и попробуйте создать файл \texttt{casper.txt} в корневой директории, используя \texttt{sudo} и \texttt{touch}.
\item На пользователе \texttt{casper} просмотрите файл \texttt{/etc/shadow}.
\item На пользователе \texttt{casper} обновите списки доступных пакетов из репозиториев (\texttt{apt update}).
\end{itemize}

\end{enumerate}


\subsection{SUID}
В репозитории \texttt{devtools\_course} можно найти файл \texttt{seminar5\_permissions/practice/write\_time.c}. В нём содержится простая программа на языке C:
\begin{lstlisting}[style=csMiptCStyle]
#include <stdio.h>
#include <stdlib.h>
#include <time.h>

int main()
{
    FILE* fp = fopen("/mytimes.txt", "a");
    if (fp == NULL)
    {
        perror("Cannot open file");
        exit(1);
    }

    time_t t = time(NULL);
    fprintf(fp, "Hello, %ld seconds from 1970-01-01\n", (long)t);
    fclose(fp);
    return 0;
}
\end{lstlisting}
Данная программа делает следующее:
\begin{itemize}
\item Открывает файл \texttt{/mytimes.txt} в режиме дозаписи в конец файла (\texttt{"a"} -- append). Если такого файла не существует, то \texttt{fopen} сначала создаст новый пустой файл, а уже затем откроет его. Обратите внимание, что файл создаётся в корневой директории.
\item Если открыть или создать файл не получилось (например, если для этого недостаточно прав), то \texttt{fopen} вернёт значение \texttt{NULL} и на экран напечатается сообщение об ошибке.
\item Если же файл \texttt{/mytimes.txt} был успешно создан, то в конец этого файла запишется строка \texttt{"Hello, \%ld seconds from 1970-01-01"}, где за место \texttt{\%ld} будет подставлено количество секунд с 1-го января 1970-го года.
\end{itemize}
Проделайте следующие шаги:
\begin{enumerate}[(a)]
\item Если на вашей системе не установлен \texttt{gcc}, то установите его:
\begin{lstlisting}
$ sudo apt install gcc
\end{lstlisting}

\item Скопируйте код программы в вашу систему linux и скомпилируйте его с помощью компилятора \texttt{gcc}:
\begin{lstlisting}
$ gcc write_time.c
\end{lstlisting}
После этого в текущей директории появится исполняемый файл \texttt{a.out}.

\item Измените владельца исполняемого файла \texttt{a.out} на пользователя \texttt{root}, а группу-владельца на группу \texttt{root}. 
\item Запустите \texttt{a.out} от имени root, используя \texttt{sudo}. Проверьте, что файл \texttt{/mytimes.txt} был создан и содержит одну строку.
\item Попробуйте запустить \texttt{a.out} в вашем текущем пользователе. У вашего пользователя нет прав для создания файла в корневой директории и программа должна завершиться с ошибкой.
\item Дайте исполняемому файлу \texttt{a.out} права \texttt{SUID}.
\item Запустите \texttt{a.out} от имени вашего пользователя без использования \texttt{sudo}. Проверьте, что новая строка была добавлена в файл.
\end{enumerate}


\subsection{Права доступа вместе с SUID, SGID и Sticky Bit}

\begin{enumerate}[a.]
\item \textbf{Конвертация из символьного представления в числовое}\\
Конвертируйте запись следующих прав доступа из символьного представления в числовое:
\begin{lstlisting}
--------T
-----S---
--S------
--------t
-----s---
--t------
--S--S--T
--s--s--t
rwsr-xr-x
rwxrwsr-x
rwxrwxrwt
rwsrwsrwt
\end{lstlisting}
Для решение этой подзадачи нужно в файле \texttt{10.txt} записать все ответы в виде:
\begin{lstlisting}
# Subtask a
1000
...
\end{lstlisting}

\item \textbf{Просмотр прав доступа системных файлов и папок}\\
Используйте \texttt{ls}, чтобы просмотреть владельцев, групп-владельцев и права доступа следующих системных файлов и папок:
\begin{itemize}
\item \texttt{/}
\item \texttt{/etc/passwd}
\item \texttt{/etc/shadow}
\item \texttt{/bin/cat}
\item \texttt{/bin/passwd}
\item \texttt{/bin/sudo}
\item \texttt{/bin/su}
\item \texttt{/bin/chage}
\item \texttt{/tmp}
\item \texttt{/var/mail}
\end{itemize}
\end{enumerate}


\iffalse
\newpage
~
\newpage
\subsection*{Идеи}
\begin{enumerate}[a.]

\item \textbf{Просмотр прав}\\
Просмотр прав разных системных файлов и директорий.



\item \textbf{Изменение UID}\\
Выполните команду, которая изменяет UID пользователя + переписывает UID всех файлов в системе, принадлежащие этому пользователю. 

\item \textbf{Владелец, группа владельца}\\
Два пользователя \texttt{alice} и \texttt{bob} с одноименными основными группами. \texttt{alice} не входит в группу \texttt{bob}, а \texttt{bob} не входит в группу \texttt{alice}. Создайте файл, такой чтобы её владельцем была \texttt{alice}, а группа-владелец была группа \texttt{bob}.

\item \textbf{Папки /one/two}\\
--- rwx
--x rwx

\item \textbf{Программа stat}\\
Используйте \texttt{stat} чтобы напечатать информацию о файле \texttt{a.txt}.

\item \textbf{Просмотр прав с помощью stat}\\
Используйте \texttt{stat} чтобы напечатать: 
\begin{itemize}
\item только права пользователя в символьном представлении
\item только права пользователя в числовом представлении
\end{itemize}

\item Папка с правами rw-rw-rw-.\\
можно ли в посмотреть её содержимое, создать в ней файл.


\item Папка с w, а файл без прав для данного пользователя.\\
Пользователь может удалять файлы, но не может изменять их.


\item \textbf{простая}\\
Файлы с w но без r.


\item \texttt{newgrp}\\

\item Пароли групп\\

\item Исполняемый файл с SUID\\


\item Найти все файлы в системе, принадлежащие данному пользователю (системному)\\

\end{enumerate}
\fi

\end{document}
