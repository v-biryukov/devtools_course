\documentclass{article}
\usepackage[english,russian]{babel}
\usepackage{textcomp}
\usepackage{geometry}
  \geometry{left=2cm}
  \geometry{right=1.5cm}
  \geometry{top=1.5cm}
  \geometry{bottom=2cm}
\usepackage{tikz}
\usepackage{multicol}
\usepackage{hyperref}
\usepackage{listings}
\pagenumbering{gobble}

\lstdefinestyle{csMiptCppStyle}{
  language=C++,
  basicstyle=\linespread{1.1}\ttfamily,
  columns=fixed,
  fontadjust=true,
  basewidth=0.5em,
  keywordstyle=\color{blue}\bfseries,
  commentstyle=\color{gray},
  texcl=true,
  stringstyle=\ttfamily\color{orange!50!black},
  showstringspaces=false,
  numbersep=5pt,
  numberstyle=\tiny\color{black},
  numberfirstline=true,
  stepnumber=1,      
  numbersep=10pt,
  backgroundcolor=\color{white},
  showstringspaces=false,
  captionpos=b,
  breaklines=true
  breakatwhitespace=true,
  xleftmargin=.2in,
  extendedchars=\true,
  keepspaces = true,
  tabsize=4,
  upquote=true,
}


\lstdefinestyle{csMiptCppLinesStyle}{
  style=csMiptCppStyle,
  frame=lines,
}

\lstdefinestyle{csMiptCppBorderStyle}{
  style=csMiptCppStyle,
  framexleftmargin=5mm, 
  frame=shadowbox, 
  rulesepcolor=\color{gray}
}


\lstdefinestyle{csMiptBash}{
breaklines=true,
frame=tb,
language=bash,
breakatwhitespace=true,
alsoletter={*()"'0123456789.},
alsoother={\{\=\}},
basicstyle={\ttfamily},
keywordstyle={\bfseries},
literate={{=}{{{=}}}1},
prebreak={\textbackslash},
sensitive=true,
stepnumber=1,
tabsize=4,
morekeywords={echo, function},
otherkeywords={-, \{, \}},
literate={\$\{}{{{{\bfseries{}\$\{}}}}2,
upquote=true,
frame=none
}


\lstset{style=csMiptCppLinesStyle}
\lstset{literate={~}{{\raisebox{0.5ex}{\texttildelow}}}{1}}


\renewcommand{\thesection}{\arabic{section}}
\makeatletter
\def\@seccntformat#1{\@ifundefined{#1@cntformat}%
   {\csname the#1\endcsname\quad}%    default
   {\csname #1@cntformat\endcsname}}% enable individual control
\newcommand\section@cntformat{Часть \thesection:\space}
\makeatother



\begin{document}
\title{Семинар \#5: Права доступа \vspace{-5ex}}\date{}\maketitle

\subsection*{Просмотр информации в системных файлах}
\texttt{
\begin{flushleft}
\begin{tabular}{ p{4cm} | l }
 cat /etc/passwd & просмотр файла с пользователями \\ 
 cat /etc/group  & просмотр файла с группами \\ 
 sudo cat /etc/shadow  & просмотр файла с хэшированными паролями пользователей \\ 
 sudo cat /etc/gshadow  & просмотр файла с хэшированными паролями групп \\ 
 sudo cat /etc/sudoers  & просмотр файла с настройками sudo \\ 
 ls -la /etc/skel       & просмотр файлов, создаваемых в домашней директории нового пользователя
\end{tabular}
\end{flushleft}
}

\subsection*{Создание и удаление пользователей}
Все эти команды требуют прав \texttt{root}. Нужно использовать \texttt{sudo} или \texttt{su}.
\texttt{
\begin{flushleft}
\begin{tabular}{ p{4cm} | l }
 useradd alice 		& создаёт пользователя alice без создание домашней папки\\ \hline
 опции:~ -m      	& создаёт пользователя alice и домашнюю папку /home/alice\\ 
 ~~~~~~~ -s оболочка & указать оболочку \\
 ~~~~~~~ -g группа  & указать основную группу (по имени или GID) \\
 ~~~~~~~ -G группы  & указать дополнительные группы (через запятую, по имени или GID) \\
 ~~~~~~~ -d папка   & задать нестандартный путь до домашней папки\\
 ~~~~~~~ -k папка   & задать нестандартный skeleton-каталог\\ 
 ~~~~~~~ -u число   & задаёт UID пользователя\\ 
 ~~~~~~~ -e 2024-12-31 & задать срок действия учётной записи\\ \hline
 userdel alice		& удаляет пользователя alice, без удаления домашней папки\\ \hline
 userdel -r alice	& удаляет пользователя alice и домашнюю папку\\ 
 					& также удаляет файлы почты из /var/mail/alice\\
 					& также удаляет временные файлы из /tmp и /var/tmp\\\hline
\end{tabular}
\end{flushleft}
}
\noindent Например, \texttt{sudo useradd -s /bin/bash -m alice} создаст \texttt{alice} с домашней директорией и оболочкой \texttt{/bin/bash}.

\subsection*{Изменение пользователей}
Все эти команды требуют прав \texttt{root}. Нужно использовать \texttt{sudo} или \texttt{su}.
\texttt{
\begin{flushleft}
\begin{tabular}{ p{4cm} | l }
 usermod 			& \\ \hline
 опции:~ -s оболочка	& изменяет оболочку пользователя\\ 
 ~~~~~~~ -l имя 		& изменяет имя пользователя\\ 
 ~~~~~~~ -d папка 		& изменяет расположение домашней папки\\ 
 ~~~~~~~ -d папка -m 	& изменяет домашнюю папку и переносит туда файлы из старой\\
 ~~~~~~~ -g группа 		& изменяет основную группу\\ 
 ~~~~~~~ -G группы 		& изменяет дополнительные группы, группы перечисляются через запятую\\ 
 ~~~~~~~ -aG группы 	& добавляет дополнительные группы\\ 
 ~~~~~~~ -u число 		& изменяет UID\\ \hline
\end{tabular}
\end{flushleft}
}
\noindent Например, \texttt{sudo usermod -s /bin/bash} изменит оболочку на \texttt{/bin/bash}.

\subsection*{Удобные команды для создания и удаления пользователей}
Все эти команды требуют прав \texttt{root}. Нужно использовать \texttt{sudo} или \texttt{su}.
\texttt{
\begin{flushleft}
\begin{tabular}{ p{4cm} | l }
 adduser alice		& интерактивное создание пользователя, создаёт домашнюю папку\\ \hline	
 deluser alice		& удаляет пользователя alice, но не домашнюю папку\\
 deluser -{}-remove-home alice & удаляет пользователя alice с домашней папкой\\
\end{tabular}
\end{flushleft}
}

\subsection*{Выполнение команд от имени другого пользователя}
\texttt{
\begin{flushleft}
\begin{tabular}{ p{4cm} | l }
 su				& переключиться на пользователя root (Ctrl-D или exit для выхода)\\
 				& сохраняется текущее окружение \\
 				& нужно будет ввести пароль root\\
 su -			& Как su, но сбрасывает и загружает новое окружение \\ \hline
 su	alice 		& переключиться на пользователя alice\\
 				& нужно будет ввести пароль alice\\
 su - alice		& Как su alice, но сбрасывает и загружает новое окружение \\ \hline
 sudo команда 	& выполнить команду от имени пользователя root, \\
 				& нужно будет ввести пароль текущего пользователя\\
 sudo -u alice команда 	& выполнить команду от имени пользователя alice\\
 				& нужно будет ввести пароль текущего пользователя\\ \hline
 sudo -i 		& переключиться на пользователя root\\
 				& похоже на su -, но нужно вводить пароль текущего пользователя, а не root\\
 sudo -i -u alice 		& переключиться на пользователя alice\\
\end{tabular}
\end{flushleft}
}
\noindent \texttt{sudo} для текущего пользователя будет работать только если это указано в файле \texttt{/etc/sudoers}.

\end{document}
