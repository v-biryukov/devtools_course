\documentclass{article}
\usepackage[english,russian]{babel}
\usepackage{textcomp}
\usepackage{geometry}
  \geometry{left=2cm}
  \geometry{right=1.5cm}
  \geometry{top=1.5cm}
  \geometry{bottom=2cm}
\usepackage{tikz}
\usepackage{multicol}
\usepackage{hyperref}
\usepackage{listings}
\pagenumbering{gobble}

\lstdefinestyle{csMiptCppStyle}{
  language=C++,
  basicstyle=\linespread{1.1}\ttfamily,
  columns=fixed,
  fontadjust=true,
  basewidth=0.5em,
  keywordstyle=\color{blue}\bfseries,
  commentstyle=\color{gray},
  texcl=true,
  stringstyle=\ttfamily\color{orange!50!black},
  showstringspaces=false,
  numbersep=5pt,
  numberstyle=\tiny\color{black},
  numberfirstline=true,
  stepnumber=1,      
  numbersep=10pt,
  backgroundcolor=\color{white},
  showstringspaces=false,
  captionpos=b,
  breaklines=true
  breakatwhitespace=true,
  xleftmargin=.2in,
  extendedchars=\true,
  keepspaces = true,
  tabsize=4,
  upquote=true,
}


\lstdefinestyle{csMiptCppLinesStyle}{
  style=csMiptCppStyle,
  frame=lines,
}

\lstdefinestyle{csMiptCppBorderStyle}{
  style=csMiptCppStyle,
  framexleftmargin=5mm, 
  frame=shadowbox, 
  rulesepcolor=\color{gray}
}


\lstdefinestyle{csMiptBash}{
breaklines=true,
frame=tb,
language=bash,
breakatwhitespace=true,
alsoletter={*()"'0123456789.},
alsoother={\{\=\}},
basicstyle={\ttfamily},
keywordstyle={\bfseries},
literate={{=}{{{=}}}1},
prebreak={\textbackslash},
sensitive=true,
stepnumber=1,
tabsize=4,
morekeywords={echo, function},
otherkeywords={-, \{, \}},
literate={\$\{}{{{{\bfseries{}\$\{}}}}2,
upquote=true,
frame=none
}




\lstset{style=csMiptCppLinesStyle}
\lstset{literate={~}{{\raisebox{0.5ex}{\texttildelow}}}{1}}


\renewcommand{\thesection}{\arabic{section}}
\makeatletter
\def\@seccntformat#1{\@ifundefined{#1@cntformat}%
   {\csname the#1\endcsname\quad}%    default
   {\csname #1@cntformat\endcsname}}% enable individual control
\newcommand\section@cntformat{Часть \thesection:\space}
\makeatother




\begin{document}
\title{Семинар \#0: Основы терминала Linux и основы git \vspace{-5ex}}\date{}\maketitle

\subsection*{Задание на работу с терминалом}
\begin{enumerate}
\item Откройте терминал и узнайте в какой папке вы находитесь. Для этого напечатайте \texttt{pwd} и нажмите \texttt{Enter}.
\item Перейдите в папку  \texttt{/home-local/student}. Для этого введите команду:
\begin{verbatim}
cd /home-local/student
\end{verbatim}
\item С помощью команды \texttt{pwd} проверьте, что вы действительно находитесь в нужной папке.
\item С помощью команды \texttt{ls} просмотрите всё содержимое папки \texttt{/home-local/student}. Для этого введите \texttt{ls} и нажмите \texttt{Enter}.
\item Создайте вашу папку. Используйте команду:
\begin{verbatim}
mkdir имя_папки
\end{verbatim}
За место \texttt{имя\_папки}  подставьте название вашей папки. Желательно, чтобы название содержало только латинские символы или цифры без пробелов.
\item С помощью команды \texttt{ls} убедитесь, что ваша папка создалась.
\item Перейдите в вашу созданную папку командой \texttt{cd имя\_папки}.

\item Перейдите в эту папку с помощью файлового менеджера(проводника) и создайте там файл \texttt{hello.c}. Файл обязан оканчиваться на \texttt{.c}.

\item С помощью обычного текстового редактора (например, gedit или Sublime Text) напишите в файле \texttt{hello.c} текст программы \textit{HelloWorld}:
\begin{lstlisting}
#include <stdio.h>
int main() 
{
    printf("Hello World\n");
}
\end{lstlisting}

\item В терминале проверьте, что этот файл существует, используя команду \texttt{ls}.
\item Скомпилируйте этот файл следующей командой:
\begin{verbatim}
gcc hello.c
\end{verbatim}
После этого в папке создастся новый файл по имени \texttt{a.out}.
\item Запустите исполняемый файл \texttt{a.out} напечатав полный путь до этого файла:
\begin{verbatim}
/home/имя_пользователя/имя_папки/a.out
\end{verbatim}
\item Точка в имени файлового пути является сокращением для текущей папки. То есть в данном случае точка является сокращением для \texttt{/home-local/student/ваша\_папка}. Поэтому команду для запуска файла \texttt{a.out} можно сократить до:
\begin{verbatim}
./a.out
\end{verbatim}
\item Можно объединить команды компиляции и запуска:
\begin{verbatim}
gcc hello.c && ./a.out
\end{verbatim}

\end{enumerate}

\newpage
\subsection*{Задание на работу с git}

\begin{enumerate}
\item Создайте новый git репозиторий.
\item Создайте два новых файла \texttt{cat.txt} и \texttt{dog.txt}. Содержимое файлов можете задать сами.
\item Используйте \texttt{git status}, чтобы посмотреть изменения.
\item Добавьте эти файлы в индекс.
\item Используйте \texttt{git status}, чтобы посмотреть изменения.
\item Добавьте файлы из индекса в локальный репозиторий.
\item Создайте новый файл \texttt{mouse.txt} и измените уже существующий файл \texttt{cat.txt}.
\item Используйте \texttt{git status}, чтобы посмотреть изменения.
\item Добавьте изменения в индекс.
\item Используйте \texttt{git status}, чтобы посмотреть изменения.
\item Добавьте изменения из индекса в локальный репозиторий.
\item Удалите файл \texttt{dog.txt} и добавьте это изменение сначала в индекс, а потом в локальный репозиторий.
\item Сделайте ещё одно любое изменение и добавьте это изменение сначала в индекс, а потом в локальный репозиторий.
\item Используйте команду \texttt{git log} чтобы посмотреть сделанные изменения.
\item Используйте команду \texttt{git switch}, чтобы перейти на второй коммит.
\item Используйте команду \texttt{git switch}, чтобы вернуться обратно на самый последний коммит.
\item Создайте свой новый пустой репозиторий на gitlab.
\item Свяжите свой локальный репозиторий с удалённым с помощью \texttt{git remote add}.
\item Добавьте изменения из локального репозитория в репозиторий на gitlab с помощью \texttt{git push} (или \texttt{git push -f}).
\item Сделайте ещё одно изменение и добавьте его сначала в индекс, затем локальный репозиторий, а затем в удалённый репозиторий.
\end{enumerate}

\subsection*{Задание совместную работу с репозиторием}
Сделайте задание по адресу  \texttt{\href{https://mipt-hsse.gitlab.yandexcloud.net/pro100savant/hellow_word_2025}{mipt-hsse.gitlab.yandexcloud.net/pro100savant/hellow\_word\_2025}}.


\end{document}
