 \documentclass{article}
\usepackage[english,russian]{babel}
\usepackage{textcomp}
\usepackage{geometry}
  \geometry{left=2cm}
  \geometry{right=1.5cm}
  \geometry{top=1.5cm}
  \geometry{bottom=2cm}
\usepackage{tikz}
\usepackage{multicol}
\usepackage{hyperref}
\usepackage{listings}
\usepackage{upquote}
\pagenumbering{gobble}

\lstdefinestyle{csMiptCppStyle}{
  language=C++,
  basicstyle=\linespread{1.1}\ttfamily,
  columns=fixed,
  fontadjust=true,
  basewidth=0.5em,
  keywordstyle=\color{blue}\bfseries,
  commentstyle=\color{gray},
  texcl=true,
  stringstyle=\ttfamily\color{orange!50!black},
  showstringspaces=false,
  numbersep=5pt,
  numberstyle=\tiny\color{black},
  numberfirstline=true,
  stepnumber=1,      
  numbersep=10pt,
  backgroundcolor=\color{white},
  showstringspaces=false,
  captionpos=b,
  breaklines=true
  breakatwhitespace=true,
  xleftmargin=.2in,
  extendedchars=\true,
  keepspaces = true,
  tabsize=4,
  upquote=true,
}


\lstdefinestyle{csMiptCppLinesStyle}{
  style=csMiptCppStyle,
  frame=lines,
}

\lstdefinestyle{csMiptCppBorderStyle}{
  style=csMiptCppStyle,
  framexleftmargin=5mm, 
  frame=shadowbox, 
  rulesepcolor=\color{gray}
}


\lstdefinestyle{csMiptBash}{
breaklines=true,
frame=tb,
language=bash,
breakatwhitespace=true,
alsoletter={*()"'0123456789.},
alsoother={\{\=\}},
basicstyle={\ttfamily},
keywordstyle={\bfseries},
literate={{=}{{{=}}}1},
prebreak={\textbackslash},
sensitive=true,
stepnumber=1,
tabsize=4,
morekeywords={echo, function},
otherkeywords={-, \{, \}},
literate={\$\{}{{{{\bfseries{}\$\{}}}}2,
upquote=true,
frame=none
}


\lstset{style=csMiptCppLinesStyle}
\lstset{literate={~}{{\raisebox{0.5ex}{\texttildelow}}}{1}}


\renewcommand{\thesection}{\arabic{section}}
\makeatletter
\def\@seccntformat#1{\@ifundefined{#1@cntformat}%
   {\csname the#1\endcsname\quad}%    default
   {\csname #1@cntformat\endcsname}}% enable individual control
\newcommand\section@cntformat{Часть \thesection:\space}
\makeatother



\begin{document}
\title{Семинар \#6: Диски и файловые системы \vspace{-5ex}}\date{}\maketitle

\subsection*{Единицы измерения информации}
\renewcommand{\arraystretch}{1.3}
\begin{center}
\begin{tabular}{ l | l | l  | l  | l  }
 единица & единица (англ.) & сокращение & сокращение (англ.) & величина   \\ \hline
 байт & byte & Б & B & $1$ байт   \\ \hline
 килобайт & kilobyte & КБ & KB    & $1000 = 10^3$ байт   \\
 мегабайт & megabyte & МБ & MB    & $1000^2 = 10^6$ байт   \\
 гигабайт & gigabyte & ГБ & GB    & $1000^3 = 10^9$ байт   \\
 терабайт & terabyte & ТБ & TB    & $1000^4 = 10^{12}$ байт   \\ \hline
 кибибайт & kibibyte & КиБ & KiB  & $1024 = 2^{10}$ байт   \\
 мебибайт & mebibyte & МиБ & MiB  & $1024^2 = 2^{20} = 1,048,576$ байт   \\
 гибибайт & gibibyte & ГиБ & GiB  & $1024^3 = 2^{30} = 1,073,741,824$ байт   \\
 тебибайт & tebibyte & ТиБ & TiB  & $1024^4 = 2^{40} = 1,099,511,627,776$ байт   \\ \hline
\end{tabular}
\end{center}
\renewcommand{\arraystretch}{1.1}

\subsection*{Просмотр информации в дисках и разделах}
\texttt{
\begin{flushleft}
\begin{tabular}{ p{4cm} | l }
 lsblk & просмотр информации о блочных устройствах (дисках и разделах) \\ \hline
 lsblk -f & просмотр UUID блочных устройств и тип файловых систем  \\ 
 blkid & просмотр UUID блочных устройств и тип файловых систем \\ \hline
 findmnt & просмотр всех смонтированных в данный момент файловых систем\\ \hline
 df -h & просмотр использованного места на файловых системах  \\ \hline
 fdisk -l & просмотр таблицы разделов и информации о разделах  \\
 parted -l & просмотр таблицы разделов и информации о разделах  \\
\end{tabular}
\end{flushleft}
}


\subsection*{Разметка диска с помощью fdisk}


\subsection*{Разметка диска с помощью parted}


\subsection*{Создание файловых систем}
\texttt{
\begin{flushleft}
\begin{tabular}{ p{4cm} | l }
 sudo mkfs.ext4 /dev/sdb1 & создаёт ФС ext4 в разделе /dev/sdb1 \\ \hline
 sudo mkfs.xfs /dev/sdb1 & создаёт ФС xfs в разделе /dev/sdb1 \\ \hline
 sudo mkfs.btrfs /dev/sdb1 & создаёт ФС btrfs в разделе /dev/sdb1 \\ \hline
 sudo mkfs.ext4 /dev/sdb1 & создаёт ФС ext4 в разделе /dev/sdb1 \\ \hline
 sudo mkfs.ext4 /dev/sdb1 & создаёт ФС ext4 в разделе /dev/sdb1 \\ \hline
\end{tabular}
\end{flushleft}
}



\end{document}
