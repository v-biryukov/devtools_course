\documentclass{article}
\usepackage[english,russian]{babel}
\usepackage{textcomp}
\usepackage{geometry}
  \geometry{left=2cm}
  \geometry{right=1.5cm}
  \geometry{top=1.5cm}
  \geometry{bottom=2cm}
\usepackage{tikz}
\usepackage{multicol}
\usepackage{hyperref}
\usepackage{listings}
\usepackage{pmboxdraw}
\usepackage{fancyvrb}
\usepackage[shortlabels]{enumitem}
\pagenumbering{gobble}

\lstdefinestyle{csMiptCStyle}{
  language=C,
  basicstyle=\linespread{1.1}\ttfamily,
  columns=fixed,
  fontadjust=true,
  basewidth=0.5em,
  keywordstyle=\color{blue}\bfseries,
  commentstyle=\color{gray},
  texcl=true,
  stringstyle=\ttfamily\color{orange!50!black},
  showstringspaces=false,
  numbersep=5pt,
  numberstyle=\tiny\color{black},
  numberfirstline=true,
  stepnumber=1,      
  numbersep=10pt,
  backgroundcolor=\color{white},
  showstringspaces=false,
  captionpos=b,
  breaklines=true
  breakatwhitespace=true,
  xleftmargin=.2in,
  extendedchars=\true,
  keepspaces = true,
  tabsize=4,
  upquote=true,
}


\lstdefinestyle{csMiptCLinesStyle}{
  style=csMiptCStyle,
  frame=lines,
}

\lstdefinestyle{csMiptCBorderStyle}{
  style=csMiptCStyle,
  framexleftmargin=5mm, 
  frame=shadowbox, 
  rulesepcolor=\color{gray}
}


\lstdefinestyle{csMiptBash}{
  	style=csMiptCStyle,
	breaklines=true,
	frame=tb,
	language=bash,
	breakatwhitespace=true,
	alsoletter={*()"'0123456789.},
	alsoother={\{\=\}},
	basicstyle={\ttfamily},
	keywordstyle={\bfseries},
	literate={{=}{{{=}}}1},
	prebreak={\textbackslash},
	sensitive=true,
	stepnumber=1,
	tabsize=4,
	morekeywords={echo, function},
	otherkeywords={-, \{, \}},
	literate={\$\{}{{{{\bfseries{}\$\{}}}}2,
	upquote=true,
	frame=none
}

\lstset{style=csMiptBash}
\lstset{
        literate={~}{{\raisebox{0.5ex}{\texttildelow}}}{1}
}

\renewcommand{\thesubsection}{\arabic{subsection}}
\makeatletter
\def\@seccntformat#1{\@ifundefined{#1@cntformat}%
   {\csname the#1\endcsname\quad}
   {\csname #1@cntformat\endcsname}}
\newcommand\section@cntformat{}     
\newcommand\subsection@cntformat{Задача \thesubsection.\space} 
\newcommand\subsubsection@cntformat{\thesubsubsection.\space}
\makeatother



\begin{document}
\title{Семинар \#6: Диски и файловые системы. Практика. \vspace{-5ex}}\date{}\maketitle
\subsection*{Как сдавать задачи}
Для сдачи ДЗ вам нужно создать репозиторий на GitLab (если он ещё не создан) под названием \texttt{devtools-homework}. Структура репозитория должна иметь вид:
\begin{center}
\begin{BVerbatim}
├── seminar6_disks_and_filesystems/
│   ├── 01.txt
│   ├── 02.sh
│   └── ...
└── ...
\end{BVerbatim}
\end{center}
Для каждой задачи, если не сказано иное, нужно создать 1 файл решения с расширением \texttt{.sh}. Если в задаче есть подзадачи, то их нужно оформлять внутри каждого из файлов в следующем формате:
\begin{lstlisting}
# Subtask a
lsblk
# Subtask b
...
\end{lstlisting}
Для каждой подзадачи нужно прописать все команды, которые исполняются в ходе выполнения этой подзадачи. Если в задаче встречается вопрос, то на этот вопрос нужно ответить в комментариях (начинаются с \texttt{\#}) скрипта.



\subsection*{Предварительные замечания}
\begin{itemize}
\item \textbf{Важно! Резервное клонирование ВМ.} \\
При выполнении данного задания придётся работать от имени суперпользователя (\texttt{root}), используя команды \texttt{su} и \texttt{sudo}. Выполняя такие команды, можно случайно сломать систему. Поэтому перед выполнением данного задания на всякий случай клонируйте вашу виртуальную машину. В VirtualBox для этого нужно нажать правой кнопкой мыши на виртуальной машине и выбрать "Клонировать".\\
\textbf{Это задание нужно делать только в виртуальной машине!}



\item \textbf{Создание диска для виртуальной машины.} \\
Это задание нужно выполнять в виртуальной машине. Для выполнения этого задания вам понадобиться создать новый виртуальный диск. В \texttt{Virtual Box} это можно сделать следующим образом:
\begin{itemize}
\item Выключите виртуальную машину, если она включена.
\item Выберите вашу виртуальную машину и нажмите настроить.
\item Выберите вкладку "Носители".
\item Выберите "Контроллер SATA"{} и нажмите "Создать жёсткий диск".
\item В верхней панели выберите "Создать".
\item Установите размер жёсткого диска в 3.15 гигабайта (это примерно равно 3000 мебибайтам). 
\item Добавьте новый диск к вашей виртуальной машине (он должен отображаться на вкладке носители). 
\item Запустите виртуальную машину и выполните \texttt{lsblk}, чтобы посмотреть, что создался новый диск размером 3 ГиБ. Скорей всего он будет иметь имя \texttt{sdb}.
\end{itemize}


\item \textbf{Проверяйте имя диска} \\
Во всех задачах задания (после задачи 2) необходимо будет работать с новым, специально созданным диском, а не с диском на котором установлена операционная система. Команды, которые вы будете использовать в этом задании будут изменять или стирать важные структуры на диске. Применение этих команд к диску, на котором установлена операционная система, немедленно приведет к поломке системы.

Скорей всего диск, на котором установлена ОС у вас будет называться \texttt{sda}, а новый пустой диск будет называться \texttt{sdb}, но это необязательно. Более того, имена дисков (хоть это и маловероятно) могут поменяться после перезапуска системы. Поэтому всегда проверяйте к какому диску вы применяете ту или иную команду, используя команду \texttt{lsblk}.




\end{itemize}



\newpage


\subsection{Единицы измерения информации}
Произведите конвертацию одних величин в другие с точностью до трёх знаков после запятой.
%\begin{multicols}{2}
\begin{itemize}
\item 1МБ в байты
\item 1МиБ в байты
\item 1ГиБ в гигабайты
\item 1ГБ в мебибайты
\end{itemize}
%\end{multicols}
Для решения этой задачи создайте файл \texttt{01.txt} с ответами.


\subsection{Просмотр информации о дисках и разделах}
Некоторые из этих команд требуют прав суперпользователя, используйте \texttt{sudo}, чтобы запустить их.
\begin{enumerate}[a.]
\item \textbf{Просмотр информации о дисках и разделах}\\
Просмотрите ваши текущие диски и разделы, используя команду \texttt{lsblk}.

\item \textbf{Просмотр информации о UUID и файловых системах}\\
Просмотрите уникальные идентификаторы дисков и разделов, а также используемые на этих разделах файловые системы. Используйте команду  \texttt{sudo blkid}.

\item \textbf{Просмотр всех смонтированных файловых систем}\\
Просмотрите все смонтированные на данный момент файловые системы, используя команду \texttt{findmnt}. Команда \texttt{findmnt} показывает не только физические разделы и диски, но и виртуальные (\texttt{proc}, \texttt{sysfs}) и другие.

\item \textbf{Просмотр таблицы разделов}\\
Просмотрите подробную информацию о разделах \texttt{parted -l}. Какая таблица разделов используется на вашем основном диске?

\item \textbf{Просмотрите системный файл \texttt{/etc/fstab}}\\
Файл \texttt{/etc/fstab} хранит в себе файловых системах, которые должны быть автоматически смонтированы при загрузке системы. Для просмотра этого файла используйте команду:
\begin{lstlisting}
$ cat /etc/fstab
\end{lstlisting}

\item \textbf{Просмотрите директорию, содержащую файлы устройств}\\
Просмотрите директорию \texttt{/dev}, используя подробный вывод \texttt{ls -la}. Обратите внимание на  файлы блочных устройств, например \texttt{sda} (файл устройства -- диска), \texttt{sda1} (файл устройства -- раздела диска) и другие подобные файлы.

\item \textbf{Просмотрите подробную информацию о диске/разделе}\\
Используйте команду \texttt{file -s} на файлах устройств, чтобы посмотреть подробную информацию о соответствующем диске и разделе.
\begin{lstlisting}
$ sudo file -s /dev/sda
$ sudo file -s /dev/sda1
\end{lstlisting}

\item \textbf{Просмотр использованного места}\\
Просмотрите использованное место на различных разделах, используя команду \texttt{df -h}.

\item \textbf{Просмотр общего размера директорий}\\
Используйте команду \texttt{du}, чтобы просмотреть сколько места на диске занимают директории \texttt{/home} и \texttt{/usr} вместе со всеми внутренними файлами. Сравните полученные числа с тем, что выводит \texttt{ls -l}.

\end{enumerate}


\newpage
\subsection{Разметка диска, создание файловой системы и монтирование}
\begin{enumerate}[a.]
\item \textbf{Новый диск}\\
Найдите как в системе называется новый диск, созданный в части "Предварительные замечания". Далее во всём задании будет предполагаться, что он называется \texttt{sdb} и его файл устройства находится в \texttt{/dev/sdb}.

\item \textbf{Новый раздел}\\
Используйте программу \texttt{parted}, чтобы создать новый раздел на диске \texttt{sdb}. Раздел должен занимать весь диск. Раздел будет иметь имя \texttt{sdb1}, а файл устройства этого раздела будет \texttt{/dev/sdb1}. Посмотрите, что новый раздел создался, используя \texttt{lsblk}.

\item \textbf{Создаём файловую систему}\\
Создайте файловую систему ext4 на разделе \texttt{sdb1}, используя команду \texttt{mkfs}. Используйте \texttt{lsblk -f}, чтобы посмотреть, что в разделе \texttt{sdb1} используется файловая система ext4.

\item \textbf{Монтируем файловую систему}\\
Создайте новую папку \texttt{/mnt/myfs}. Измените владельца этой папки на вашего пользователя, чтобы можно было работать с папкой без использования \texttt{sudo}. Используйте команду \texttt{mount}, чтобы примонтировать файловую систему к директории \texttt{/mnt/myfs}. Используйте команду \texttt{findmnt}, чтобы убедиться, что файловая система была примонтирована.

\item \textbf{Используем файловую систему}\\
Перейдите в директорию \texttt{/mnt/myfs} и создайте там два файла: \texttt{a.txt}, который будет содержать строку \texttt{"Alpaca"} и \texttt{b.txt}, который будет содержать строку \texttt{"Bison"}. Создайте большой пустой файл под названием \texttt{large}, размером 100 МиБ, используя команду:
\begin{lstlisting}
dd if=/dev/zero of=./large bs=1M count=100
\end{lstlisting}
Выполните команду \texttt{df -h}, чтобы убедиться что количество занятого места на разделе увеличилось.

\item \textbf{Перемонтирование в другом месте}\\
Создайте новую директорию \texttt{/home/shared}. Дайте этой папке полные права (\texttt{rwxrwxrwx}), чтобы можно было работать с ней без использования \texttt{sudo}.  Размонтируйте файловую систему с раздела \texttt{sdb1} из папки \texttt{/mnt/myfs} и примонтируйте её к новой директории \texttt{/home/shared}. Зайдите в эту директорию и убедитесь, что все файлы сохранились. 

\item \textbf{Перезагрузка}\\
Перезагрузите виртуальную машину и проверьте файлы в директории \texttt{/home/shared}.
Заново примонтируйте файловую систему в папку \texttt{/home/shared} и убедитесь, что все файлы сохранились.

\item \textbf{Запись в \texttt{/etc/fstab}}\\
Файловая система в \texttt{/home/shared} была примонтирована временно. После перезагрузки системы её придётся монтировать снова. Чтобы указать, что её нужно монтировать автоматически при загрузке, нужно добавить новую запись в \texttt{/etc/fstab}. Этот файл нужно редактировать очень осторожно, так как ошибка в этом файле может привести к тому, что система не запустится.
\begin{itemize}
\item Укажите файловую систему по UUID раздела, на который она установлена. UUID можно найти, используя команду \texttt{sudo blkid}.
\item Для поля \texttt{options} укажите значение \texttt{defaults}.
\item Для полей \texttt{dump} и \texttt{pass} укажите значения 0.
\end{itemize}

\item \textbf{Проверьте, что запись в \texttt{/etc/fstab} корректна}\\
Размонтируйте вашу файловую систему и выполните \texttt{mount -a}. Эта команда проверит запись в \texttt{/etc/fstab} на корректность и, если запись корректна, она смонтирует её. Команда не будет работать, если файловая система уже смонтирована.
\begin{lstlisting}
$ sudo umount /home/shared
$ sudo mount -a
\end{lstlisting}
Перезагрузите виртуальную машину и убедитесь, что файловая система в \texttt{/home/shared} была автоматически примонтирована.

\end{enumerate}

\subsection{Несколько разделов (Таблица разделов MBR)}
Для создания таблицы разделов и удаления/создания разделов в этой задаче используйте программу \texttt{parted}. Размонтируйте файловую систему на \texttt{sdb} и удалите раздел \texttt{sdb1}. \textbf{Удалите запись, соответствующую файловой системе удалённого раздела из \texttt{/etc/fstab}}.

Создайте на диске таблицу разделов \texttt{MBR}. Создайте на диске 4 раздела: 3 основных (primary) и 1 расширенный (extended). Внутри расширенного раздела создайте 2 логических (logical) раздела. Затем создайте в этих разделах файловые системы, используя \texttt{mkfs}, в соответствии со следующей таблицей:
\begin{center}
\begin{tabular}{lll}
 раздел & размер  	& файловая система   \\
 \texttt{sdb1} 	& 1000 MiB 		& xsf   \\
 \texttt{sdb2} 	& 1000 MiB 		& btrfs \\
 \texttt{sdb3} 	& 100 MiB 	& ext4 \\
 \texttt{sdb4} 	& --	    & расширенный раздел \\
 \texttt{sdb5} 	& 100 MiB 	& ext4 \\
 \texttt{sdb6} 	& 800 MiB 	& fat32 \\
\end{tabular}
\end{center}

\begin{itemize}
\item Для btrfs может понадобиться установить пакет \texttt{btrfs-progs}, а для xfs -- \texttt{xfsprogs}.
\item Первый и/или последний раздел может получиться чуть меньше или чуть больше.
\item Проверьте, что созданный диск использует таблицу разделов \texttt{MBR}. Для этого выполните \texttt{sudo parted -l}. В выводе должна присутствовать строка:
\begin{lstlisting}
Disklabel type: msdos
\end{lstlisting}
\item После создания разделов может потребоваться вызвать команду \texttt{partprobe}, чтобы система их увидела.
\item Проверьте, что все разделы были созданы вместе с соответствующими файловыми системами, используя команду \texttt{lsblk -f}.
\end{itemize}


\subsection{Несколько разделов (Таблица разделов GPT)}
Для создания таблицы разделов и удаления/создания разделов в этой задаче используйте программу \texttt{parted}. Создайте на диске \texttt{sdb} таблицу разделов \texttt{GPT} и 5 разделов. Затем создайте в этих разделах файловые системы, используя \texttt{mkfs}, в соответствии со следующей таблицей:
\begin{center}
\begin{tabular}{lll}
 раздел & размер  	& файловая система   \\
 \texttt{sdb1} 	& 1000 MiB 		& xsf   \\
 \texttt{sdb2} 	& 1000 MiB 		& btrfs \\
 \texttt{sdb3} 	& 100 MiB 		& ext4 \\
 \texttt{sdb4} 	& 100 MiB 		& ext4 \\
 \texttt{sdb5} 	& 800 MiB 		& fat32 \\
\end{tabular}
\end{center}

\begin{itemize}
\item Проверьте, что созданный диск использует таблицу разделов \texttt{GPT}. Для этого выполните \texttt{sudo parted -l}. В выводе должна присутствовать строка:
\begin{lstlisting}
Disklabel type: gpt
\end{lstlisting}
\item Проверьте, что все разделы были созданы вместе с соответствующими файловыми системами, используя команду \texttt{lsblk -f}.
\item Примонтируйте эти разделы к директориям \texttt{/mnt/01}, \texttt{/mnt/02}, ... \texttt{/mnt/05}.
\end{itemize}








\subsection{Программа \texttt{dd}}

\begin{enumerate}[(a)]
\item \textbf{Файл из нулевых байт}\\
Создайте файл из нулевых байт размером в 1 кибибайт, используя программу \texttt{dd} и псевдоустройство \texttt{/dev/zero}. Используйте программу \texttt{xxd} или \texttt{hexdump -C}, чтобы просмотреть все байты созданного файла.

\item \textbf{Файл из случайных байт}\\
Создайте файл из случайных байт размером в 500 байт, используя программу \texttt{dd} и псевдоустройство \texttt{/dev/random}. Используйте \texttt{xxd}, чтобы просмотреть все байты созданного файла.


\item \textbf{Обнуление части файла}\\
В созданном файле из предыдущей подзадачи обнулите байты с 100-го до 200-го используя \texttt{dd} и \texttt{/dev/zero}. Просмотрите байты изменённого файла. Используйте опцию \texttt{conv=notrunc}, иначе \texttt{dd} будет обрезать файл после последнего записанного байта.

\item \textbf{Патч}\\
В файле из предыдущей задачи задайте байты с индексами от 50 до 55 значениями \texttt{A1 B2 C3 D4 E5 F6}. Остальные байты файла измениться не должны, размер файла должен остаться прежним. Используйте \texttt{printf} и \texttt{dd}. Просмотрите байты изменённого файла.

\item \textbf{Текстовый файл}\\
Создайте файл \texttt{a.txt}, содержащий фразу \texttt{Sapere Aude}.
\begin{enumerate}
\item Используйте \texttt{dd}, чтобы вывести содержимое этого файла на экран (\texttt{stdout}).
\item Используйте \texttt{dd}, чтобы вывести первые 6 символов файла на экран.
\item Используйте \texttt{dd}, чтобы вывести символы начиная с 8-го и до конца файла на экран.
\item Используйте \texttt{dd}, чтобы скопировать файл \texttt{a.txt} в файл \texttt{b.txt}.
\item Используйте \texttt{dd}, чтобы скопировать файл \texttt{a.txt} в файл \texttt{c.txt}, переведя все строчные буквы в прописные.
\end{enumerate}

\item \textbf{Байты диска}\\
Используйте \texttt{dd} и \texttt{xxd}, чтобы просмотреть первые 1000 байт диска \texttt{/dev/sdb}.

\item \textbf{Копирование файловой системы} ~~ \textbf{Будьте осторожны при указании раздела в этой подзадаче! Если вы укажите системный раздел, то система сломается.}\\
В файловой системе на разделе \texttt{sdb3} создайте 100 файлов с именами \texttt{file000}, \texttt{file001} ... \texttt{file100}. Размонтируйте файловые системы на разделах \texttt{sdb3} и \texttt{sdb4}. Скопируйте содержимое файловой системы на разделе \texttt{sdb3} в раздел \texttt{sdb4}, используя \texttt{dd}. Используйте значение \texttt{bs} как минимум 1 МиБ для ускорения копирования. Используйте опцию \texttt{status=progress} для отслеживания хода выполнения. Примонтируйте файловые системы обратно и проверьте содержимое раздела \texttt{sdb4}.
\end{enumerate}

\subsection{Блоки хранения данных}
Дайте определения следующим понятиям, указав для каждого типичные размеры.
\begin{enumerate}[(a)]
\item Сектор (sector) -- в контексте жёстких дисков (HDD).
\item Страница (page) -- в контексте твердотельных накопителей (SSD).
\item Блок (block) -- в контексте твердотельных накопителей (SSD).
\item Блок (block) -- в контексте файловых систем (например, ext4).
\item Кластер (cluster) -- в контексте файловых систем FAT и NTFS.
\item Страница (page) -- в контексте оперативной и виртуальной памяти.
\end{enumerate}
Для решения этой задачи создайте файл \texttt{07.txt} и напишите в этом файле все определения.


\subsection{Задача на ссылки}
\begin{enumerate}[(a)]
\item Создайте новый файл \texttt{file.txt}, содержащий строку \texttt{"I am file"}.
\item Используйте \texttt{ls -li} чтобы посмотреть на значение номера \texttt{inode} файла. Также обратите внимание на число 1, следующее после прав. Это количество жёстких ссылок на этот файл.
\item Используйте программу \texttt{stat}, чтобы посмотреть метаинформацию о заданном файле. Обратите внимание на поля \texttt{Inode} и \texttt{Links}.
\item Создайте жёсткую ссылку (\textit{hard link}) на \texttt{file.txt} по имени \texttt{hl.txt}.
\item Создайте мягкую ссылку (\textit{soft link}) на \texttt{file.txt} по имени \texttt{sl.txt}. Мягкая ссылка также называется как символическая ссылка или симлинк.
\item Используйте \texttt{ls -li}, чтобы посмотреть значения номеров inode для файла и двух ссылок. Также обратите на значение количества жёстких ссылок. Чему равны эти значения?
\item Используйте \texttt{stat} что посмотреть номера inode и количество жёстких ссылок для начального файла и двух ссылок. 
\item Допишите в файл строку \texttt{"From hard link"}, используя жёсткую ссылку \texttt{hl.txt}.
\begin{lstlisting}
$ echo "From hard link" >> hl.txt
\end{lstlisting}
Что содержит файл после этой операции?
\item Допишите в файл строку \texttt{"From soft link"}, используя мягкую ссылку \texttt{sl.txt}. Что содержит файл после этой операции?
\item Измените права жёсткой ссылки на 111. 
\begin{lstlisting}
$ chmod 111 hl.txt
\end{lstlisting}
Как изменились права начального файла и ссылок после этого?
\item Измените права мягкой ссылки на 222. 
\begin{lstlisting}
$ chmod 222 sl.txt
\end{lstlisting}
Как изменились права начального файла и ссылок после этого?
\item Установите права файла, мягкой и жёсткой ссылки на 777.
\item Удалите изначальный файл \texttt{file.txt}.
\item Можно ли теперь прочитать файл по жёсткой ссылке?
\item Можно ли теперь прочитать файл по мягкой ссылке?
\item Создайте новый файл с тем же именем \texttt{file.txt}, содержащий \texttt{"Second file"}.
\item Что будет, если напечатать файл через жёсткую ссылку \texttt{hl.txt}?
\item Что будет, если напечатать файл через мягкую ссылку \texttt{sl.txt}?
\end{enumerate}


\subsection{Поиск ссылок}
\begin{enumerate}[(a)]
\item Используйте команду \texttt{find} чтобы найти в системе все обычные файлы, на которых существует хотя бы 2 жёстких ссылки.
\item Выберете один из файлов из предыдущей подзадачи и найдите все жёсткие ссылки для этого файла.
\item Примените команду \texttt{diff} для двух жёстких ссылок на один файл, чтобы убедиться, что эти файлы побайтово совпадают.
\item Используйте команды \texttt{xxd} и \texttt{head}, чтобы посмотреть первые байты этих файлов и также убедиться, что они совпадают.
\item Используйте команду \texttt{find}, чтобы найти все мягкие ссылки в директории \texttt{/etc/} рекурсивно, вызовите команду \texttt{ls -l} для каждой из этих ссылок, чтобы понять куда они указывают.

\item Просмотрите директорию \texttt{/usr/bin}, используя \texttt{ls}, чтобы посмотреть какие файлы в этой директории являются мягкими ссылками.

\item Создайте символическую ссылку \texttt{/usr/bin/dog} на файл \texttt{/usr/bin/cat}. Директория \texttt{/usr/bin} является системной директорией, в которой происходит поиск исполняемых файлов. Если туда поместить исполняемый файл или ссылку на него, то его можно будет вызывать без прописывания пути. Используйте новую команду \texttt{dog} для просмотра файлов.
\end{enumerate}


\subsection{Свойства файловых систем}
\begin{enumerate}[(a)]
\item \textbf{Ссылки в разных файловых системах}
\begin{itemize}
\item Зайдите в директорию \texttt{/mnt/01}, в которой содержатся файлы ФС xfs.
\item Создайте файл \texttt{a.txt} с содержимым \texttt{"File from xfs"}.
\item В этой же директории создайте жёсткую и мягкую ссылку на этот файл. Работают ли эти ссылки?
\item Перейдите в директорию \texttt{/mnt/02}, в которой содержатся файлы файловой системы btrfs.
\item Создайте в директории \texttt{/mnt/02} жёсткую и мягкую ссылку на \texttt{/mnt/01/a.txt}. Работают ли ссылки?
\end{itemize}


\item \textbf{FAT}
\begin{itemize}
\item Зайдите в директорию \texttt{/mnt/05}, в которой содержатся файлы ФС FAT32 и создайте файл \texttt{b.txt}.
\item Можно ли создать в этой директории жёсткую и мягкую ссылку на файл \texttt{b.txt}?
\item Просмотрите права файла \texttt{b.txt}. Попробуйте изменить права файлов, используя \texttt{sudo chmod}. Получится ли у вас это сделать и, если нет, то почему?
\item Просмотрите владельца и группу владельца файлов. Попробуйте их изменить, используя команды \texttt{sudo chown} и \texttt{sudo chgrp}. Получится ли у вас это сделать и, если нет, то почему?
\iffalse
\item Используйте \texttt{stat}, чтобы распечатать метаданные файлов \texttt{a.txt} и \texttt{alpha}.
\fi
\end{itemize}

\item \textbf{Переполнение таблицы inode-ов в файловой системе ext4}
\begin{itemize}
\item Зайдите в директорию \texttt{/mnt/03}, в которой содержатся файлы ФС ext4 размером 100 МиБ.
\item Просмотрите и запомните количество свободного места в этой файловой системе, используя \texttt{df -h}.
\item Просмотрите и запомните количество свободных inode в этой файловой системе, используя \texttt{df -i}.
\item Попытайтесь создать как можно больше пустых файлов на этой файловой системе, например 50000 файлов с именами \texttt{file00000}, \texttt{file00001} и т. д. Создавайте пустые файлы в этом разделе пока команда не выдаст ошибку "No space left on device". Используйте команду \texttt{touch} вместе с brace expansion.
\item Просмотрите количество свободного места в этой файловой системе, используя \texttt{df -h}.
\item Просмотрите количество свободных inode в этой файловой системе, используя \texttt{df -i}.
\item Сравните значение свободного места и количества свободных inode с тем, что было до создания файлов.
\item Используйте \texttt{du -sh}, чтобы найти размер папки \texttt{/mnt/03} и убедиться, что он меньше 100 МиБ.
\end{itemize}

\end{enumerate}


\subsection{Какая информация хранится в inode?}
В файловой системе \texttt{ext4} inode каждого файла хранится в специальной таблице inode-ов. Обычно inode занимает 256 байт и содержит множество полей, также называемых метаданными. Напишите, какие поля содержатся в inode для разных типов файлов.
\begin{enumerate}[(a)]
\item Поля inode общие для всех типов файлов.
\item Поля inode, характерные для обычных файлов. inode для обычных файлов содержит указатели на блоки данных. Сколько указателей на блоки данных содержит такой inode?  Что хранится в этих блоках данных?
\item Поля inode, характерные для директорий. inode для файлов-директорий содержит указатели на блоки данных. Что и в каком формате хранится в этих блоках?
\item Поля inode, характерные для мягких ссылок. inode для таких файлов может содержать указатели на блоки данных. В каких случаях мягкая ссылка может использовать блоки данных и что хранится в этих блоках?
\item Поля inode, характерные для файлов блочных и символьных устройств.
\end{enumerate}
Для решения этой задачи создайте файл \texttt{11.txt} с ответами на вопросы.

\subsection{Расширение раздела и файловой системы}
После выполнения предыдущих задач, раздел \texttt{sdb3} должен быть полностью заполнен пустыми файлами. Создать новый файл без удаления существующих на этом разделе не получится. Используйте \texttt{parted}, чтобы удалить раздел \texttt{sdb4}, следующий после раздела \texttt{sdb3}. Расширьте раздел \texttt{sdb3} и соответствующую файловую систему. Теперь в эту файловую систему, помимо блоков данных, добавились inode-таблицы. Создайте новый файл в этой файловой системе. Посмотрите на процент свободных inode в расширенной файловой системе.

\iffalse
\newpage
~
\newpage

\subsection*{Идеи}
\begin{itemize}
\item Вопросы по ext2
\begin{itemize}
\item Сколько памяти может занимать 1 файл в ext2 (block=4kb, размер указателя = 4 или 8).
\end{itemize}
\item Ограничение на размер файла в ext4 - 256 байт.
\end{itemize}


\fi

\end{document}
