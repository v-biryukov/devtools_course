\documentclass{article}
\usepackage[english,russian]{babel}
\usepackage{textcomp}
\usepackage{geometry}
  \geometry{left=2cm}
  \geometry{right=1.5cm}
  \geometry{top=1.5cm}
  \geometry{bottom=2cm}
\usepackage{tikz}
\usepackage{multicol}
\usepackage{hyperref}
\usepackage{listings}
\usepackage{pmboxdraw}
\usepackage{fancyvrb}
\usepackage[shortlabels]{enumitem}
\pagenumbering{gobble}

\lstdefinestyle{csMiptCStyle}{
  language=C,
  basicstyle=\linespread{1.1}\ttfamily,
  columns=fixed,
  fontadjust=true,
  basewidth=0.5em,
  keywordstyle=\color{blue}\bfseries,
  commentstyle=\color{gray},
  texcl=true,
  stringstyle=\ttfamily\color{orange!50!black},
  showstringspaces=false,
  numbersep=5pt,
  numberstyle=\tiny\color{black},
  numberfirstline=true,
  stepnumber=1,      
  numbersep=10pt,
  backgroundcolor=\color{white},
  showstringspaces=false,
  captionpos=b,
  breaklines=true
  breakatwhitespace=true,
  xleftmargin=.2in,
  extendedchars=\true,
  keepspaces = true,
  tabsize=4,
  upquote=true,
}


\lstdefinestyle{csMiptCLinesStyle}{
  style=csMiptCStyle,
  frame=lines,
}

\lstdefinestyle{csMiptCBorderStyle}{
  style=csMiptCStyle,
  framexleftmargin=5mm, 
  frame=shadowbox, 
  rulesepcolor=\color{gray}
}


\lstdefinestyle{csMiptBash}{
  	style=csMiptCStyle,
	breaklines=true,
	frame=tb,
	language=bash,
	breakatwhitespace=true,
	alsoletter={*()"'0123456789.},
	alsoother={\{\=\}},
	basicstyle={\ttfamily},
	keywordstyle={\bfseries},
	literate={{=}{{{=}}}1},
	prebreak={\textbackslash},
	sensitive=true,
	stepnumber=1,
	tabsize=4,
	morekeywords={echo, function},
	otherkeywords={-, \{, \}},
	literate={\$\{}{{{{\bfseries{}\$\{}}}}2,
	upquote=true,
	frame=none
}

\lstset{style=csMiptBash}
\lstset{
        literate={~}{{\raisebox{0.5ex}{\texttildelow}}}{1}
}

\renewcommand{\thesubsection}{\arabic{subsection}}
\makeatletter
\def\@seccntformat#1{\@ifundefined{#1@cntformat}%
   {\csname the#1\endcsname\quad}
   {\csname #1@cntformat\endcsname}}
\newcommand\section@cntformat{}     
\newcommand\subsection@cntformat{Задача \thesubsection.\space} 
\newcommand\subsubsection@cntformat{\thesubsubsection.\space}
\makeatother



\begin{document}
\title{Семинар \#6: Диски и файловые системы. Практика. \vspace{-5ex}}\date{}\maketitle
\subsection*{Как сдавать задачи}
Для сдачи ДЗ вам нужно создать репозиторий на GitLab (если он ещё не создан) под названием \texttt{devtools-homework}. Структура репозитория должна иметь вид:
\begin{center}
\begin{BVerbatim}
├── seminar6_disks_and_filesystems/
│   ├── 01.txt
│   ├── 02.sh
│   └── ...
└── ...
\end{BVerbatim}
\end{center}
Для каждой задачи нужно создать 1 файл решения с расширением \texttt{.sh}. Подзадачи внутри каждого из файлов нужно оформлять в следующем формате:
\begin{lstlisting}
# Subtask a
lsblk
# Subtask b
...
\end{lstlisting}
Для каждой подзадачи нужно прописать все команды, которые исполняются в ходе выполнения этой подзадачи.\\



\subsection*{Предварительные замечания}
\begin{itemize}
\item \textbf{Важно! Резервное клонирование ВМ.} \\
При выполнении данного задания придётся работать от имени суперпользователя (\texttt{root}), используя команды \texttt{su} и \texttt{sudo}. Выполняя такие команды, можно случайно сломать систему. Поэтому перед выполнением данного задания на всякий случай клонируйте вашу виртуальную машину. В VirtualBox для этого нужно нажать правой кнопкой мыши на виртуальной машине и выбрать "Клонировать".



\item \textbf{Создание диска для виртуальной машины.} \\
Это задание нужно выполнять в виртуальной машине. Для выполнения этого задания вам понадобиться создать новый виртуальный диск. В \texttt{Virtual Box} это можно сделать следующим образом:
\begin{itemize}
\item Выключите виртуальную машину, если она включена.
\item Выберите вашу виртуальную машину и нажмите настроить.
\item Выберите вкладку "Носители".
\item Выберите "Контроллер SATA"{} и нажмите "Создать жёсткий диск".
\item В верхней панели выберите "Создать".
\item Установите размер жёсткого диска в 3 гибибайта. 
\item Добавьте новый диск к вашей виртуальной машине (он должен отображаться на вкладке носители). 
\item Запустите виртуальную машину и выполните \texttt{lsblk}, чтобы посмотреть, что создался новый диск размером 3 ГиБ. Скорей всего он будет иметь имя \texttt{sdb}.
\end{itemize}


\item \textbf{Проверяйте имя диска} \\
Во всех задачах задания (кроме задачи 2) необходимо будет работать с новыми, специально созданными дисками, а не с диском на котором установлена операционная система. Команды, которые вы будете использовать в этом задании будут изменять или стирать важные структуры на диске. Применение этих команд к диску, на котором установлена операционная система, немедленно приведет к поломке системы.

Скорей всего диск, на котором установлена ОС у вас будет называться \texttt{sda}, а новый пустой диск будет называться \texttt{sdb}, но это необязательно. Более того, имена дисков (хоть это и маловероятно) могут поменяться после перезапуска системы. Поэтому всегда проверяйте к какому диску вы применяете ту или иную команду, используя команду \texttt{lsblk}.




\end{itemize}



\newpage


\subsection{Единицы измерения информации}
Произведите конвертацию одних величин в другие с точностью до трёх знаков после запятой.
%\begin{multicols}{2}
\begin{itemize}
\item 1МБ в байты
\item 1МиБ в байты
\item 1ГиБ в гигабайты
\item 1ГБ в мебибайты
\end{itemize}
%\end{multicols}


\subsection{Просмотр информации о дисках и разделах}
Некоторые из этих команд требуют прав суперпользователя, используйте \texttt{sudo}, чтобы запустить их.
\begin{enumerate}[a.]
\item \textbf{Просмотр информации о дисках и разделах}\\
Просмотрите ваши текущие диски и разделы, используя команду \texttt{lsblk}.

\item \textbf{Просмотр информации о UUID и файловых системах}\\
Просмотрите уникальные идентификаторы дисков и разделов, а также используемые на этих разделах файловые системы. Используйте команду  \texttt{sudo blkid}.

\item \textbf{Просмотр всех смонтированных файловых систем}\\
Просмотрите все смонтированные на данный момент файловые системы, используя команду \texttt{findmnt}. Команда \texttt{findmnt} показывает не только физические разделы и диски, но и виртуальные (\texttt{proc}, \texttt{sysfs}) и другие.

\item \textbf{Просмотр таблицы разделов}\\
Просмотрите подробную информацию о разделах \texttt{parted -l}.

\item \textbf{Просмотрите системный файл \texttt{/etc/fstab}}\\
Файл \texttt{/etc/fstab} хранит в себе файловых системах, которые должны быть автоматически смонтированы при загрузке системы. Для просмотра этого файла используйте команду:
\begin{lstlisting}
$ cat /etc/fstab
\end{lstlisting}

\item \textbf{Просмотрите директорию, содержащую файлы устройств}\\
Просмотрите директорию \texttt{/dev}, используя подробный вывод \texttt{ls -la}. Обратите внимание на  файлы блочных устройств, например \texttt{sda} (файл устройства -- диска), \texttt{sda1} (файл устройства -- раздела диска) и другие подобные файлы.

\item \textbf{Просмотрите подробную информацию о диске/разделе}\\
Используйте команду \texttt{file -s} на файлах устройств, чтобы посмотреть подробную информацию о соответствующем диске и разделе.
\begin{lstlisting}
$ sudo file -s /dev/sda
$ sudo file -s /dev/sda1
\end{lstlisting}

\item \textbf{Просмотр использованного места}\\
Просмотрите использованное место на различных разделах, используя команду \texttt{df -h}.

\item \textbf{Просмотр общего размера директорий}\\
Используйте команду \texttt{du}, чтобы просмотреть сколько места на диске занимают директории \texttt{/home} и \texttt{/usr} вместе со всеми внутренними файлами. Сравните полученные числа с тем, что выводит \texttt{ls -l}.

\end{enumerate}


\newpage
\subsection{Разметка диска, создание файловой системы и монтирование}
\begin{enumerate}[a.]
\item \textbf{Новый диск}\\
Найдите как в системе называется новый диск, созданный в части "Предварительные замечания". Далее будет предполагаться, что он называется \texttt{sdb} и его файл устройства находится в \texttt{/dev/sdb}.

\item \textbf{Новый раздел}\\
Используйте программу \texttt{parted}, чтобы создать новый раздел на диске \texttt{sdb}. Раздел должен занимать весь диск. Раздел будет иметь имя \texttt{sdb1}, а файл устройства этого раздела будет \texttt{/dev/sdb1}. Посмотрите, что новый раздел создался, используя \texttt{lsblk}.

\item \textbf{Создаём файловую систему}\\
Создайте файловую систему ext4 на разделе \texttt{sdb1}, используя команду \texttt{mkfs}. Используйте \texttt{lsblk -f}, чтобы посмотреть, что в разделе \texttt{sdb1} используется файловая система ext4.

\item \textbf{Монтируем файловую систему}\\
Создайте новую папку \texttt{/mnt/myfs}. Измените владельца этой папки на вашего пользователя, чтобы можно было работать с папкой без использования \texttt{sudo}. Используйте команду \texttt{mount}, чтобы примонтировать файловую систему к директории \texttt{/mnt/myfs}. Используйте команду \texttt{findmnt}, чтобы убедиться, что файловая система была примонтирована.

\item \textbf{Используем файловую систему}\\
Перейдите в директорию \texttt{/mnt/myfs} и создайте там два файла: \texttt{a.txt}, который будет содержать строку \texttt{"Alpaca"} и \texttt{b.txt}, который будет содержать строку \texttt{"Bison"}. Создайте большой пустой файл под названием \texttt{large}, размером 100 Мб, используя команду:
\begin{lstlisting}
dd if=/dev/zero of=./large bs=1M count=100
\end{lstlisting}
Выполните команду \texttt{df -h}, чтобы убедиться что количество занятого места на разделе увеличилось.

\item \textbf{Перемонтирование в другом месте}\\
Создайте новую директорию \texttt{/home/shared}. Дайте этой папке полные права (\texttt{rwxrwxrwx}), чтобы можно было работать с ней без использования \texttt{sudo}.  Размонтируйте файловую систему с раздела \texttt{sdb1} из папки \texttt{/mnt/myfs} и примонтируйте её к новой директории \texttt{/home/shared}. Зайдите в эту директорию и убедитесь, что все файлы сохранились. 

\item \textbf{Перезагрузка}\\
Перезагрузите виртуальную машину и проверьте файлы в директории \texttt{/home/shared}.
Заново примонтируйте файловую систему в папку \texttt{/home/shared} и убедитесь, что все файлы сохранились.

\item \textbf{Запись в \texttt{/etc/fstab}}\\
Файловая система в \texttt{/home/shared} была примонтирована временно. После перезагрузки системы её придётся монтировать снова. Чтобы указать, что эту систему нужно монтировать автоматически при загрузке системы, нужно добавить новую запись в \texttt{/etc/fstab}. Этот файл нужно редактировать очень осторожно, так как ошибка в этом файле может привести к тому, что система не запустится.
\begin{itemize}
\item Укажите файловую систему по UUID раздела, на который она установлена. UUID можно найти, используя команду \texttt{sudo blkid}.
\item Для поля \texttt{options} укажите значение \texttt{defaults}.
\item Для полей \texttt{dump} и \texttt{pass} укажите значение 0.
\end{itemize}

\item \textbf{Проверьте, что запись в \texttt{/etc/fstab} корректна}\\
Размонтируйте вашу файловую систему и выполните \texttt{mount -a}. Эта команда проверит запись в \texttt{/etc/fstab} на корректность и, если запись корректна, она смонтирует её. Команда не будет работать, если файловая система уже смонтирована.
\begin{lstlisting}
$ sudo umount /home/shared
$ sudo mount -a
\end{lstlisting}
Перезагрузите виртуальную машину и убедитесь, что файловая система в \texttt{/home/shared} была автоматически примонтирована.

\end{enumerate}

\subsection{Несколько разделов (Таблица разделов MBR)}
Для создания таблицы разделов и удаления/создания разделов в этой задаче используйте программу \texttt{parted}. Размонтируйте файловую систему на \texttt{sdb} и удалите раздел \texttt{sdb1}. Создайте на диске таблицу разделов \texttt{MBR}. Создайте на диске 4 раздела: 3 основных (primary) и 1 расширенный (extended). Внутри расширенного раздела создайте 2 логических (logical) раздела. Затем создайте в этих разделах файловые системы, используя \texttt{mkfs}, в соответствии со следующей таблицей:
\begin{center}
\begin{tabular}{lll}
 раздел & размер  	& файловая система   \\
 \texttt{sdb1} 	& 1000 MiB 		& xsf   \\
 \texttt{sdb2} 	& 1000 MiB 		& btrfs \\
 \texttt{sdb3} 	& 100 MiB 	& ext4 \\
 \texttt{sdb4} 	& --	    & расширенный раздел \\
 \texttt{sdb5} 	& 100 MiB 	& ext4 \\
 \texttt{sdb6} 	& 800 MiB 	& fat32 \\
\end{tabular}
\end{center}

\begin{itemize}
\item Для \texttt{btrfs} может понадобиться установить пакет \texttt{btrfs-progs}.
\item Последний раздел может получиться чуть меньше или больше.
\item Проверьте, что созданный диск использует таблицу разделов \texttt{MBR}. Для этого выполните \texttt{sudo parted -l}. В выводе должна присутствовать строка:
\begin{lstlisting}
Disklabel type: msdos
\end{lstlisting}
\item Проверьте, что все разделы были созданы вместе с соответствующими файловыми системами, используя команду \texttt{lsblk -f}.
\end{itemize}


\subsection{Несколько разделов (Таблица разделов GPT)}
Для создания таблицы разделов и удаления/создания разделов в этой задаче используйте программу \texttt{parted}. Создайте на диске \texttt{sdb} таблицу разделов \texttt{GPT} и 5 разделов. Затем создайте в этих разделах файловые системы, используя \texttt{mkfs}, в соответствии со следующей таблицей:
\begin{center}
\begin{tabular}{lll}
 раздел & размер  	& файловая система   \\
 \texttt{sdb1} 	& 1000 MiB 		& xsf   \\
 \texttt{sdb2} 	& 1000 MiB 		& btrfs \\
 \texttt{sdb3} 	& 100 MiB 		& ext4 \\
 \texttt{sdb4} 	& 100 MiB 		& ext4 \\
 \texttt{sdb5} 	& 800 MiB 		& fat32 \\
\end{tabular}
\end{center}

\begin{itemize}
\item Проверьте, что созданный диск использует таблицу разделов \texttt{GPT}. Для этого выполните \texttt{sudo parted -l}. В выводе должна присутствовать строка:
\begin{lstlisting}
Disklabel type: gpt
\end{lstlisting}
\item Проверьте, что все разделы были созданы вместе с соответствующими файловыми системами, используя команду \texttt{lsblk -f}.
\item Примонтируйте эти разделы к директориям \texttt{/mnt/01}, \texttt{/mnt/02}, ... \texttt{/mnt/05}.
\end{itemize}




\subsection{Свойства файловых систем}
\begin{enumerate}[(a)]
\item \textbf{FAT}
\begin{itemize}
\item Зайдите в директорию \texttt{/mnt/05}, в которой содержатся файлы ФС FAT32 размером 800 МиБ.
\item Создайте в этой директории файл \texttt{a.txt} и директорию \texttt{alpha}.
\item Просмотрите права этих файлов. Попробуйте изменить права файлов, используя \texttt{sudo chmod}. Получится ли у вас это сделать и, если нет, то почему?
\item Просмотрите владельца и группу владельца файлов. Попробуйте их изменить, используя команды \texttt{sudo chown} и \texttt{sudo chgrp}. Получится ли у вас это сделать и, если нет, то почему?
\iffalse
\item Используйте \texttt{stat}, чтобы распечатать метаданные файлов \texttt{a.txt} и \texttt{alpha}.
\fi
\end{itemize}

\item \textbf{Переполнение таблицы inod-ов}
\begin{itemize}
\item Зайдите в директорию \texttt{/mnt/03}, в которой содержатся файлы ФС ext4 размером 100 МиБ.
\item Просмотрите и запомните количество свободного места в этой файловой системе, используя \texttt{df -h}.
\item Просмотрите и запомните количество свободных inode в этой файловой системе, используя \texttt{df -i}.
\item Попытайтесь создать как можно больше пустых файлов на этой файловой системе, например 50000 файлов с именами \texttt{file00000}, \texttt{file00001} и т. д. Создавайте пустые файлы в этом разделе пока команда не выдаст ошибку "No space left on device". Используйте команду \texttt{touch} вместе с brace expansion.
\item Просмотрите количество свободного места в этой файловой системе, используя \texttt{df -h}.
\item Просмотрите количество свободных inode в этой файловой системе, используя \texttt{df -i}.
\item Используйте \texttt{du -sh}, чтобы найти размер папки \texttt{/mnt/03} и убедиться, что он меньше 100 МиБ.
\end{itemize}



\end{enumerate}



\subsection{Программа \texttt{dd}}

\begin{enumerate}[a.]
\item \textbf{Файл из нулевых байт}\\
Создайте файл из нулевых байт размером в 10 килобайт, используя программу \texttt{dd} и псевдоустройство \texttt{/dev/zero}. Используйте программу \texttt{xxd}, чтобы просмотреть все байты созданного файла.

\item \textbf{Файл из случайных байт}\\
Создайте файл из случайных байт размером в 10 килобайт, используя программу \texttt{dd} и псевдоустройство \texttt{/dev/urandom}. Используйте программу \texttt{xxd}, чтобы просмотреть все байты созданного файла.

\item \textbf{Вывод на экран}\\
Создайте файл \texttt{a.txt}, содержащий фразу \texttt{Sapere Aude}.
\begin{enumerate}
\item Используйте программу \texttt{dd}, чтобы вывести содержимое этого файла на экран (\texttt{stdout}).
\item Используйте \texttt{dd}, чтобы вывести первые 6 символов файла на экран.
\end{enumerate}

\item \textbf{Копирование файла}\\
Используйте программу \texttt{dd}, чтобы скопировать файл \texttt{a.txt} в файл \texttt{b.txt}.
\end{enumerate}

\subsection{Блоки хранения данных}
Дайте определения следующим понятиям, указав для каждого типичные размеры.
\begin{enumerate}[(a)]
\item Сектор (sector) -- в контексте жёстких дисков (HDD).
\item Страница (page) -- в контексте твердотельных накопителей (SSD).
\item Блок (block) -- в контексте твердотельных накопителей (SSD).
\item Блок (block) -- в контексте файловых систем (например, ext4).
\item Кластер (cluster) -- в контексте файловых систем FAT и NTFS.
\item Страница (page) -- в контексте оперативной и виртуальной памяти.
\end{enumerate}



\subsection{Задача на ссылки}

\subsection{Разные типы файлов и \texttt{stat}}


\subsection{Какая информация хранится в inode}

\subsection{Расширение раздела}



\iffalse
\newpage
~
\newpage

\subsection*{Идеи}
\begin{itemize}
\item Вопросы по ext2
\begin{itemize}
\item Сколько памяти может занимать 1 файл в ext2 (block=4kb, размер указателя = 4 или 8).
\end{itemize}
\end{itemize}


\fi

\end{document}
